\documentclass[ipa,twoside]{report} 

\usepackage[spanish]{babel}
\usepackage[phonetic]{lexikon}
\usepackage[utf8]{inputenc}
\usepackage{fullpage}
\usepackage{multicol}
\usepackage[colorlinks=true]{hyperref}
\usepackage{fancyhdr}
\usepackage{hyphenat}
\author{Comunidad Lojbánica}
\title{Diccionario\\Lojban - Castellano\\Castellano - Lojban}

\begin{document}

\def\/{\discretionary{.}{}{.}}


\setlength{\leftfield}{0.1\textwidth}
\setlength{\rightfield}{0.3\textwidth}
%\setcounter{secnumdepth}{-2} 
\fancypagestyle{plain}{%
	\fancyhf{}%
	\fancyfoot[LE,RO]{\thepage}%
	\renewcommand{\headrulewidth}{0pt}
	\renewcommand{\footrulewidth}{0pt}
}

\maketitle

\tableofcontents

\addcontentsline{toc}{chapter}{Historial de versiones}
\chapter*{Historial de versiones}
\label{cha:historial}
Esta es la versión {\bfseries v0.01}, sólo para mí :)

\addcontentsline{toc}{chapter}{Cómo usar este diccionario}
\chapter*{Cómo usar este diccionario}
\label{cha:howto}
En el capítulo \textbf{Lojban - Castellano}, encontrará la lista con todos los \textsl{valsi} ordenados alfabéticamente (en el alfabeto de Lojban). Se parecerán a:

\begin{center}
\begin{minipage}[t]{0.55\textwidth}
	\dictchar{cy.}
	\dictentry{ce}{/she/}{c}{\textit{cec}\\\textbf{JOI}} %cambiar!!!
 	{}{}{}{conector no lógico: enlace de conjunto, no ordenado; formando un conjunto}
\end{minipage}
\end{center}

En la columna de la izquierda, se encuentra el \textsl{valsi} que se definirá. Estará seguido por una de las siguientes letras, que indicarán el tipo de palabra, según la siguiente lista:

\begin{tabular}{ l l }
  \textit{g} & \textbf{\emph{gismu}}  \\
  \textit{f} & \textbf{\emph{fu'ivla}}  \\
  \textit{l} & \textbf{\emph{lujvo}}  \\
  \textit{c} & \textbf{\emph{cmavo}} \\
  \textit{n} & \textbf{\emph{cmene}} \\
  \textit{cc} & \textbf{\emph{clúster de cmavo}} \\
  \textit{eg} & \textbf{\emph{experimental gismu}} \\
\end{tabular}

Abajo del \textsl{valsi}, entre barras (``/'') se encontrará la pronunciación de la palabra, escrita en la representación \textbf{IPA}. Luego, en \textit{itálicas}, puede aparecer el rafsi asociado al \textsl{valsi}. Más abajo, en \textbf{negrita}, puede aparecer el \textsl{selma'o} al que pertenece el \textsl{valsi} (si se trata de un \textsl{cmavo}).

En la columna de la derecha está la definición en castellano, con su correspondiente esctructura de lugares. Debajo de ésta, puede aparecer en \textit{itálicas} un comentario acerca de la definición.

\newpage

\setlength{\headheight}{25pt}
\fancyhead{}				          % empty out the header
\fancyfoot{}        				  % empty out the footer
\fancyhead[LE,LO]{\rightmark} % left side, odd and even pages
\fancyhead[RE,RO]{\leftmark}  % right side, odd and even pages
\fancyfoot[LE,RO]{\thepage}   % left side even, right side odd
\setlength{\columnseprule}{0.3pt}
\pagestyle{fancy}
\thispagestyle{plain}

\chapter*{Lojban - Castellano}
\addcontentsline{toc}{chapter}{Lojban - Castellano}

\label{cha:lojcas}
\begin{multicols}{2}

\input{jbocas}
  
\end{multicols}

\newpage 
\setlength{\leftfield}{0.25\textwidth}
\setlength{\rightfield}{0.25\textwidth}

\thispagestyle{plain}
\chapter*{Castellano - Lojban}
\addcontentsline{toc}{chapter}{Castellano - Lojban}

\label{cha:casloj}
\begin{multicols}{2}

\dictchar{\#}\phantomsection \addcontentsline{toc}{section}{\#}
\dictentry{'}{}{}{}
{\hyperlink{val:yhy}{y'y}}{}{}{}

\dictentry{\&}{}{}{}
{\hyperlink{val:joibu}{joibu}}{}{}{}

\dictchar{123}\phantomsection\addcontentsline{toc}{section}{123}
\dictentry{1}{}{}{}
{\hyperlink{val:pa}{pa}}{}{}{}

\dictentry{1,000}{}{}{}
{\hyperlink{val:pakiho}{paki'o}}{}{}{}

\dictentry{10}{}{}{}
{\hyperlink{val:pano}{pano}}{}{}{}

\dictentry{100}{}{}{}
{\hyperlink{val:panono}{panono}}{}{}{}

\dictentry{11}{}{}{}
{\hyperlink{val:papa}{papa}}{}{}{}

\dictentry{12}{}{}{}
{\hyperlink{val:pare}{pare}}{}{}{}

\dictentry{13}{}{}{}
{\hyperlink{val:paci}{paci}}{}{}{}

\dictentry{14}{}{}{}
{\hyperlink{val:pavo}{pavo}}{}{}{}

\dictentry{15}{}{}{}
{\hyperlink{val:pamu}{pamu}}{}{}{}

\dictentry{16}{}{}{}
{\hyperlink{val:paxa}{paxa}}{}{}{}

\dictentry{17}{}{}{}
{\hyperlink{val:paze}{paze}}{}{}{}

\dictentry{18}{}{}{}
{\hyperlink{val:pabi}{pabi}}{}{}{}

\dictentry{19}{}{}{}
{\hyperlink{val:paso}{paso}}{}{}{}

\dictentry{2}{}{}{}
{\hyperlink{val:re}{re}}{}{}{}

\dictentry{2,000}{}{}{}
{\hyperlink{val:rekiho}{reki'o}}{}{}{}

\dictentry{20}{}{}{}
{\hyperlink{val:reno}{reno}}{}{}{}

\dictentry{200}{}{}{}
{\hyperlink{val:renono}{renono}}{}{}{}

\dictentry{2da conversión}{}{}{}
{\hyperlink{val:se}{se}}{}{}{}

\dictentry{3}{}{}{}
{\hyperlink{val:ci}{ci}}{}{}{}

\dictentry{3,000}{}{}{}
{\hyperlink{val:cikiho}{ciki'o}}{}{}{}

\dictentry{30}{}{}{}
{\hyperlink{val:cino}{cino}}{}{}{}

\dictentry{300}{}{}{}
{\hyperlink{val:cinono}{cinono}}{}{}{}

\dictentry{3ra conversión}{}{}{}
{\hyperlink{val:te}{te}}{}{}{}

\dictentry{4}{}{}{}
{\hyperlink{val:vo}{vo}}{}{}{}

\dictentry{4,000}{}{}{}
{\hyperlink{val:vokiho}{voki'o}}{}{}{}

\dictentry{40}{}{}{}
{\hyperlink{val:vono}{vono}}{}{}{}

\dictentry{400}{}{}{}
{\hyperlink{val:vonono}{vonono}}{}{}{}

\dictentry{4ta conversión}{}{}{}
{\hyperlink{val:ve}{ve}}{}{}{}

\dictentry{5}{}{}{}
{\hyperlink{val:mu}{mu}}{}{}{}

\dictentry{5,000}{}{}{}
{\hyperlink{val:mukiho}{muki'o}}{}{}{}

\dictentry{50}{}{}{}
{\hyperlink{val:muno}{muno}}{}{}{}

\dictentry{500}{}{}{}
{\hyperlink{val:munono}{munono}}{}{}{}

\dictentry{5ta conversión}{}{}{}
{\hyperlink{val:xe}{xe}}{}{}{}

\dictentry{6}{}{}{}
{\hyperlink{val:xa}{xa}}{}{}{}

\dictentry{6,000}{}{}{}
{\hyperlink{val:xakiho}{xaki'o}}{}{}{}

\dictentry{60}{}{}{}
{\hyperlink{val:xano}{xano}}{}{}{}

\dictentry{600}{}{}{}
{\hyperlink{val:xanono}{xanono}}{}{}{}

\dictentry{64,000}{}{}{}
{\hyperlink{val:xavokiho}{xavoki'o}}{}{}{}

\dictentry{7}{}{}{}
{\hyperlink{val:ze}{ze}}{}{}{}

\dictentry{7,000}{}{}{}
{\hyperlink{val:zekiho}{zeki'o}}{}{}{}

\dictentry{70}{}{}{}
{\hyperlink{val:zeno}{zeno}}{}{}{}

\dictentry{700}{}{}{}
{\hyperlink{val:zenono}{zenono}}{}{}{}

\dictentry{8}{}{}{}
{\hyperlink{val:bi}{bi}}{}{}{}

\dictentry{8,000}{}{}{}
{\hyperlink{val:bikiho}{biki'o}}{}{}{}

\dictentry{80}{}{}{}
{\hyperlink{val:bino}{bino}}{}{}{}

\dictentry{800}{}{}{}
{\hyperlink{val:binono}{binono}}{}{}{}

\dictentry{8004}{}{}{}
{\hyperlink{val:binonovo}{binonovo}}{}{}{}

\dictentry{9}{}{}{}
{\hyperlink{val:so}{so}}{}{}{}

\dictentry{9,000}{}{}{}
{\hyperlink{val:sokiho}{soki'o}}{}{}{}

\dictentry{90}{}{}{}
{\hyperlink{val:sono}{sono}}{}{}{}

\dictentry{900}{}{}{}
{\hyperlink{val:sonono}{sonono}}{}{}{}

\dictentry{97.5}{}{}{}
{\hyperlink{val:sozepimu}{sozepimu}}{}{}{}

\dictentry{99 por ciento}{}{}{}
{\hyperlink{val:sosocehi}{sosoce'i}}{}{}{}

\dictchar{A}\phantomsection\addcontentsline{toc}{section}{A}
\dictentry{a}{}{}{}
{\hyperlink{val:abu}{abu}}{}{}{}

\dictentry{a la derecha de}{}{}{}
{\hyperlink{val:rihu}{ri'u}}{}{}{}

\dictentry{a la hora de}{}{}{}
{\hyperlink{val:setihu}{seti'u}}{}{}{}

\dictentry{a la izquierda de}{}{}{}
{\hyperlink{val:zuha}{zu'a}}{}{}{}

\dictentry{a las (horas)}{}{}{}
{\hyperlink{val:tihu}{ti'u}}{}{}{}

\dictentry{a lo largo}{}{}{}
{\hyperlink{val:mohireho}{mo'ire'o}}{}{}{}

\dictentry{a lo sumo}{}{}{}
{\hyperlink{val:suhe}{su'e}}{}{}{}

\dictentry{a pesar de (causa)}{}{}{}
{\hyperlink{val:rihanai}{ri'anai}}{}{}{}

\dictentry{a pesar de (lógica)}{}{}{}
{\hyperlink{val:nihinai}{ni'inai}}{}{}{}

\dictentry{a pesar de (motivo)}{}{}{}
{\hyperlink{val:muhinai}{mu'inai}}{}{}{}

\dictentry{a pesar de (razón)}{}{}{}
{\hyperlink{val:kihunai}{ki'unai}}{}{}{}

\dictentry{a propósito}{}{}{}
{\hyperlink{val:taho}{ta'o}}{}{}{}

\dictentry{a través de}{}{}{}
{\hyperlink{val:paho}{pa'o}}{}{}{}

\dictentry{abajo}{}{}{}
{\hyperlink{val:cnita}{cnita}}{}{}{}

\dictentry{abalorio}{}{}{}
{\hyperlink{val:bidju}{bidju}}{}{}{}

\dictentry{abandona sugerencia}{}{}{}
{\hyperlink{val:ehucuhi}{e'ucu'i}}{}{}{}

\dictentry{abarcar}{}{}{}
{\hyperlink{val:kuspe}{kuspe}}{}{}{}

\dictentry{abdomen}{}{}{}
{\hyperlink{val:betfu}{betfu}}{}{}{}

\dictentry{abeja}{}{}{}
{\hyperlink{val:bifce}{bifce}}{}{}{}

\dictentry{abierto}{}{}{}
{\hyperlink{val:kalri}{kalri}}{}{}{}

\dictentry{abre paréntesis}{}{}{}
{\hyperlink{val:vei}{vei}}{}{}{}

\dictentry{abrigo}{}{}{}
{\hyperlink{val:kosta}{kosta}}{}{}{}

\dictentry{absorber}{}{}{}
{\hyperlink{val:cokcu}{cokcu}}{}{}{}

\dictentry{abstinencia sexual}{}{}{}
{\hyperlink{val:rohunai}{ro'unai}}{}{}{}

\dictentry{abstracción}{}{}{}
{\hyperlink{val:sucta}{sucta}}{}{}{}

\dictentry{abstracción no específica}{}{}{}
{\hyperlink{val:suhu}{su'u}}{}{}{}

\dictentry{aburrido}{}{}{}
{\hyperlink{val:tolzdi}{tolzdi}}{}{}{}

\dictentry{acariciar}{}{}{}
{\hyperlink{val:satre}{satre}}{}{}{}

\dictentry{acaso ?}{}{}{}
{\hyperlink{val:xu}{xu}}{}{}{}

\dictentry{accidental}{}{}{}
{\hyperlink{val:snuti}{snuti}}{}{}{}

\dictentry{aceptación}{}{}{}
{\hyperlink{val:iha}{i'a}}{}{}{}

\dictentry{acercandose}{}{}{}
{\hyperlink{val:mohizohi}{mo'izo'i}}{}{}{}

\dictentry{acero}{}{}{}
{\hyperlink{val:gasta}{gasta}}{}{}{}

\dictentry{ácido}{}{}{}
{\hyperlink{val:slami}{slami}}{}{}{}

\dictentry{acompañar}{}{}{}
{\hyperlink{val:kansa}{kansa}}{}{}{}

\dictentry{acoplar otros sumti}{}{}{}
{\hyperlink{val:bei}{bei}}{}{}{}

\dictentry{acoplar sumti}{}{}{}
{\hyperlink{val:be}{be}}{}{}{}

\dictentry{acople corto}{}{}{}
{\hyperlink{val:bo}{bo}}{}{}{}

\dictentry{acople de oración}{}{}{}
{\hyperlink{val:i}{i}}{}{}{}

\dictentry{acre}{}{}{}
{\hyperlink{val:kramu}{kramu}}{}{}{}

\dictentry{actividad}{}{}{}
{\hyperlink{val:zuho}{zu'o}}{}{}{}

\dictentry{actualización de pro-asignados}{}{}{}
{\hyperlink{val:raho}{ra'o}}{}{}{}

\dictentry{actuar}{}{}{}
{\hyperlink{val:zukte}{zukte}}{}{}{}

\dictentry{acuerdo}{}{}{}
{\hyperlink{val:ie}{ie}}{}{}{}

\dictentry{acullá}{}{}{}
{\hyperlink{val:vu}{vu}}{}{}{}

\dictentry{adelante}{}{}{}
{\hyperlink{val:crane}{crane}}{}{}{}

\dictentry{además}{}{}{}
{\hyperlink{val:jiha}{ji'a}}{}{}{}

\dictentry{adentro}{}{}{}
{\hyperlink{val:nenri}{nenri}}{}{}{}

\dictentry{adherirse}{}{}{}
{\hyperlink{val:snipa}{snipa}}{}{}{}

\dictentry{adiós}{}{}{}
{\hyperlink{val:coho}{co'o}}{}{}{}

\dictentry{adivinar}{}{}{}
{\hyperlink{val:smadi}{smadi}}{}{}{}

\dictentry{adorno}{}{}{}
{\hyperlink{val:jadni}{jadni}}{}{}{}

\dictentry{advertir}{}{}{}
{\hyperlink{val:kajde}{kajde}}{}{}{}

\dictentry{adyacente}{}{}{}
{\hyperlink{val:lamji}{lamji}}{}{}{}

\dictentry{adyacente a}{}{}{}
{\hyperlink{val:reho}{re'o}}{}{}{}

\dictentry{afijo}{}{}{}
{\hyperlink{val:rafsi}{rafsi}}{}{}{}

\dictentry{afirmación escalar}{}{}{}
{\hyperlink{val:jeha}{je'a}}{}{}{}

\dictentry{afirmador de bridi}{}{}{}
{\hyperlink{val:jaha}{ja'a}}{}{}{}

\dictentry{afirmar}{}{}{}
{\hyperlink{val:xusra}{xusra}}{}{}{}

\dictentry{afligir}{}{}{}
{\hyperlink{val:raktu}{raktu}}{}{}{}

\dictentry{africano}{}{}{}
{\hyperlink{val:friko}{friko}}{}{}{}

\dictentry{afterthought termset}{}{}{}
{\hyperlink{val:cehe}{ce'e}}{}{}{}

\dictentry{afuera}{}{}{}
{\hyperlink{val:bartu}{bartu}}{}{}{}

\dictentry{agarrar}{}{}{}
{\hyperlink{val:jgari}{jgari}}{}{}{}

\dictentry{agotamiento}{}{}{}
{\hyperlink{val:ahenai}{a'enai}}{}{}{}

\dictentry{agradar}{}{}{}
{\hyperlink{val:pluka}{pluka}}{}{}{}

\dictentry{agradecer}{}{}{}
{\hyperlink{val:ckire}{ckire}}{}{}{}

\dictentry{agregar}{}{}{}
{\hyperlink{val:jmina}{jmina}}{}{}{}

\dictentry{agresivo}{}{}{}
{\hyperlink{val:leho}{le'o}}{}{}{}

\dictentry{agrio}{}{}{}
{\hyperlink{val:slari}{slari}}{}{}{}

\dictentry{agua}{}{}{}
{\hyperlink{val:djacu}{djacu}}{}{}{}

\dictentry{aguantar}{}{}{}
{\hyperlink{val:renvi}{renvi}}{}{}{}

\dictentry{agudo}{}{}{}
{\hyperlink{val:kinli}{kinli}}{}{}{}

\dictentry{aguja}{}{}{}
{\hyperlink{val:jesni}{jesni}}{}{}{}

\dictentry{ahora}{}{}{}
{\hyperlink{val:cabna}{cabna}}{}{}{}

\dictentry{aire}{}{}{}
{\hyperlink{val:vacri}{vacri}}{}{}{}

\dictentry{ajo}{}{}{}
{\hyperlink{val:sunga}{sunga}}{}{}{}

\dictentry{ajustado}{}{}{}
{\hyperlink{val:tagji}{tagji}}{}{}{}

\dictentry{ajustar}{}{}{}
{\hyperlink{val:stika}{stika}}{}{}{}

\dictentry{al este de}{}{}{}
{\hyperlink{val:duha}{du'a}}{}{}{}

\dictentry{al norte de}{}{}{}
{\hyperlink{val:beha}{be'a}}{}{}{}

\dictentry{al oeste de}{}{}{}
{\hyperlink{val:vuha}{vu'a}}{}{}{}

\dictentry{al revés de}{}{}{}
{\hyperlink{val:sefahe}{sefa'e}}{}{}{}

\dictentry{al sur de}{}{}{}
{\hyperlink{val:nehu}{ne'u}}{}{}{}

\dictentry{ala}{}{}{}
{\hyperlink{val:nalci}{nalci}}{}{}{}

\dictentry{albahaca}{}{}{}
{\hyperlink{val:albahaka}{alba'aka}}{}{}{}

\dictentry{albaricoque}{}{}{}
{\hyperlink{val:birkoku}{birkoku}}{}{}{}

\dictentry{alcachofa}{}{}{}
{\hyperlink{val:xarcufu}{xarcufu}}{}{}{}

\dictentry{alcalino}{}{}{}
{\hyperlink{val:jilka}{jilka}}{}{}{}

\dictentry{alcance de indicador}{}{}{}
{\hyperlink{val:fuhe}{fu'e}}{}{}{}

\dictentry{alcaucil}{}{}{}
{\hyperlink{val:xarcufu}{xarcufu}}{}{}{}

\dictentry{alcohol}{}{}{}
{\hyperlink{val:xalka}{xalka}}{}{}{}

\dictentry{aleatorio}{}{}{}
{\hyperlink{val:cunso}{cunso}}{}{}{}

\dictentry{alemán}{}{}{}
{\hyperlink{val:dotco}{dotco}}{}{}{}

\dictentry{alerta}{}{}{}
{\hyperlink{val:ahe}{a'e}}{}{}{}

\dictentry{alfiler}{}{}{}
{\hyperlink{val:pijne}{pijne}}{}{}{}

\dictentry{alforfón}{}{}{}
{\hyperlink{val:xruba}{xruba}}{}{}{}

\dictentry{algo de}{}{}{}
{\hyperlink{val:pisuho}{pisu'o}}{}{}{}

\dictentry{algodón}{}{}{}
{\hyperlink{val:mapni}{mapni}}{}{}{}

\dictentry{algún miembro de}{}{}{}
{\hyperlink{val:luha}{lu'a}}{}{}{}

\dictentry{algún selbri 1}{}{}{}
{\hyperlink{val:buha}{bu'a}}{}{}{}

\dictentry{algún selbri 2}{}{}{}
{\hyperlink{val:buhe}{bu'e}}{}{}{}

\dictentry{algún selbri 3}{}{}{}
{\hyperlink{val:buhi}{bu'i}}{}{}{}

\dictentry{alguno 1}{}{}{}
{\hyperlink{val:da}{da}}{}{}{}

\dictentry{alguno 2}{}{}{}
{\hyperlink{val:de}{de}}{}{}{}

\dictentry{alguno 3}{}{}{}
{\hyperlink{val:di}{di}}{}{}{}

\dictentry{allí}{}{}{}
{\hyperlink{val:va}{va}}{}{}{}

\dictentry{almidón}{}{}{}
{\hyperlink{val:jalna}{jalna}}{}{}{}

\dictentry{almohadón}{}{}{}
{\hyperlink{val:kicne}{kicne}}{}{}{}

\dictentry{alrededor de}{}{}{}
{\hyperlink{val:ruhu}{ru'u}}{}{}{}

\dictentry{alternativa}{}{}{}
{\hyperlink{val:vlina}{vlina}}{}{}{}

\dictentry{alteza}{}{}{}
{\hyperlink{val:gahi}{ga'i}}{}{}{}

\dictentry{alto}{}{}{}
{\hyperlink{val:galtu}{galtu}}{}{}{}

\dictentry{amable}{}{}{}
{\hyperlink{val:xendo}{xendo}}{}{}{}

\dictentry{amar}{}{}{}
{\hyperlink{val:prami}{prami}}{}{}{}

\dictentry{amargo}{}{}{}
{\hyperlink{val:kurki}{kurki}}{}{}{}

\dictentry{amarillo}{}{}{}
{\hyperlink{val:pelxu}{pelxu}}{}{}{}

\dictentry{ambiente}{}{}{}
{\hyperlink{val:vanbi}{vanbi}}{}{}{}

\dictentry{amenaza}{}{}{}
{\hyperlink{val:ehunai}{e'unai}}{}{}{}

\dictentry{amigo}{}{}{}
{\hyperlink{val:pendo}{pendo}}{}{}{}

\dictentry{amor}{}{}{}
{\hyperlink{val:iu}{iu}}{}{}{}

\dictentry{amparar}{}{}{}
{\hyperlink{val:marbi}{marbi}}{}{}{}

\dictentry{ampere}{}{}{}
{\hyperlink{val:xampo}{xampo}}{}{}{}

\dictentry{analizar}{}{}{}
{\hyperlink{val:lanli}{lanli}}{}{}{}

\dictentry{ancestro}{}{}{}
{\hyperlink{val:dzena}{dzena}}{}{}{}

\dictentry{ancho}{}{}{}
{\hyperlink{val:ganra}{ganra}}{}{}{}

\dictentry{andar}{}{}{}
{\hyperlink{val:cadzu}{cadzu}}{}{}{}

\dictentry{anfibio}{}{}{}
{\hyperlink{val:banfi}{banfi}}{}{}{}

\dictentry{angosto}{}{}{}
{\hyperlink{val:jarki}{jarki}}{}{}{}

\dictentry{anguila}{}{}{}
{\hyperlink{val:angila}{angila}}{}{}{}

\dictentry{ángulo}{}{}{}
{\hyperlink{val:jganu}{jganu}}{}{}{}

\dictentry{angustiarse}{}{}{}
{\hyperlink{val:dunku}{dunku}}{}{}{}

\dictentry{anillo}{}{}{}
{\hyperlink{val:djine}{djine}}{}{}{}

\dictentry{animal}{}{}{}
{\hyperlink{val:danlu}{danlu}}{}{}{}

\dictentry{animar}{}{}{}
{\hyperlink{val:darsygau}{darsygau}}{}{}{}

\dictentry{animarse}{}{}{}
{\hyperlink{val:darsi}{darsi}}{}{}{}

\dictentry{ano}{}{}{}
{\hyperlink{val:ganxo}{ganxo}}{}{}{}

\dictentry{año}{}{}{}
{\hyperlink{val:nanca}{nanca}}{}{}{}

\dictentry{ansiedad}{}{}{}
{\hyperlink{val:oirohi}{oiro'i}}{}{}{}

\dictentry{antártico}{}{}{}
{\hyperlink{val:dzipo}{dzipo}}{}{}{}

\dictentry{antes}{}{}{}
{\hyperlink{val:pu}{pu}}{}{}{}

\dictentry{antes y después}{}{}{}
{\hyperlink{val:pujeba}{pujeba}}{}{}{}

\dictentry{antes y durante}{}{}{}
{\hyperlink{val:pujeca}{pujeca}}{}{}{}

\dictentry{anticipación}{}{}{}
{\hyperlink{val:baha}{ba'a}}{}{}{}

\dictentry{antílope}{}{}{}
{\hyperlink{val:antilope}{antilope}}{}{}{}

\dictentry{antisocial}{}{}{}
{\hyperlink{val:rohanai}{ro'anai}}{}{}{}

\dictentry{aparato}{}{}{}
{\hyperlink{val:cabra}{cabra}}{}{}{}

\dictentry{apático}{}{}{}
{\hyperlink{val:norpahi}{norpa'i}}{}{}{}

\dictentry{aplastar}{}{}{}
{\hyperlink{val:marxa}{marxa}}{}{}{}

\dictentry{apreciación}{}{}{}
{\hyperlink{val:iho}{i'o}}{}{}{}

\dictentry{aprender}{}{}{}
{\hyperlink{val:cilre}{cilre}}{}{}{}

\dictentry{apresuramiento}{}{}{}
{\hyperlink{val:ohinai}{o'inai}}{}{}{}

\dictentry{aprobación}{}{}{}
{\hyperlink{val:ihe}{i'e}}{}{}{}

\dictentry{aprobado por}{}{}{}
{\hyperlink{val:zau}{zau}}{}{}{}

\dictentry{aprobando}{}{}{}
{\hyperlink{val:sezau}{sezau}}{}{}{}

\dictentry{aprobar}{}{}{}
{\hyperlink{val:zanru}{zanru}}{}{}{}

\dictentry{aproximadamente}{}{}{}
{\hyperlink{val:jihi}{ji'i}}{}{}{}

\dictentry{aproximadamente todo}{}{}{}
{\hyperlink{val:pijihi}{piji'i}}{}{}{}

\dictentry{aproximandose}{}{}{}
{\hyperlink{val:mohineha}{mo'ine'a}}{}{}{}

\dictentry{aquel}{}{}{}
{\hyperlink{val:levu}{levu}}{}{}{}

\dictentry{aquél}{}{}{}
{\hyperlink{val:tu}{tu}}{}{}{}

\dictentry{aquello}{}{}{}
{\hyperlink{val:tu}{tu}}{}{}{}

\dictentry{aquí}{}{}{}
{\hyperlink{val:vi}{vi}}{}{}{}

\dictentry{aquí y ahora}{}{}{}
{\hyperlink{val:nau}{nau}}{}{}{}

\dictentry{árabe}{}{}{}
{\hyperlink{val:xrabo}{xrabo}}{}{}{}

\dictentry{araña}{}{}{}
{\hyperlink{val:jukni}{jukni}}{}{}{}

\dictentry{arar}{}{}{}
{\hyperlink{val:plixa}{plixa}}{}{}{}

\dictentry{árbol}{}{}{}
{\hyperlink{val:tricu}{tricu}}{}{}{}

\dictentry{arce}{}{}{}
{\hyperlink{val:ahorne}{a'orne}}{}{}{}

\dictentry{arcilla}{}{}{}
{\hyperlink{val:kliti}{kliti}}{}{}{}

\dictentry{arco}{}{}{}
{\hyperlink{val:bargu}{bargu}}{}{}{}

\dictentry{arena}{}{}{}
{\hyperlink{val:canre}{canre}}{}{}{}

\dictentry{argeliano}{}{}{}
{\hyperlink{val:jerxo}{jerxo}}{}{}{}

\dictentry{argentino}{}{}{}
{\hyperlink{val:gento}{gento}}{}{}{}

\dictentry{argu:ir}{}{}{}
{\hyperlink{val:darlu}{darlu}}{}{}{}

\dictentry{argumento}{}{}{}
{\hyperlink{val:sumti}{sumti}}{}{}{}

\dictentry{arma}{}{}{}
{\hyperlink{val:xarci}{xarci}}{}{}{}

\dictentry{armonioso}{}{}{}
{\hyperlink{val:sarxe}{sarxe}}{}{}{}

\dictentry{arrecife}{}{}{}
{\hyperlink{val:jmifa}{jmifa}}{}{}{}

\dictentry{arrepentimiento}{}{}{}
{\hyperlink{val:uhu}{u'u}}{}{}{}

\dictentry{arriba}{}{}{}
{\hyperlink{val:gapru}{gapru}}{}{}{}

\dictentry{arrojar}{}{}{}
{\hyperlink{val:renro}{renro}}{}{}{}

\dictentry{arroz}{}{}{}
{\hyperlink{val:rismi}{rismi}}{}{}{}

\dictentry{arruga}{}{}{}
{\hyperlink{val:cinje}{cinje}}{}{}{}

\dictentry{arte}{}{}{}
{\hyperlink{val:larcu}{larcu}}{}{}{}

\dictentry{artefacto}{}{}{}
{\hyperlink{val:rutni}{rutni}}{}{}{}

\dictentry{artículo}{}{}{}
{\hyperlink{val:gadri}{gadri}}{}{}{}

\dictentry{artista}{}{}{}
{\hyperlink{val:larpra}{larpra}}{}{}{}

\dictentry{asegurar}{}{}{}
{\hyperlink{val:binra}{binra}}{}{}{}

\dictentry{asiático}{}{}{}
{\hyperlink{val:xazdo}{xazdo}}{}{}{}

\dictentry{asigna pro-bridi}{}{}{}
{\hyperlink{val:cei}{cei}}{}{}{}

\dictentry{asigna pro-sumti}{}{}{}
{\hyperlink{val:goi}{goi}}{}{}{}

\dictentry{asignar}{}{}{}
{\hyperlink{val:snigau}{snigau}}{}{}{}

\dictentry{asno}{}{}{}
{\hyperlink{val:xasli}{xasli}}{}{}{}

\dictentry{asombro}{}{}{}
{\hyperlink{val:uhe}{u'e}}{}{}{}

\dictentry{aspectos espaciales}{}{}{}
{\hyperlink{val:fehe}{fe'e}}{}{}{}

\dictentry{áspero}{}{}{}
{\hyperlink{val:rufsu}{rufsu}}{}{}{}

\dictentry{asunto}{}{}{}
{\hyperlink{val:cuntu}{cuntu}}{}{}{}

\dictentry{asustarse}{}{}{}
{\hyperlink{val:xalni}{xalni}}{}{}{}

\dictentry{atacar}{}{}{}
{\hyperlink{val:gunta}{gunta}}{}{}{}

\dictentry{atar}{}{}{}
{\hyperlink{val:lasna}{lasna}}{}{}{}

\dictentry{atención}{}{}{}
{\hyperlink{val:aha}{a'a}}{}{}{}

\dictentry{atender}{}{}{}
{\hyperlink{val:jundi}{jundi}}{}{}{}

\dictentry{atípicamente}{}{}{}
{\hyperlink{val:nahonai}{na'onai}}{}{}{}

\dictentry{átomo}{}{}{}
{\hyperlink{val:ratni}{ratni}}{}{}{}

\dictentry{atraer}{}{}{}
{\hyperlink{val:trina}{trina}}{}{}{}

\dictentry{atrás}{}{}{}
{\hyperlink{val:trixe}{trixe}}{}{}{}

\dictentry{atravesando}{}{}{}
{\hyperlink{val:mohipaho}{mo'ipa'o}}{}{}{}

\dictentry{atravesar}{}{}{}
{\hyperlink{val:pagre}{pagre}}{}{}{}

\dictentry{atto}{}{}{}
{\hyperlink{val:xatsi}{xatsi}}{}{}{}

\dictentry{audaz}{}{}{}
{\hyperlink{val:darsi}{darsi}}{}{}{}

\dictentry{aumentar}{}{}{}
{\hyperlink{val:zenba}{zenba}}{}{}{}

\dictentry{australiano}{}{}{}
{\hyperlink{val:sralo}{sralo}}{}{}{}

\dictentry{auto}{}{}{}
{\hyperlink{val:karce}{karce}}{}{}{}

\dictentry{automático}{}{}{}
{\hyperlink{val:zmiku}{zmiku}}{}{}{}

\dictentry{autoorientado}{}{}{}
{\hyperlink{val:sehi}{se'i}}{}{}{}

\dictentry{autopresentación}{}{}{}
{\hyperlink{val:mihe}{mi'e}}{}{}{}

\dictentry{autoridad}{}{}{}
{\hyperlink{val:catni}{catni}}{}{}{}

\dictentry{autoridad basada en}{}{}{}
{\hyperlink{val:tecahi}{teca'i}}{}{}{}

\dictentry{autosuficiencia}{}{}{}
{\hyperlink{val:seha}{se'a}}{}{}{}

\dictentry{avena}{}{}{}
{\hyperlink{val:mavji}{mavji}}{}{}{}

\dictentry{avergonzarse}{}{}{}
{\hyperlink{val:ckeji}{ckeji}}{}{}{}

\dictentry{avión}{}{}{}
{\hyperlink{val:vinji}{vinji}}{}{}{}

\dictentry{ayudando a}{}{}{}
{\hyperlink{val:sesihu}{sesi'u}}{}{}{}

\dictentry{ayudando en}{}{}{}
{\hyperlink{val:tesihu}{tesi'u}}{}{}{}

\dictentry{ayudar}{}{}{}
{\hyperlink{val:sidju}{sidju}}{}{}{}

\dictentry{azúcar}{}{}{}
{\hyperlink{val:sakta}{sakta}}{}{}{}

\dictentry{azufre}{}{}{}
{\hyperlink{val:sliri}{sliri}}{}{}{}

\dictentry{azul}{}{}{}
{\hyperlink{val:blanu}{blanu}}{}{}{}

\dictchar{B}\phantomsection\addcontentsline{toc}{section}{B}
\dictentry{b}{}{}{}
{\hyperlink{val:by}{by}}{}{}{}

\dictentry{bahasa}{}{}{}
{\hyperlink{val:baxso}{baxso}}{}{}{}

\dictentry{bahía}{}{}{}
{\hyperlink{val:zbani}{zbani}}{}{}{}

\dictentry{bajar}{}{}{}
{\hyperlink{val:nitkla}{nitkla}}{}{}{}

\dictentry{bajo}{}{}{}
{\hyperlink{val:dizlo}{dizlo}}{}{}{}

\dictentry{bajo condiciones}{}{}{}
{\hyperlink{val:tetahi}{teta'i}}{}{}{}

\dictentry{bajo condiciones ...}{}{}{}
{\hyperlink{val:tepuha}{tepu'a}}{}{}{}

\dictentry{bajo control de}{}{}{}
{\hyperlink{val:jiho}{ji'o}}{}{}{}

\dictentry{bajo sistema lógico ...}{}{}{}
{\hyperlink{val:tenihi}{teni'i}}{}{}{}

\dictentry{balcón}{}{}{}
{\hyperlink{val:balni}{balni}}{}{}{}

\dictentry{balde}{}{}{}
{\hyperlink{val:baktu}{baktu}}{}{}{}

\dictentry{baldosa}{}{}{}
{\hyperlink{val:tapla}{tapla}}{}{}{}

\dictentry{banana}{}{}{}
{\hyperlink{val:badna}{badna}}{}{}{}

\dictentry{banco}{}{}{}
{\hyperlink{val:banxa}{banxa}}{}{}{}

\dictentry{banda}{}{}{}
{\hyperlink{val:bende}{bende}}{}{}{}

\dictentry{bandeja}{}{}{}
{\hyperlink{val:palne}{palne}}{}{}{}

\dictentry{bandera}{}{}{}
{\hyperlink{val:lanci}{lanci}}{}{}{}

\dictentry{bar}{}{}{}
{\hyperlink{val:barja}{barja}}{}{}{}

\dictentry{barra}{}{}{}
{\hyperlink{val:garna}{garna}}{}{}{}

\dictentry{barra de fracción}{}{}{}
{\hyperlink{val:fihu}{fi'u}}{}{}{}

\dictentry{barrio}{}{}{}
{\hyperlink{val:jarbu}{jarbu}}{}{}{}

\dictentry{basado en}{}{}{}
{\hyperlink{val:jihu}{ji'u}}{}{}{}

\dictentry{base}{}{}{}
{\hyperlink{val:jicmu}{jicmu}}{}{}{}

\dictentry{base numérica}{}{}{}
{\hyperlink{val:juhu}{ju'u}}{}{}{}

\dictentry{baya}{}{}{}
{\hyperlink{val:jbari}{jbari}}{}{}{}

\dictentry{beber}{}{}{}
{\hyperlink{val:pinxe}{pinxe}}{}{}{}

\dictentry{bello}{}{}{}
{\hyperlink{val:melbi}{melbi}}{}{}{}

\dictentry{beneficio}{}{}{}
{\hyperlink{val:vahu}{va'u}}{}{}{}

\dictentry{bengalí}{}{}{}
{\hyperlink{val:bengo}{bengo}}{}{}{}

\dictentry{besar}{}{}{}
{\hyperlink{val:cinba}{cinba}}{}{}{}

\dictentry{bienvenido}{}{}{}
{\hyperlink{val:zanvihe}{zanvi'e}}{}{}{}

\dictentry{blanco}{}{}{}
{\hyperlink{val:blabi}{blabi}}{}{}{}

\dictentry{blando}{}{}{}
{\hyperlink{val:ranti}{ranti}}{}{}{}

\dictentry{bloque}{}{}{}
{\hyperlink{val:bliku}{bliku}}{}{}{}

\dictentry{boca}{}{}{}
{\hyperlink{val:moklu}{moklu}}{}{}{}

\dictentry{bola}{}{}{}
{\hyperlink{val:bolci}{bolci}}{}{}{}

\dictentry{boleto}{}{}{}
{\hyperlink{val:pikta}{pikta}}{}{}{}

\dictentry{bolsa}{}{}{}
{\hyperlink{val:dakli}{dakli}}{}{}{}

\dictentry{bolsillo}{}{}{}
{\hyperlink{val:daski}{daski}}{}{}{}

\dictentry{bomba}{}{}{}
{\hyperlink{val:pambe}{pambe}}{}{}{}

\dictentry{bordeando}{}{}{}
{\hyperlink{val:tehe}{te'e}}{}{}{}

\dictentry{borrar discurso}{}{}{}
{\hyperlink{val:su}{su}}{}{}{}

\dictentry{borrar expresión}{}{}{}
{\hyperlink{val:sa}{sa}}{}{}{}

\dictentry{borrar palabra}{}{}{}
{\hyperlink{val:si}{si}}{}{}{}

\dictentry{bote}{}{}{}
{\hyperlink{val:bloti}{bloti}}{}{}{}

\dictentry{botella}{}{}{}
{\hyperlink{val:botpi}{botpi}}{}{}{}

\dictentry{botón}{}{}{}
{\hyperlink{val:batke}{batke}}{}{}{}

\dictentry{brasileño}{}{}{}
{\hyperlink{val:brazo}{brazo}}{}{}{}

\dictentry{brazo}{}{}{}
{\hyperlink{val:birka}{birka}}{}{}{}

\dictentry{brea}{}{}{}
{\hyperlink{val:tarla}{tarla}}{}{}{}

\dictentry{bridi}{}{}{}
{\hyperlink{val:duhu}{du'u}}{}{}{}

\dictentry{bridi ?}{}{}{}
{\hyperlink{val:mo}{mo}}{}{}{}

\dictentry{bridi actual}{}{}{}
{\hyperlink{val:nei}{nei}}{}{}{}

\dictentry{bridi anterior}{}{}{}
{\hyperlink{val:gohu}{go'u}}{}{}{}

\dictentry{bridi anteúltimo}{}{}{}
{\hyperlink{val:gohe}{go'e}}{}{}{}

\dictentry{bridi conector ?}{}{}{}
{\hyperlink{val:gihi}{gi'i}}{}{}{}

\dictentry{bridi no especificado}{}{}{}
{\hyperlink{val:cohe}{co'e}}{}{}{}

\dictentry{bridi o}{}{}{}
{\hyperlink{val:giha}{gi'a}}{}{}{}

\dictentry{bridi o exclusivo}{}{}{}
{\hyperlink{val:gihonai}{gi'onai}}{}{}{}

\dictentry{bridi pero no}{}{}{}
{\hyperlink{val:gihenai}{gi'enai}}{}{}{}

\dictentry{bridi posterior}{}{}{}
{\hyperlink{val:goho}{go'o}}{}{}{}

\dictentry{bridi reciente}{}{}{}
{\hyperlink{val:goha}{go'a}}{}{}{}

\dictentry{bridi sea o no}{}{}{}
{\hyperlink{val:gihu}{gi'u}}{}{}{}

\dictentry{bridi sii}{}{}{}
{\hyperlink{val:giho}{gi'o}}{}{}{}

\dictentry{bridi sólo si}{}{}{}
{\hyperlink{val:nagiha}{nagi'a}}{}{}{}

\dictentry{bridi último}{}{}{}
{\hyperlink{val:gohi}{go'i}}{}{}{}

\dictentry{bridi y}{}{}{}
{\hyperlink{val:gihe}{gi'e}}{}{}{}

\dictentry{brisa}{}{}{}
{\hyperlink{val:brife}{brife}}{}{}{}

\dictentry{británico}{}{}{}
{\hyperlink{val:brito}{brito}}{}{}{}

\dictentry{bronce}{}{}{}
{\hyperlink{val:ransu}{ransu}}{}{}{}

\dictentry{brújula}{}{}{}
{\hyperlink{val:makfartci}{makfartci}}{}{}{}

\dictentry{bruma}{}{}{}
{\hyperlink{val:bumru}{bumru}}{}{}{}

\dictentry{brumoso}{}{}{}
{\hyperlink{val:bumru}{bumru}}{}{}{}

\dictentry{budista}{}{}{}
{\hyperlink{val:budjo}{budjo}}{}{}{}

\dictentry{bueno}{}{}{}
{\hyperlink{val:xamgu}{xamgu}}{}{}{}

\dictentry{búho}{}{}{}
{\hyperlink{val:glauka}{glauka}}{}{}{}

\dictentry{bulbo}{}{}{}
{\hyperlink{val:balji}{balji}}{}{}{}

\dictentry{burlarse}{}{}{}
{\hyperlink{val:ckasu}{ckasu}}{}{}{}

\dictentry{buscar}{}{}{}
{\hyperlink{val:sisku}{sisku}}{}{}{}

\dictchar{C}\phantomsection\addcontentsline{toc}{section}{C}
\dictentry{c}{}{}{}
{\hyperlink{val:cy}{cy}}{}{}{}

\dictentry{Ca}{}{}{}
{\hyperlink{val:bogjinme}{bogjinme}}{}{}{}

\dictentry{caballa}{}{}{}
{\hyperlink{val:skomberu}{skomberu}}{}{}{}

\dictentry{caballo}{}{}{}
{\hyperlink{val:xirma}{xirma}}{}{}{}

\dictentry{cabeza}{}{}{}
{\hyperlink{val:stedu}{stedu}}{}{}{}

\dictentry{cabina}{}{}{}
{\hyperlink{val:sabnu}{sabnu}}{}{}{}

\dictentry{cabra}{}{}{}
{\hyperlink{val:kanba}{kanba}}{}{}{}

\dictentry{cada}{}{}{}
{\hyperlink{val:ro}{ro}}{}{}{}

\dictentry{cadena}{}{}{}
{\hyperlink{val:linsi}{linsi}}{}{}{}

\dictentry{caer}{}{}{}
{\hyperlink{val:farlu}{farlu}}{}{}{}

\dictentry{café}{}{}{}
{\hyperlink{val:ckafi}{ckafi}}{}{}{}

\dictentry{caja}{}{}{}
{\hyperlink{val:tanxe}{tanxe}}{}{}{}

\dictentry{cajón}{}{}{}
{\hyperlink{val:dacru}{dacru}}{}{}{}

\dictentry{calamar}{}{}{}
{\hyperlink{val:kalmari}{kalmari}}{}{}{}

\dictentry{calcetín}{}{}{}
{\hyperlink{val:smoka}{smoka}}{}{}{}

\dictentry{calcio}{}{}{}
{\hyperlink{val:bogjinme}{bogjinme}}{}{}{}

\dictentry{calcular}{}{}{}
{\hyperlink{val:kanji}{kanji}}{}{}{}

\dictentry{caliente}{}{}{}
{\hyperlink{val:glare}{glare}}{}{}{}

\dictentry{calle}{}{}{}
{\hyperlink{val:klaji}{klaji}}{}{}{}

\dictentry{cama}{}{}{}
{\hyperlink{val:ckana}{ckana}}{}{}{}

\dictentry{cámara}{}{}{}
{\hyperlink{val:kacma}{kacma}}{}{}{}

\dictentry{cambia letra siguiente}{}{}{}
{\hyperlink{val:tau}{tau}}{}{}{}

\dictentry{cambiar}{}{}{}
{\hyperlink{val:cenba}{cenba}}{}{}{}

\dictentry{cambio}{}{}{}
{\hyperlink{val:muho}{mu'o}}{}{}{}

\dictentry{cambio a arábigo}{}{}{}
{\hyperlink{val:joho}{jo'o}}{}{}{}

\dictentry{cambio a cirílico}{}{}{}
{\hyperlink{val:ruho}{ru'o}}{}{}{}

\dictentry{cambio a griego}{}{}{}
{\hyperlink{val:geho}{ge'o}}{}{}{}

\dictentry{cambio a hebreo}{}{}{}
{\hyperlink{val:jeho}{je'o}}{}{}{}

\dictentry{cambio a lojban}{}{}{}
{\hyperlink{val:loha}{lo'a}}{}{}{}

\dictentry{cambio a mayúsculas}{}{}{}
{\hyperlink{val:gahe}{ga'e}}{}{}{}

\dictentry{cambio a minúsculas}{}{}{}
{\hyperlink{val:toha}{to'a}}{}{}{}

\dictentry{cambio de tipo de letra}{}{}{}
{\hyperlink{val:ceha}{ce'a}}{}{}{}

\dictentry{cambio y fuera}{}{}{}
{\hyperlink{val:feho}{fe'o}}{}{}{}

\dictentry{camello}{}{}{}
{\hyperlink{val:kumte}{kumte}}{}{}{}

\dictentry{caminar}{}{}{}
{\hyperlink{val:stapa}{stapa}}{}{}{}

\dictentry{camino}{}{}{}
{\hyperlink{val:dargu}{dargu}}{}{}{}

\dictentry{camisa}{}{}{}
{\hyperlink{val:creka}{creka}}{}{}{}

\dictentry{campamento}{}{}{}
{\hyperlink{val:ginka}{ginka}}{}{}{}

\dictentry{campana}{}{}{}
{\hyperlink{val:janbe}{janbe}}{}{}{}

\dictentry{campo}{}{}{}
{\hyperlink{val:foldi}{foldi}}{}{}{}

\dictentry{canadiense}{}{}{}
{\hyperlink{val:kadno}{kadno}}{}{}{}

\dictentry{canal}{}{}{}
{\hyperlink{val:naxle}{naxle}}{}{}{}

\dictentry{canasto}{}{}{}
{\hyperlink{val:lanka}{lanka}}{}{}{}

\dictentry{cancela asignaciones}{}{}{}
{\hyperlink{val:daho}{da'o}}{}{}{}

\dictentry{cancela cambios de alfabeto}{}{}{}
{\hyperlink{val:naha}{na'a}}{}{}{}

\dictentry{cáncer}{}{}{}
{\hyperlink{val:kenra}{kenra}}{}{}{}

\dictentry{candela}{}{}{}
{\hyperlink{val:delno}{delno}}{}{}{}

\dictentry{canjear}{}{}{}
{\hyperlink{val:canja}{canja}}{}{}{}

\dictentry{cansarse}{}{}{}
{\hyperlink{val:tatpi}{tatpi}}{}{}{}

\dictentry{cantar}{}{}{}
{\hyperlink{val:sanga}{sanga}}{}{}{}

\dictentry{cantidad}{}{}{}
{\hyperlink{val:ni}{ni}}{}{}{}

\dictentry{cantidad de}{}{}{}
{\hyperlink{val:lahu}{la'u}}{}{}{}

\dictentry{caoba}{}{}{}
{\hyperlink{val:mahagni}{ma'agni}}{}{}{}

\dictentry{caótico}{}{}{}
{\hyperlink{val:kalsa}{kalsa}}{}{}{}

\dictentry{capital}{}{}{medios de producción}
{\hyperlink{val:pracmu}{pracmu}}{}{}{}

\dictentry{capitán}{}{}{}
{\hyperlink{val:jatna}{jatna}}{}{}{}

\dictentry{capturar}{}{}{}
{\hyperlink{val:kavbu}{kavbu}}{}{}{}

\dictentry{cara}{}{}{}
{\hyperlink{val:flira}{flira}}{}{}{}

\dictentry{caracol}{}{}{}
{\hyperlink{val:skargolu}{skargolu}}{}{}{}

\dictentry{característico de}{}{}{}
{\hyperlink{val:kai}{kai}}{}{}{}

\dictentry{carbón}{}{}{}
{\hyperlink{val:tabno}{tabno}}{}{}{}

\dictentry{carecer}{}{}{}
{\hyperlink{val:claxu}{claxu}}{}{}{}

\dictentry{carecido por}{}{}{}
{\hyperlink{val:cau}{cau}}{}{}{}

\dictentry{carencia}{}{}{}
{\hyperlink{val:behu}{be'u}}{}{}{}

\dictentry{carne}{}{}{}
{\hyperlink{val:rectu}{rectu}}{}{}{}

\dictentry{caro}{}{}{}
{\hyperlink{val:kargu}{kargu}}{}{}{}

\dictentry{carro}{}{}{}
{\hyperlink{val:carce}{carce}}{}{}{}

\dictentry{carta}{}{}{}
{\hyperlink{val:xatra}{xatra}}{}{}{}

\dictentry{cáscara}{}{}{}
{\hyperlink{val:pilka}{pilka}}{}{}{}

\dictentry{casi todo}{}{}{}
{\hyperlink{val:pisoha}{piso'a}}{}{}{}

\dictentry{casi todos}{}{}{}
{\hyperlink{val:soha}{so'a}}{}{}{}

\dictentry{castigar}{}{}{}
{\hyperlink{val:sfasa}{sfasa}}{}{}{}

\dictentry{causado bajo condiciones ...}{}{}{}
{\hyperlink{val:teriha}{teri'a}}{}{}{}

\dictentry{causado por}{}{}{}
{\hyperlink{val:riha}{ri'a}}{}{}{}

\dictentry{causar}{}{}{}
{\hyperlink{val:rinka}{rinka}}{}{}{}

\dictentry{cavar}{}{}{}
{\hyperlink{val:kakpa}{kakpa}}{}{}{}

\dictentry{cazar}{}{}{}
{\hyperlink{val:kalte}{kalte}}{}{}{}

\dictentry{cebada}{}{}{}
{\hyperlink{val:bavmi}{bavmi}}{}{}{}

\dictentry{cebolla}{}{}{}
{\hyperlink{val:sluni}{sluni}}{}{}{}

\dictentry{ceder}{}{}{}
{\hyperlink{val:randa}{randa}}{}{}{}

\dictentry{celebrar}{}{}{}
{\hyperlink{val:salci}{salci}}{}{}{}

\dictentry{célula}{}{}{}
{\hyperlink{val:selci}{selci}}{}{}{}

\dictentry{centavo}{}{}{}
{\hyperlink{val:fepni}{fepni}}{}{}{}

\dictentry{centeno}{}{}{}
{\hyperlink{val:mraji}{mraji}}{}{}{}

\dictentry{centi}{}{}{}
{\hyperlink{val:centi}{centi}}{}{}{}

\dictentry{centro-rango}{}{}{}
{\hyperlink{val:mihi}{mi'i}}{}{}{}

\dictentry{cepillo}{}{}{}
{\hyperlink{val:burcu}{burcu}}{}{}{}

\dictentry{cera}{}{}{}
{\hyperlink{val:lakse}{lakse}}{}{}{}

\dictentry{cerámica}{}{}{}
{\hyperlink{val:staku}{staku}}{}{}{}

\dictentry{cerca}{}{}{}
{\hyperlink{val:jibni}{jibni}}{}{}{}

\dictentry{cercanía}{}{}{}
{\hyperlink{val:ohe}{o'e}}{}{}{}

\dictentry{cerdo}{}{}{}
{\hyperlink{val:xarju}{xarju}}{}{}{}

\dictentry{cerebro}{}{}{}
{\hyperlink{val:besna}{besna}}{}{}{}

\dictentry{cero}{}{}{}
{\hyperlink{val:no}{no}}{}{}{}

\dictentry{cerrado}{}{}{}
{\hyperlink{val:ganlo}{ganlo}}{}{}{}

\dictentry{cerradura}{}{}{}
{\hyperlink{val:stela}{stela}}{}{}{}

\dictentry{certeza}{}{}{}
{\hyperlink{val:juho}{ju'o}}{}{}{}

\dictentry{cerveza}{}{}{}
{\hyperlink{val:birje}{birje}}{}{}{}

\dictentry{césped}{}{}{}
{\hyperlink{val:srasu}{srasu}}{}{}{}

\dictentry{chapa}{}{}{}
{\hyperlink{val:romge}{romge}}{}{}{}

\dictentry{chequear}{}{}{}
{\hyperlink{val:cipcta}{cipcta}}{}{}{}

\dictentry{chino}{}{}{}
{\hyperlink{val:jungo}{jungo}}{}{}{}

\dictentry{chocar}{}{}{}
{\hyperlink{val:janli}{janli}}{}{}{}

\dictentry{chocolate}{}{}{}
{\hyperlink{val:cakla}{cakla}}{}{}{}

\dictentry{chorro}{}{}{}
{\hyperlink{val:jetce}{jetce}}{}{}{}

\dictentry{chotacabras}{}{}{}
{\hyperlink{val:ctecmocpi}{ctecmocpi}}{}{}{}

\dictentry{chupar}{}{}{}
{\hyperlink{val:sakci}{sakci}}{}{}{}

\dictentry{cielo}{}{}{}
{\hyperlink{val:tsani}{tsani}}{}{}{}

\dictentry{ciencia}{}{}{}
{\hyperlink{val:saske}{saske}}{}{}{}

\dictentry{cierra paréntesis}{}{}{}
{\hyperlink{val:toi}{toi}}{}{}{}

\dictentry{cierra parentesis}{}{}{}
{\hyperlink{val:veho}{ve'o}}{}{}{}

\dictentry{ciervo}{}{}{}
{\hyperlink{val:mirli}{mirli}}{}{}{}

\dictentry{cigarro}{}{}{}
{\hyperlink{val:sigja}{sigja}}{}{}{}

\dictentry{cilindro}{}{}{}
{\hyperlink{val:slanu}{slanu}}{}{}{}

\dictentry{cinco}{}{}{}
{\hyperlink{val:mu}{mu}}{}{}{}

\dictentry{cinta}{}{}{}
{\hyperlink{val:dasri}{dasri}}{}{}{}

\dictentry{ciruela}{}{}{}
{\hyperlink{val:flaume}{flaume}}{}{}{}

\dictentry{cita de error}{}{}{}
{\hyperlink{val:lohu}{lo'u}}{}{}{}

\dictentry{cita de una palabra}{}{}{}
{\hyperlink{val:zo}{zo}}{}{}{}

\dictentry{cita no lojbánica}{}{}{}
{\hyperlink{val:zoi}{zoi}}{}{}{}

\dictentry{citar}{}{}{}
{\hyperlink{val:sitna}{sitna}}{}{}{}

\dictentry{cítrico}{}{}{}
{\hyperlink{val:nimre}{nimre}}{}{}{}

\dictentry{ciudad}{}{}{}
{\hyperlink{val:tcadu}{tcadu}}{}{}{}

\dictentry{claramente}{}{}{}
{\hyperlink{val:liha}{li'a}}{}{}{}

\dictentry{clarinete}{}{}{}
{\hyperlink{val:xagri}{xagri}}{}{}{}

\dictentry{clase}{}{}{}
{\hyperlink{val:klesi}{klesi}}{}{}{}

\dictentry{cláusula descriptiva}{}{}{}
{\hyperlink{val:voi}{voi}}{}{}{}

\dictentry{cláusula incidental}{}{}{}
{\hyperlink{val:noi}{noi}}{}{}{}

\dictentry{cláusula restrictiva}{}{}{}
{\hyperlink{val:poi}{poi}}{}{}{}

\dictentry{clavo}{}{}{}
{\hyperlink{val:dinko}{dinko}}{}{}{}

\dictentry{clima}{}{}{}
{\hyperlink{val:tcima}{tcima}}{}{}{}

\dictentry{cloro}{}{}{}
{\hyperlink{val:kliru}{kliru}}{}{}{}

\dictentry{cobarde}{}{}{}
{\hyperlink{val:tolvri}{tolvri}}{}{}{}

\dictentry{cobardía}{}{}{}
{\hyperlink{val:uhonai}{u'onai}}{}{}{}

\dictentry{cobayo}{}{}{}
{\hyperlink{val:smacrkobaiu}{smacrkobaiu}}{}{}{}

\dictentry{cobre}{}{}{}
{\hyperlink{val:tunka}{tunka}}{}{}{}

\dictentry{cociente}{}{}{}
{\hyperlink{val:dilcu}{dilcu}}{}{}{}

\dictentry{cocinar}{}{}{}
{\hyperlink{val:jukpa}{jukpa}}{}{}{}

\dictentry{código}{}{}{}
{\hyperlink{val:mifra}{mifra}}{}{}{}

\dictentry{código de caracteres}{}{}{}
{\hyperlink{val:sehe}{se'e}}{}{}{}

\dictentry{cohete}{}{}{}
{\hyperlink{val:jakne}{jakne}}{}{}{}

\dictentry{coincidente con}{}{}{}
{\hyperlink{val:buhu}{bu'u}}{}{}{}

\dictentry{colchón}{}{}{}
{\hyperlink{val:matci}{matci}}{}{}{}

\dictentry{colgar}{}{}{}
{\hyperlink{val:dandu}{dandu}}{}{}{}

\dictentry{collar}{}{}{}
{\hyperlink{val:karli}{karli}}{}{}{}

\dictentry{color}{}{}{}
{\hyperlink{val:skari}{skari}}{}{}{}

\dictentry{columna}{}{}{}
{\hyperlink{val:kamju}{kamju}}{}{}{}

\dictentry{coma cerrada}{}{}{}
{\hyperlink{val:slaka bu}{slaka bu}}{}{}{}

\dictentry{combustible}{}{}{}
{\hyperlink{val:livla}{livla}}{}{}{}

\dictentry{comentario}{}{}{}
{\hyperlink{val:pinka}{pinka}}{}{}{}

\dictentry{comentario editorial}{}{}{}
{\hyperlink{val:tohi}{to'i}}{}{}{}

\dictentry{comenzar}{}{}{}
{\hyperlink{val:cfari}{cfari}}{}{}{}

\dictentry{comer}{}{}{}
{\hyperlink{val:citka}{citka}}{}{}{}

\dictentry{cómico}{}{}{}
{\hyperlink{val:xajmi}{xajmi}}{}{}{}

\dictentry{comida}{}{}{}
{\hyperlink{val:sanmi}{sanmi}}{}{}{}

\dictentry{comienza alcance de texto}{}{}{}
{\hyperlink{val:tuhe}{tu'e}}{}{}{}

\dictentry{comienzo agrupación}{}{}{}
{\hyperlink{val:ke}{ke}}{}{}{}

\dictentry{comienzo de emoción}{}{}{}
{\hyperlink{val:buho}{bu'o}}{}{}{}

\dictentry{comienzo grupo de términos}{}{}{}
{\hyperlink{val:nuhi}{nu'i}}{}{}{}

\dictentry{comité}{}{}{}
{\hyperlink{val:kamni}{kamni}}{}{}{}

\dictentry{como agente de}{}{}{}
{\hyperlink{val:segau}{segau}}{}{}{}

\dictentry{como borde de}{}{}{}
{\hyperlink{val:sekoi}{sekoi}}{}{}{}

\dictentry{como categoría de}{}{}{}
{\hyperlink{val:seleha}{sele'a}}{}{}{}

\dictentry{como condiciones de}{}{}{}
{\hyperlink{val:sevaho}{seva'o}}{}{}{}

\dictentry{como estándar de}{}{}{}
{\hyperlink{val:semahi}{sema'i}}{}{}{}

\dictentry{como forma de}{}{}{}
{\hyperlink{val:setai}{setai}}{}{}{}

\dictentry{como locación de}{}{}{}
{\hyperlink{val:setuhi}{setu'i}}{}{}{}

\dictentry{como método para}{}{}{}
{\hyperlink{val:setahi}{seta'i}}{}{}{}

\dictentry{como nombre de}{}{}{}
{\hyperlink{val:semehe}{seme'e}}{}{}{}

\dictentry{como origen de}{}{}{}
{\hyperlink{val:serahi}{sera'i}}{}{}{}

\dictentry{como parte de}{}{}{}
{\hyperlink{val:sepahu}{sepa'u}}{}{}{}

\dictentry{¿cómo?}{}{}{}
{\hyperlink{val:tahi ma}{ta'i ma}}{}{}{}

\dictentry{cómodo}{}{}{}
{\hyperlink{val:kufra}{kufra}}{}{}{}

\dictentry{compadecer}{}{}{}
{\hyperlink{val:kecti}{kecti}}{}{}{}

\dictentry{compañía}{}{}{}
{\hyperlink{val:kagni}{kagni}}{}{}{}

\dictentry{comparar}{}{}{}
{\hyperlink{val:karbi}{karbi}}{}{}{}

\dictentry{compás}{}{}{}
{\hyperlink{val:cukyxratci}{cukyxratci}}{}{}{}

\dictentry{compasión}{}{}{}
{\hyperlink{val:uu}{uu}}{}{}{}

\dictentry{compelido por}{}{}{}
{\hyperlink{val:bai}{bai}}{}{}{}

\dictentry{competencia}{}{}{}
{\hyperlink{val:ehe}{e'e}}{}{}{}

\dictentry{competir}{}{}{}
{\hyperlink{val:jivna}{jivna}}{}{}{}

\dictentry{complacido por}{}{}{}
{\hyperlink{val:puha}{pu'a}}{}{}{}

\dictentry{complaciendo a}{}{}{}
{\hyperlink{val:sepuha}{sepu'a}}{}{}{}

\dictentry{complejo}{}{}{}
{\hyperlink{val:pluja}{pluja}}{}{}{}

\dictentry{completarse}{}{}{}
{\hyperlink{val:mulbiho}{mulbi'o}}{}{}{}

\dictentry{completivo}{}{}{}
{\hyperlink{val:mohu}{mo'u}}{}{}{}

\dictentry{completo}{}{}{}
{\hyperlink{val:mulno}{mulno}}{}{}{}

\dictentry{comportarse}{}{}{}
{\hyperlink{val:tarti}{tarti}}{}{}{}

\dictentry{compostura}{}{}{}
{\hyperlink{val:ohucuhi}{o'ucu'i}}{}{}{}

\dictentry{comprensión}{}{}{}
{\hyperlink{val:kihanai}{ki'anai}}{}{}{}

\dictentry{compuesto}{}{}{}
{\hyperlink{val:lujvo}{lujvo}}{}{}{}

\dictentry{computadora}{}{}{}
{\hyperlink{val:skami}{skami}}{}{}{}

\dictentry{común}{}{}{}
{\hyperlink{val:kampu}{kampu}}{}{}{}

\dictentry{comunidad}{}{}{}
{\hyperlink{val:cecmu}{cecmu}}{}{}{}

\dictentry{con acción ...}{}{}{}
{\hyperlink{val:sezuhe}{sezu'e}}{}{}{}

\dictentry{con actor ...}{}{}{}
{\hyperlink{val:zuhe}{zu'e}}{}{}{}

\dictentry{con autoridad sobre}{}{}{}
{\hyperlink{val:secahi}{seca'i}}{}{}{}

\dictentry{con ayuda de}{}{}{}
{\hyperlink{val:sihu}{si'u}}{}{}{}

\dictentry{con beneficiario ...}{}{}{}
{\hyperlink{val:sevahu}{seva'u}}{}{}{}

\dictentry{con componentes de sistema ...}{}{}{}
{\hyperlink{val:tecihe}{teci'e}}{}{}{}

\dictentry{con evento ...}{}{}{}
{\hyperlink{val:fau}{fau}}{}{}{}

\dictentry{con fecha ...}{}{}{}
{\hyperlink{val:dehi}{de'i}}{}{}{}

\dictentry{con forma ...}{}{}{}
{\hyperlink{val:tetai}{tetai}}{}{}{}

\dictentry{con intención ...}{}{}{}
{\hyperlink{val:tezuhe}{tezu'e}}{}{}{}

\dictentry{con nombre ...}{}{}{}
{\hyperlink{val:mehe}{me'e}}{}{}{}

\dictentry{con parte ...}{}{}{}
{\hyperlink{val:pahu}{pa'u}}{}{}{}

\dictentry{con propiedad ...}{}{}{}
{\hyperlink{val:sekai}{sekai}}{}{}{}

\dictentry{¿con qué nombre?}{}{}{}
{\hyperlink{val:mehe ma}{me'e ma}}{}{}{}

\dictentry{con relación ...}{}{}{}
{\hyperlink{val:tekihi}{teki'i}}{}{}{}

\dictentry{con relación a}{}{}{}
{\hyperlink{val:raha}{ra'a}}{}{}{}

\dictentry{con respecto a}{}{}{}
{\hyperlink{val:seraha}{sera'a}}{}{}{}

\dictentry{con resultado ...}{}{}{}
{\hyperlink{val:jahe}{ja'e}}{}{}{}

\dictentry{con sinergía en}{}{}{}
{\hyperlink{val:vecihe}{veci'e}}{}{}{}

\dictentry{con superlativo ...}{}{}{}
{\hyperlink{val:rai}{rai}}{}{}{}

\dictentry{concepto}{}{}{}
{\hyperlink{val:siho}{si'o}}{}{}{}

\dictentry{concernir}{}{}{}
{\hyperlink{val:srana}{srana}}{}{}{}

\dictentry{concha}{}{}{}
{\hyperlink{val:calku}{calku}}{}{}{}

\dictentry{conclusión}{}{}{}
{\hyperlink{val:jaho}{ja'o}}{}{}{}

\dictentry{concordar}{}{}{}
{\hyperlink{val:tugni}{tugni}}{}{}{}

\dictentry{condensarse}{}{}{}
{\hyperlink{val:lunsa}{lunsa}}{}{}{}

\dictentry{condimento}{}{}{}
{\hyperlink{val:tsapi}{tsapi}}{}{}{}

\dictentry{conectivo vago}{}{}{}
{\hyperlink{val:juhe}{ju'e}}{}{}{}

\dictentry{conector ? preposicionado}{}{}{}
{\hyperlink{val:gehi}{ge'i}}{}{}{}

\dictentry{conejillo de Indias}{}{}{}
{\hyperlink{val:smacrkobaiu}{smacrkobaiu}}{}{}{}

\dictentry{conejo}{}{}{}
{\hyperlink{val:ractu}{ractu}}{}{}{}

\dictentry{confiar}{}{}{}
{\hyperlink{val:lacri}{lacri}}{}{}{}

\dictentry{confundir}{}{}{}
{\hyperlink{val:cfipu}{cfipu}}{}{}{}

\dictentry{confusión}{}{}{}
{\hyperlink{val:oirohe}{oiro'e}}{}{}{}

\dictentry{confusión textual}{}{}{}
{\hyperlink{val:kiha}{ki'a}}{}{}{}

\dictentry{congelarse}{}{}{}
{\hyperlink{val:dunja}{dunja}}{}{}{}

\dictentry{conífera}{}{}{}
{\hyperlink{val:ckunu}{ckunu}}{}{}{}

\dictentry{conjunción}{}{}{}
{\hyperlink{val:kanxe}{kanxe}}{}{}{}

\dictentry{conjunto}{}{}{}
{\hyperlink{val:selcmi}{selcmi}}{}{}{}

\dictentry{cono}{}{}{}
{\hyperlink{val:konju}{konju}}{}{}{}

\dictentry{consciente}{}{}{}
{\hyperlink{val:sanji}{sanji}}{}{}{}

\dictentry{constante}{}{}{}
{\hyperlink{val:stodi}{stodi}}{}{}{}

\dictentry{cónsul}{}{}{}
{\hyperlink{val:jansu}{jansu}}{}{}{}

\dictentry{contar}{}{}{}
{\hyperlink{val:kancu}{kancu}}{}{}{}

\dictentry{contención de emoción}{}{}{}
{\hyperlink{val:rihenai}{ri'enai}}{}{}{}

\dictentry{contener}{}{}{}
{\hyperlink{val:vasru}{vasru}}{}{}{}

\dictentry{continúa emoción}{}{}{}
{\hyperlink{val:buhocuhi}{bu'ocu'i}}{}{}{}

\dictentry{continuamente}{}{}{}
{\hyperlink{val:ruhi}{ru'i}}{}{}{}

\dictentry{continuando}{}{}{}
{\hyperlink{val:kehunai}{ke'unai}}{}{}{}

\dictentry{continuando demasiado lejos}{}{}{}
{\hyperlink{val:fehezaho}{fe'eza'o}}{}{}{}

\dictentry{continuativo}{}{}{}
{\hyperlink{val:caho}{ca'o}}{}{}{}

\dictentry{continuo}{}{}{}
{\hyperlink{val:ranji}{ranji}}{}{}{}

\dictentry{contrario}{}{}{}
{\hyperlink{val:dukti}{dukti}}{}{}{}

\dictentry{contrario escalar}{}{}{}
{\hyperlink{val:nahe}{na'e}}{}{}{}

\dictentry{controlando a}{}{}{}
{\hyperlink{val:sejiho}{seji'o}}{}{}{}

\dictentry{controlando en}{}{}{}
{\hyperlink{val:tejiho}{teji'o}}{}{}{}

\dictentry{controlar}{}{}{}
{\hyperlink{val:jitro}{jitro}}{}{}{}

\dictentry{conversión de agente}{}{}{}
{\hyperlink{val:jaigau}{jaigau}}{}{}{}

\dictentry{conversión de locación}{}{}{}
{\hyperlink{val:jaivi}{jaivi}}{}{}{}

\dictentry{conversión de tiempo}{}{}{}
{\hyperlink{val:jaica}{jaica}}{}{}{}

\dictentry{conversión modal}{}{}{}
{\hyperlink{val:jai}{jai}}{}{}{}

\dictentry{convertir}{}{}{}
{\hyperlink{val:galfi}{galfi}}{}{}{}

\dictentry{cónyuge}{}{}{}
{\hyperlink{val:speni}{speni}}{}{}{}

\dictentry{copia}{}{}{}
{\hyperlink{val:fukpi}{fukpi}}{}{}{}

\dictentry{copularse}{}{}{}
{\hyperlink{val:gletu}{gletu}}{}{}{}

\dictentry{coraje}{}{}{}
{\hyperlink{val:uho}{u'o}}{}{}{}

\dictentry{corazón}{}{}{}
{\hyperlink{val:risna}{risna}}{}{}{}

\dictentry{corcho}{}{}{}
{\hyperlink{val:korka}{korka}}{}{}{}

\dictentry{correcto}{}{}{}
{\hyperlink{val:drani}{drani}}{}{}{}

\dictentry{corregir}{}{}{}
{\hyperlink{val:dragau}{dragau}}{}{}{}

\dictentry{correr}{}{}{}
{\hyperlink{val:bajra}{bajra}}{}{}{}

\dictentry{cortar}{}{}{}
{\hyperlink{val:katna}{katna}}{}{}{}

\dictentry{cortés}{}{}{}
{\hyperlink{val:clite}{clite}}{}{}{}

\dictentry{cortina}{}{}{}
{\hyperlink{val:murta}{murta}}{}{}{}

\dictentry{corto}{}{}{}
{\hyperlink{val:tordu}{tordu}}{}{}{}

\dictentry{cosechar}{}{}{}
{\hyperlink{val:crepu}{crepu}}{}{}{}

\dictentry{coser}{}{}{}
{\hyperlink{val:fenso}{fenso}}{}{}{}

\dictentry{costumbre}{}{}{}
{\hyperlink{val:tcaci}{tcaci}}{}{}{}

\dictentry{creado de}{}{}{}
{\hyperlink{val:vefihe}{vefi'e}}{}{}{}

\dictentry{creado para}{}{}{}
{\hyperlink{val:tefihe}{tefi'e}}{}{}{}

\dictentry{creado por}{}{}{}
{\hyperlink{val:fihe}{fi'e}}{}{}{}

\dictentry{creando}{}{}{}
{\hyperlink{val:sefihe}{sefi'e}}{}{}{}

\dictentry{creando un tanru}{}{}{}
{\hyperlink{val:tahu}{ta'u}}{}{}{}

\dictentry{crear}{}{}{}
{\hyperlink{val:finti}{finti}}{}{}{}

\dictentry{crecer}{}{}{}
{\hyperlink{val:banro}{banro}}{}{}{}

\dictentry{creencia}{}{}{}
{\hyperlink{val:ia}{ia}}{}{}{}

\dictentry{creer}{}{}{}
{\hyperlink{val:krici}{krici}}{}{}{}

\dictentry{crema}{}{}{}
{\hyperlink{val:kruji}{kruji}}{}{}{}

\dictentry{crepúsculo}{}{}{}
{\hyperlink{val:murse}{murse}}{}{}{}

\dictentry{cría}{}{}{}
{\hyperlink{val:panzi}{panzi}}{}{}{}

\dictentry{criar}{}{}{}
{\hyperlink{val:rirni}{rirni}}{}{}{}

\dictentry{crimen}{}{}{}
{\hyperlink{val:zekri}{zekri}}{}{}{}

\dictentry{cristal}{}{}{}
{\hyperlink{val:krili}{krili}}{}{}{}

\dictentry{cristiano}{}{}{}
{\hyperlink{val:xriso}{xriso}}{}{}{}

\dictentry{crucifijo}{}{}{}
{\hyperlink{val:kucygahasni}{kucyga'asni}}{}{}{}

\dictentry{cruel}{}{}{}
{\hyperlink{val:kusru}{kusru}}{}{}{}

\dictentry{crueldad}{}{}{}
{\hyperlink{val:uunai}{uunai}}{}{}{}

\dictentry{cruz}{}{}{instrumento de tortura}
{\hyperlink{val:kucygaha}{kucyga'a}}{}{}{}

\dictentry{cuadrado}{}{}{}
{\hyperlink{val:kurfa}{kurfa}}{}{}{}

\dictentry{cuál sumti ?}{}{}{}
{\hyperlink{val:fiha}{fi'a}}{}{}{}

\dictentry{cualidad}{}{}{}
{\hyperlink{val:ckaji}{ckaji}}{}{}{}

\dictentry{¿cuán seguro?}{}{}{}
{\hyperlink{val:juhopei}{ju'opei}}{}{}{}

\dictentry{¿cuándo?}{}{}{}
{\hyperlink{val:ca ma}{ca ma}}{}{}{}

\dictentry{cuanto}{}{}{}
{\hyperlink{val:kantu}{kantu}}{}{}{}

\dictentry{¿cuánto?}{}{}{}
{\hyperlink{val:lahu ma}{la'u ma}}{}{}{}

\dictentry{cuarto}{}{}{}
{\hyperlink{val:kumfa}{kumfa}}{}{}{}

\dictentry{cuatro}{}{}{}
{\hyperlink{val:vo}{vo}}{}{}{}

\dictentry{cubo}{}{}{}
{\hyperlink{val:kubli}{kubli}}{}{}{}

\dictentry{cubrir}{}{}{}
{\hyperlink{val:gacri}{gacri}}{}{}{}

\dictentry{cucaracha}{}{}{}
{\hyperlink{val:jalra}{jalra}}{}{}{}

\dictentry{cuchara}{}{}{}
{\hyperlink{val:smuci}{smuci}}{}{}{}

\dictentry{cuchillo}{}{}{}
{\hyperlink{val:dakfu}{dakfu}}{}{}{}

\dictentry{cuello}{}{}{}
{\hyperlink{val:cnebo}{cnebo}}{}{}{}

\dictentry{cuento}{}{}{}
{\hyperlink{val:lisri}{lisri}}{}{}{}

\dictentry{cuerda}{}{}{}
{\hyperlink{val:skori}{skori}}{}{}{}

\dictentry{cuerno}{}{}{}
{\hyperlink{val:jirna}{jirna}}{}{}{}

\dictentry{cuerpo}{}{}{}
{\hyperlink{val:xadni}{xadni}}{}{}{}

\dictentry{cueva}{}{}{}
{\hyperlink{val:kevna}{kevna}}{}{}{}

\dictentry{cuidar}{}{}{}
{\hyperlink{val:kurji}{kurji}}{}{}{}

\dictentry{culminación}{}{}{}
{\hyperlink{val:uo}{uo}}{}{}{}

\dictentry{culpa}{}{}{}
{\hyperlink{val:ihanai}{i'anai}}{}{}{}

\dictentry{culparse}{}{}{}
{\hyperlink{val:zungi}{zungi}}{}{}{}

\dictentry{cultura}{}{}{}
{\hyperlink{val:kulnu}{kulnu}}{}{}{}

\dictentry{cuña}{}{}{}
{\hyperlink{val:cfine}{cfine}}{}{}{}

\dictentry{curar}{}{}{}
{\hyperlink{val:mikce}{mikce}}{}{}{}

\dictentry{curioso}{}{}{}
{\hyperlink{val:kucli}{kucli}}{}{}{}

\dictentry{curva}{}{}{}
{\hyperlink{val:kruvi}{kruvi}}{}{}{}

\dictentry{cyan}{}{}{}
{\hyperlink{val:cicna}{cicna}}{}{}{}

\dictchar{D}\phantomsection\addcontentsline{toc}{section}{D}
\dictentry{d}{}{}{}
{\hyperlink{val:dy}{dy}}{}{}{}

\dictentry{damasco}{}{}{}
{\hyperlink{val:birkoku}{birkoku}}{}{}{}

\dictentry{danzar}{}{}{}
{\hyperlink{val:dansu}{dansu}}{}{}{}

\dictentry{dar}{}{}{}
{\hyperlink{val:dunda}{dunda}}{}{}{}

\dictentry{dato}{}{}{}
{\hyperlink{val:datni}{datni}}{}{}{}

\dictentry{de hecho}{}{}{}
{\hyperlink{val:dahinai}{da'inai}}{}{}{}

\dictentry{de origen}{}{}{}
{\hyperlink{val:rahi}{ra'i}}{}{}{}

\dictentry{debajo de}{}{}{}
{\hyperlink{val:niha}{ni'a}}{}{}{}

\dictentry{deber}{}{}{}
{\hyperlink{val:dejni}{dejni}}{}{}{}

\dictentry{débil}{}{}{}
{\hyperlink{val:ruble}{ruble}}{}{}{}

\dictentry{deca}{}{}{}
{\hyperlink{val:dekto}{dekto}}{}{}{}

\dictentry{deci}{}{}{}
{\hyperlink{val:decti}{decti}}{}{}{}

\dictentry{decidir}{}{}{}
{\hyperlink{val:jdice}{jdice}}{}{}{}

\dictentry{decimal}{}{}{}
{\hyperlink{val:saclu}{saclu}}{}{}{}

\dictentry{decimales repetidos}{}{}{}
{\hyperlink{val:rahe}{ra'e}}{}{}{}

\dictentry{decir}{}{}{}
{\hyperlink{val:cusku}{cusku}}{}{}{}

\dictentry{declaración}{}{}{}
{\hyperlink{val:juha}{ju'a}}{}{}{}

\dictentry{dedo}{}{}{}
{\hyperlink{val:degji}{degji}}{}{}{}

\dictentry{defender}{}{}{}
{\hyperlink{val:bandu}{bandu}}{}{}{}

\dictentry{defensivo}{}{}{}
{\hyperlink{val:lehonai}{le'onai}}{}{}{}

\dictentry{definición}{}{}{}
{\hyperlink{val:cahe}{ca'e}}{}{}{}

\dictentry{definido por cualiad ...}{}{}{}
{\hyperlink{val:teleha}{tele'a}}{}{}{}

\dictentry{dejando}{}{}{}
{\hyperlink{val:mohitoho}{mo'ito'o}}{}{}{}

\dictentry{delgado}{}{}{}
{\hyperlink{val:cinla}{cinla}}{}{}{}

\dictentry{delicado}{}{}{}
{\hyperlink{val:ralci}{ralci}}{}{}{}

\dictentry{demasiadas veces}{}{}{}
{\hyperlink{val:duheroi}{du'eroi}}{}{}{}

\dictentry{demasiado}{}{}{}
{\hyperlink{val:dukse}{dukse}}{}{}{}

\dictentry{demasiado de}{}{}{}
{\hyperlink{val:piduhe}{pidu'e}}{}{}{}

\dictentry{demasiado pocas veces}{}{}{}
{\hyperlink{val:moharoi}{mo'aroi}}{}{}{}

\dictentry{demasiado poco de}{}{}{}
{\hyperlink{val:pimoha}{pimo'a}}{}{}{}

\dictentry{demasiado pocos}{}{}{}
{\hyperlink{val:moha}{mo'a}}{}{}{}

\dictentry{demasiados}{}{}{}
{\hyperlink{val:duhe}{du'e}}{}{}{}

\dictentry{denso}{}{}{}
{\hyperlink{val:denmi}{denmi}}{}{}{}

\dictentry{dentro de}{}{}{}
{\hyperlink{val:nehi}{ne'i}}{}{}{}

\dictentry{dentro de rango ...}{}{}{}
{\hyperlink{val:tediho}{tedi'o}}{}{}{}

\dictentry{dependencia}{}{}{}
{\hyperlink{val:sehanai}{se'anai}}{}{}{}

\dictentry{depósito}{}{}{}
{\hyperlink{val:sorcu}{sorcu}}{}{}{}

\dictentry{derecha}{}{}{}
{\hyperlink{val:pritu}{pritu}}{}{}{}

\dictentry{derivada}{}{}{}
{\hyperlink{val:saho}{sa'o}}{}{}{}

\dictentry{derogativo}{}{}{}
{\hyperlink{val:mabla}{mabla}}{}{}{}

\dictentry{derretirse}{}{}{}
{\hyperlink{val:runme}{runme}}{}{}{}

\dictentry{desacuerdo}{}{}{}
{\hyperlink{val:ienai}{ienai}}{}{}{}

\dictentry{desafiar}{}{}{}
{\hyperlink{val:talsa}{talsa}}{}{}{}

\dictentry{desafío}{}{}{}
{\hyperlink{val:ehinai}{e'inai}}{}{}{}

\dictentry{desanimar}{}{}{}
{\hyperlink{val:toldarsygau}{toldarsygau}}{}{}{}

\dictentry{desaparecer}{}{}{}
{\hyperlink{val:canci}{canci}}{}{}{}

\dictentry{desapasionado}{}{}{}
{\hyperlink{val:norpahi}{norpa'i}}{}{}{}

\dictentry{desaprobación}{}{}{}
{\hyperlink{val:ihenai}{i'enai}}{}{}{}

\dictentry{desarrollarse}{}{}{}
{\hyperlink{val:farvi}{farvi}}{}{}{}

\dictentry{desatención}{}{}{}
{\hyperlink{val:ahacuhi}{a'acu'i}}{}{}{}

\dictentry{descanso}{}{}{}
{\hyperlink{val:juhicuhi}{ju'icu'i}}{}{}{}

\dictentry{descarga de emoción}{}{}{}
{\hyperlink{val:rihe}{ri'e}}{}{}{}

\dictentry{descerebrado}{}{}{}
{\hyperlink{val:rohenai}{ro'enai}}{}{}{}

\dictentry{descreimiento}{}{}{}
{\hyperlink{val:ianai}{ianai}}{}{}{}

\dictentry{describir}{}{}{}
{\hyperlink{val:skicu}{skicu}}{}{}{}

\dictentry{descubrimiento}{}{}{}
{\hyperlink{val:ua}{ua}}{}{}{}

\dictentry{descubrir}{}{}{}
{\hyperlink{val:facki}{facki}}{}{}{}

\dictentry{desde}{}{}{}
{\hyperlink{val:tekaha}{teka'a}}{}{}{}

\dictentry{deseo}{}{}{}
{\hyperlink{val:au}{au}}{}{}{}

\dictentry{desesperación}{}{}{}
{\hyperlink{val:ahonai}{a'onai}}{}{}{}

\dictentry{desestimar}{}{}{}
{\hyperlink{val:tolsiha}{tolsi'a}}{}{}{}

\dictentry{desinterés}{}{}{}
{\hyperlink{val:ahucuhi}{a'ucu'i}}{}{}{}

\dictentry{desnudo}{}{}{}
{\hyperlink{val:lunbe}{lunbe}}{}{}{}

\dictentry{desperdicio}{}{}{}
{\hyperlink{val:festi}{festi}}{}{}{}

\dictentry{despiadado}{}{}{}
{\hyperlink{val:tolkehi}{tolke'i}}{}{}{}

\dictentry{despierto}{}{}{}
{\hyperlink{val:cikna}{cikna}}{}{}{}

\dictentry{despromesa}{}{}{}
{\hyperlink{val:nuhenai}{nu'enai}}{}{}{}

\dictentry{después}{}{}{}
{\hyperlink{val:ba}{ba}}{}{}{}

\dictentry{destino}{}{}{}
{\hyperlink{val:dimna}{dimna}}{}{}{}

\dictentry{destruir}{}{}{}
{\hyperlink{val:daspo}{daspo}}{}{}{}

\dictentry{detalle}{}{}{}
{\hyperlink{val:tcila}{tcila}}{}{}{}

\dictentry{detrás de}{}{}{}
{\hyperlink{val:tiha}{ti'a}}{}{}{}

\dictentry{día}{}{}{}
{\hyperlink{val:djedi}{djedi}}{}{}{}

\dictentry{diámetro}{}{}{}
{\hyperlink{val:mijgresirji}{mijgresirji}}{}{}{}

\dictentry{dibujo}{}{}{}
{\hyperlink{val:pixra}{pixra}}{}{}{}

\dictentry{dicen que}{}{}{}
{\hyperlink{val:tihe}{ti'e}}{}{}{}

\dictentry{dicho a}{}{}{}
{\hyperlink{val:tecuhu}{tecu'u}}{}{}{}

\dictentry{dicho por}{}{}{}
{\hyperlink{val:cuhu}{cu'u}}{}{}{}

\dictentry{dicho por ello-1}{}{}{}
{\hyperlink{val:cuhu koha}{cu'u ko'a}}{}{}{}

\dictentry{diente}{}{}{}
{\hyperlink{val:denci}{denci}}{}{}{}

\dictentry{diferir}{}{}{}
{\hyperlink{val:frica}{frica}}{}{}{}

\dictentry{difícil}{}{}{}
{\hyperlink{val:fuhinai}{fu'inai}}{}{}{}

\dictentry{dígito hexadecimal A}{}{}{}
{\hyperlink{val:dau}{dau}}{}{}{}

\dictentry{dígito hexadecimal B}{}{}{}
{\hyperlink{val:fei}{fei}}{}{}{}

\dictentry{dígito hexadecimal C}{}{}{}
{\hyperlink{val:gai}{gai}}{}{}{}

\dictentry{dígito hexadecimal D}{}{}{}
{\hyperlink{val:jau}{jau}}{}{}{}

\dictentry{dígito hexadecimal E}{}{}{}
{\hyperlink{val:rei}{rei}}{}{}{}

\dictentry{dígito hexadecimal F}{}{}{}
{\hyperlink{val:vai}{vai}}{}{}{}

\dictentry{dimensión}{}{}{}
{\hyperlink{val:cimde}{cimde}}{}{}{}

\dictentry{dinero}{}{}{}
{\hyperlink{val:jdini}{jdini}}{}{}{}

\dictentry{dios}{}{}{}
{\hyperlink{val:cevni}{cevni}}{}{}{}

\dictentry{dirección}{}{}{}
{\hyperlink{val:judri}{judri}}{}{}{}

\dictentry{discrepar}{}{}{}
{\hyperlink{val:toltuhi}{toltu'i}}{}{}{}

\dictentry{discursivo}{}{}{}
{\hyperlink{val:sei}{sei}}{}{}{}

\dictentry{discutir}{}{}{}
{\hyperlink{val:casnu}{casnu}}{}{}{}

\dictentry{disfrutar}{}{}{}
{\hyperlink{val:nelci}{nelci}}{}{}{}

\dictentry{disminuir}{}{}{}
{\hyperlink{val:jdika}{jdika}}{}{}{}

\dictentry{disolverse}{}{}{}
{\hyperlink{val:runta}{runta}}{}{}{}

\dictentry{disparar}{}{}{}
{\hyperlink{val:cecla}{cecla}}{}{}{}

\dictentry{distancia}{}{}{}
{\hyperlink{val:ohenai}{o'enai}}{}{}{}

\dictentry{distribuir}{}{}{}
{\hyperlink{val:fatri}{fatri}}{}{}{}

\dictentry{diurno}{}{}{}
{\hyperlink{val:donri}{donri}}{}{}{}

\dictentry{diversión}{}{}{}
{\hyperlink{val:uhi}{u'i}}{}{}{}

\dictentry{divertir}{}{}{}
{\hyperlink{val:zdile}{zdile}}{}{}{}

\dictentry{dividido por}{}{}{}
{\hyperlink{val:fehi}{fe'i}}{}{}{}

\dictentry{dividir}{}{}{}
{\hyperlink{val:fendi}{fendi}}{}{}{}

\dictentry{doblar}{}{}{}
{\hyperlink{val:polje}{polje}}{}{}{}

\dictentry{doler}{}{}{}
{\hyperlink{val:cortu}{cortu}}{}{}{}

\dictentry{dolor físico}{}{}{}
{\hyperlink{val:oiroho}{oiro'o}}{}{}{}

\dictentry{¿dónde?}{}{}{}
{\hyperlink{val:vi ma}{vi ma}}{}{}{}

\dictentry{dormir}{}{}{}
{\hyperlink{val:sipna}{sipna}}{}{}{}

\dictentry{dos}{}{}{}
{\hyperlink{val:re}{re}}{}{}{}

\dictentry{dos veces}{}{}{}
{\hyperlink{val:reroi}{reroi}}{}{}{}

\dictentry{drama}{}{}{}
{\hyperlink{val:draci}{draci}}{}{}{}

\dictentry{dudar}{}{}{}
{\hyperlink{val:senpi}{senpi}}{}{}{}

\dictentry{dulce}{}{}{}
{\hyperlink{val:titla}{titla}}{}{}{}

\dictentry{duración}{}{}{}
{\hyperlink{val:krafamtei}{krafamtei}}{}{}{}

\dictentry{durante}{}{}{}
{\hyperlink{val:ca}{ca}}{}{}{}

\dictentry{durante y después}{}{}{}
{\hyperlink{val:cajeba}{cajeba}}{}{}{}

\dictentry{duro}{}{}{}
{\hyperlink{val:jdari}{jdari}}{}{}{}

\dictchar{E}\phantomsection\addcontentsline{toc}{section}{E}
\dictentry{e}{}{}{}
{\hyperlink{val:ebu}{ebu}}{}{}{}

\dictentry{edificio}{}{}{}
{\hyperlink{val:dinju}{dinju}}{}{}{}

\dictentry{egipcio}{}{}{}
{\hyperlink{val:misro}{misro}}{}{}{}

\dictentry{egresivo}{}{}{}
{\hyperlink{val:cohu}{co'u}}{}{}{}

\dictentry{eje}{}{}{}
{\hyperlink{val:jendu}{jendu}}{}{}{}

\dictentry{ejemplo}{}{}{}
{\hyperlink{val:muha}{mu'a}}{}{}{}

\dictentry{ejemplo de}{}{}{}
{\hyperlink{val:semuhu}{semu'u}}{}{}{}

\dictentry{ejemplo entre}{}{}{}
{\hyperlink{val:temuhu}{temu'u}}{}{}{}

\dictentry{ejército}{}{}{}
{\hyperlink{val:jenmi}{jenmi}}{}{}{}

\dictentry{el ... de aquél}{}{}{}
{\hyperlink{val:letu}{letu}}{}{}{}

\dictentry{el ... de ése}{}{}{}
{\hyperlink{val:leta}{leta}}{}{}{}

\dictentry{el ... de éste}{}{}{}
{\hyperlink{val:leti}{leti}}{}{}{}

\dictentry{el agente en}{}{}{}
{\hyperlink{val:le jaigau}{le jaigau}}{}{}{}

\dictentry{el bridi implicado por}{}{}{}
{\hyperlink{val:tuha}{tu'a}}{}{}{}

\dictentry{el conjunto compuesto por}{}{}{}
{\hyperlink{val:luhi}{lu'i}}{}{}{}

\dictentry{el conjunto de los descriptos}{}{}{}
{\hyperlink{val:lehi}{le'i}}{}{}{}

\dictentry{el conjunto de los llamados}{}{}{}
{\hyperlink{val:lahi}{la'i}}{}{}{}

\dictentry{el conjunto de los que realmente son}{}{}{}
{\hyperlink{val:lohi}{lo'i}}{}{}{}

\dictentry{el descripto}{}{}{}
{\hyperlink{val:le}{le}}{}{}{}

\dictentry{el estereotípico}{}{}{}
{\hyperlink{val:lehe}{le'e}}{}{}{}

\dictentry{el grupo compuesto por}{}{}{}
{\hyperlink{val:luho}{lu'o}}{}{}{}

\dictentry{el grupo descripto}{}{}{}
{\hyperlink{val:lei}{lei}}{}{}{}

\dictentry{el llamado}{}{}{}
{\hyperlink{val:la}{la}}{}{}{}

\dictentry{el no lojbánicamente llamado}{}{}{}
{\hyperlink{val:laho}{la'o}}{}{}{}

\dictentry{el número}{}{}{}
{\hyperlink{val:li}{li}}{}{}{}

\dictentry{el referente de}{}{}{}
{\hyperlink{val:lahe}{la'e}}{}{}{}

\dictentry{el símbolo de}{}{}{}
{\hyperlink{val:luhe}{lu'e}}{}{}{}

\dictentry{el tiempo de}{}{}{}
{\hyperlink{val:le jaica}{le jaica}}{}{}{}

\dictentry{el típico}{}{}{}
{\hyperlink{val:zuhi}{zu'i}}{}{}{}

\dictentry{el x1 del último bridi}{}{}{}
{\hyperlink{val:le gohi}{le go'i}}{}{}{}

\dictentry{el x5 del último bridi}{}{}{}
{\hyperlink{val:le xegohi}{le xego'i}}{}{}{}

\dictentry{elaborando}{}{}{}
{\hyperlink{val:sahunai}{sa'unai}}{}{}{}

\dictentry{elástico}{}{}{}
{\hyperlink{val:pruni}{pruni}}{}{}{}

\dictentry{electricidad}{}{}{}
{\hyperlink{val:dikca}{dikca}}{}{}{}

\dictentry{elefante}{}{}{}
{\hyperlink{val:xanto}{xanto}}{}{}{}

\dictentry{elegir}{}{}{}
{\hyperlink{val:cuxna}{cuxna}}{}{}{}

\dictentry{elimina argumento}{}{}{}
{\hyperlink{val:ziho}{zi'o}}{}{}{}

\dictentry{ello-1}{}{}{}
{\hyperlink{val:koha}{ko'a}}{}{}{}

\dictentry{ello-10}{}{}{}
{\hyperlink{val:fohu}{fo'u}}{}{}{}

\dictentry{ello-2}{}{}{}
{\hyperlink{val:kohe}{ko'e}}{}{}{}

\dictentry{ello-3}{}{}{}
{\hyperlink{val:kohi}{ko'i}}{}{}{}

\dictentry{ello-4}{}{}{}
{\hyperlink{val:koho}{ko'o}}{}{}{}

\dictentry{ello-5}{}{}{}
{\hyperlink{val:kohu}{ko'u}}{}{}{}

\dictentry{ello-6}{}{}{}
{\hyperlink{val:foha}{fo'a}}{}{}{}

\dictentry{ello-7}{}{}{}
{\hyperlink{val:fohe}{fo'e}}{}{}{}

\dictentry{ello-8}{}{}{}
{\hyperlink{val:fohi}{fo'i}}{}{}{}

\dictentry{ello-9}{}{}{}
{\hyperlink{val:foho}{fo'o}}{}{}{}

\dictentry{embarazo}{}{}{}
{\hyperlink{val:oiroha}{oiro'a}}{}{}{}

\dictentry{embrión}{}{}{}
{\hyperlink{val:tarbi}{tarbi}}{}{}{}

\dictentry{emoción}{}{}{}
{\hyperlink{val:seciho}{seci'o}}{}{}{}

\dictentry{emoción ?}{}{}{}
{\hyperlink{val:pei}{pei}}{}{}{}

\dictentry{emoción débil}{}{}{}
{\hyperlink{val:ruhe}{ru'e}}{}{}{}

\dictentry{emoción fuerte}{}{}{}
{\hyperlink{val:sai}{sai}}{}{}{}

\dictentry{emoción intensa}{}{}{}
{\hyperlink{val:cai}{cai}}{}{}{}

\dictentry{emoción neutral}{}{}{}
{\hyperlink{val:cuhi}{cu'i}}{}{}{}

\dictentry{emoción no específica}{}{}{}
{\hyperlink{val:gehe}{ge'e}}{}{}{}

\dictentry{emoción por}{}{}{}
{\hyperlink{val:teciho}{teci'o}}{}{}{}

\dictentry{emocional}{}{}{}
{\hyperlink{val:rohi}{ro'i}}{}{}{}

\dictentry{empatía}{}{}{}
{\hyperlink{val:dai}{dai}}{}{}{}

\dictentry{empujar}{}{}{}
{\hyperlink{val:catke}{catke}}{}{}{}

\dictentry{en (sitio)}{}{}{}
{\hyperlink{val:tuhi}{tu'i}}{}{}{}

\dictentry{en (ubicación)}{}{}{}
{\hyperlink{val:diho}{di'o}}{}{}{}

\dictentry{en breve}{}{}{}
{\hyperlink{val:tohu}{to'u}}{}{}{}

\dictentry{en cantidad ...}{}{}{}
{\hyperlink{val:selahu}{sela'u}}{}{}{}

\dictentry{en categoria ...}{}{}{}
{\hyperlink{val:leha}{le'a}}{}{}{}

\dictentry{en común con}{}{}{}
{\hyperlink{val:johu}{jo'u}}{}{}{}

\dictentry{en cultura ...}{}{}{}
{\hyperlink{val:kuhu}{ku'u}}{}{}{}

\dictentry{en cultura de}{}{}{}
{\hyperlink{val:sekuhu}{seku'u}}{}{}{}

\dictentry{en detalle}{}{}{}
{\hyperlink{val:tohunai}{to'unai}}{}{}{}

\dictentry{en el estremo de}{}{}{}
{\hyperlink{val:fehemohu}{fe'emo'u}}{}{}{}

\dictentry{en el punto de}{}{}{}
{\hyperlink{val:fehecohi}{fe'eco'i}}{}{}{}

\dictentry{en escala ...}{}{}{}
{\hyperlink{val:telahu}{tela'u}}{}{}{}

\dictentry{en este lado de}{}{}{}
{\hyperlink{val:fehecoha}{fe'eco'a}}{}{}{}

\dictentry{en extremo ...}{}{}{}
{\hyperlink{val:terai}{terai}}{}{}{}

\dictentry{en forma ...}{}{}{}
{\hyperlink{val:tai}{tai}}{}{}{}

\dictentry{en forma material ...}{}{}{}
{\hyperlink{val:temahe}{tema'e}}{}{}{}

\dictentry{en gran parte}{}{}{}
{\hyperlink{val:pisohiroi}{piso'iroi}}{}{}{}

\dictentry{en las mismas palabras}{}{}{}
{\hyperlink{val:vahinai}{va'inai}}{}{}{}

\dictentry{en lenguaje ...}{}{}{}
{\hyperlink{val:bau}{bau}}{}{}{}

\dictentry{en lenguaje de}{}{}{}
{\hyperlink{val:sebau}{sebau}}{}{}{}

\dictentry{en lugar ...}{}{}{}
{\hyperlink{val:sediho}{sedi'o}}{}{}{}

\dictentry{en marco de referencia ...}{}{}{}
{\hyperlink{val:mahi}{ma'i}}{}{}{}

\dictentry{en ningún lado}{}{}{}
{\hyperlink{val:fehenoroi}{fe'enoroi}}{}{}{}

\dictentry{en otras palabras}{}{}{}
{\hyperlink{val:vahi}{va'i}}{}{}{}

\dictentry{en paralelo}{}{}{}
{\hyperlink{val:pahaku}{pa'aku}}{}{}{}

\dictentry{en primer lugar}{}{}{}
{\hyperlink{val:pamai}{pamai}}{}{}{}

\dictentry{en relación con}{}{}{}
{\hyperlink{val:kihi}{ki'i}}{}{}{}

\dictentry{en secuencia ...}{}{}{}
{\hyperlink{val:pohi}{po'i}}{}{}{}

\dictentry{en secuencia con}{}{}{}
{\hyperlink{val:ceho}{ce'o}}{}{}{}

\dictentry{en segundo lugar}{}{}{}
{\hyperlink{val:remai}{remai}}{}{}{}

\dictentry{en sistema ...}{}{}{}
{\hyperlink{val:cihe}{ci'e}}{}{}{}

\dictentry{en todos lados}{}{}{}
{\hyperlink{val:feheroroi}{fe'eroroi}}{}{}{}

\dictentry{en un conjunto con}{}{}{}
{\hyperlink{val:ce}{ce}}{}{}{}

\dictentry{en vez de}{}{}{}
{\hyperlink{val:sebahi}{seba'i}}{}{}{}

\dictentry{encajar}{}{}{}
{\hyperlink{val:mapti}{mapti}}{}{}{}

\dictentry{encima de}{}{}{}
{\hyperlink{val:gahu}{ga'u}}{}{}{}

\dictentry{encontrar}{}{}{}
{\hyperlink{val:penmi}{penmi}}{}{}{}

\dictentry{encuentro}{}{}{general}
{\hyperlink{val:nunjmaji}{nunjmaji}}{}{}{}

\dictentry{enemigo}{}{}{}
{\hyperlink{val:bradi}{bradi}}{}{}{}

\dictentry{energía}{}{}{}
{\hyperlink{val:nejni}{nejni}}{}{}{}

\dictentry{énfasis (sobre lo que sigue)}{}{}{}
{\hyperlink{val:bahe}{ba'e}}{}{}{}

\dictentry{enfatizar}{}{}{}
{\hyperlink{val:basna}{basna}}{}{}{}

\dictentry{enfermo}{}{}{}
{\hyperlink{val:bilma}{bilma}}{}{}{}

\dictentry{engañar}{}{}{}
{\hyperlink{val:tcica}{tcica}}{}{}{}

\dictentry{enojarse}{}{}{}
{\hyperlink{val:fengu}{fengu}}{}{}{}

\dictentry{enojo}{}{}{}
{\hyperlink{val:ohonai}{o'onai}}{}{}{}

\dictentry{ensalada}{}{}{}
{\hyperlink{val:salta}{salta}}{}{}{}

\dictentry{enseñar}{}{}{}
{\hyperlink{val:ctuca}{ctuca}}{}{}{}

\dictentry{entender}{}{}{}
{\hyperlink{val:jimpe}{jimpe}}{}{}{}

\dictentry{entrañas}{}{}{}
{\hyperlink{val:canti}{canti}}{}{}{}

\dictentry{entrar}{}{}{}
{\hyperlink{val:nerkla}{nerkla}}{}{}{}

\dictentry{entre}{}{}{}
{\hyperlink{val:jbini}{jbini}}{}{}{}

\dictentry{enviado a}{}{}{}
{\hyperlink{val:tebehi}{tebe'i}}{}{}{}

\dictentry{enviado por}{}{}{}
{\hyperlink{val:behi}{be'i}}{}{}{}

\dictentry{enviar}{}{}{}
{\hyperlink{val:mrilu}{mrilu}}{}{}{}

\dictentry{envidia}{}{}{}
{\hyperlink{val:ihonai}{i'onai}}{}{}{}

\dictentry{envidiar}{}{}{}
{\hyperlink{val:jilra}{jilra}}{}{}{}

\dictentry{equilibrio}{}{}{}
{\hyperlink{val:lanxe}{lanxe}}{}{}{}

\dictentry{era}{}{}{}
{\hyperlink{val:cedra}{cedra}}{}{}{}

\dictentry{erguirse}{}{}{}
{\hyperlink{val:sanli}{sanli}}{}{}{}

\dictentry{erizo}{}{}{}
{\hyperlink{val:ernace}{ernace}}{}{}{}

\dictentry{errar}{}{}{}
{\hyperlink{val:srera}{srera}}{}{}{}

\dictentry{es (identidad)}{}{}{}
{\hyperlink{val:du}{du}}{}{}{}

\dictentry{es antepenultimo}{}{}{}
{\hyperlink{val:daharemoi}{da'aremoi}}{}{}{}

\dictentry{es anteúltimo entre}{}{}{}
{\hyperlink{val:dahamoi}{da'amoi}}{}{}{}

\dictentry{es cuarto entre}{}{}{}
{\hyperlink{val:vomoi}{vomoi}}{}{}{}

\dictentry{es décimo entre}{}{}{}
{\hyperlink{val:panomoi}{panomoi}}{}{}{}

\dictentry{es el total de}{}{}{}
{\hyperlink{val:pirosihe}{pirosi'e}}{}{}{}

\dictentry{es específico a}{}{}{}
{\hyperlink{val:po}{po}}{}{}{}

\dictentry{es noveno entre}{}{}{}
{\hyperlink{val:somoi}{somoi}}{}{}{}

\dictentry{es octavo entre}{}{}{}
{\hyperlink{val:bimoi}{bimoi}}{}{}{}

\dictentry{es plural}{}{}{}
{\hyperlink{val:suhoremei}{su'oremei}}{}{}{}

\dictentry{es por lo menos segundo}{}{}{}
{\hyperlink{val:suhoremoi}{su'oremoi}}{}{}{}

\dictentry{es primero entre}{}{}{}
{\hyperlink{val:pamoi}{pamoi}}{}{}{}

\dictentry{es quinto entre}{}{}{}
{\hyperlink{val:mumoi}{mumoi}}{}{}{}

\dictentry{es segundo entre}{}{}{}
{\hyperlink{val:remoi}{remoi}}{}{}{}

\dictentry{es séptimo entre}{}{}{}
{\hyperlink{val:zemoi}{zemoi}}{}{}{}

\dictentry{es sexto entre}{}{}{}
{\hyperlink{val:xamoi}{xamoi}}{}{}{}

\dictentry{es singular}{}{}{}
{\hyperlink{val:pamei}{pamei}}{}{}{}

\dictentry{es suficientésimo entre}{}{}{}
{\hyperlink{val:raumoi}{raumoi}}{}{}{}

\dictentry{es tercero entre}{}{}{}
{\hyperlink{val:cimoi}{cimoi}}{}{}{}

\dictentry{es último entre}{}{}{}
{\hyperlink{val:romoi}{romoi}}{}{}{}

\dictentry{es un centenar}{}{}{}
{\hyperlink{val:panonomei}{panonomei}}{}{}{}

\dictentry{es un montón}{}{}{}
{\hyperlink{val:sohimei}{so'imei}}{}{}{}

\dictentry{es un octeto}{}{}{}
{\hyperlink{val:bimei}{bimei}}{}{}{}

\dictentry{es un par}{}{}{}
{\hyperlink{val:remei}{remei}}{}{}{}

\dictentry{es un quinteto}{}{}{}
{\hyperlink{val:mumei}{mumei}}{}{}{}

\dictentry{es un trío}{}{}{}
{\hyperlink{val:cimei}{cimei}}{}{}{}

\dictentry{es una decena}{}{}{}
{\hyperlink{val:panomei}{panomei}}{}{}{}

\dictentry{es una docena}{}{}{}
{\hyperlink{val:paremei}{paremei}}{}{}{}

\dictentry{es una quinta parte}{}{}{}
{\hyperlink{val:piresihe}{piresi'e}}{}{}{}

\dictentry{es una veintena}{}{}{}
{\hyperlink{val:renomei}{renomei}}{}{}{}

\dictentry{escala}{}{}{}
{\hyperlink{val:ckilu}{ckilu}}{}{}{}

\dictentry{escala midiendo}{}{}{}
{\hyperlink{val:secihu}{seci'u}}{}{}{}

\dictentry{escaleras}{}{}{}
{\hyperlink{val:serti}{serti}}{}{}{}

\dictentry{escepticismo}{}{}{}
{\hyperlink{val:iacuhi}{iacu'i}}{}{}{}

\dictentry{escocés}{}{}{}
{\hyperlink{val:skoto}{skoto}}{}{}{}

\dictentry{esconder}{}{}{}
{\hyperlink{val:mipri}{mipri}}{}{}{}

\dictentry{escribir}{}{}{}
{\hyperlink{val:ciska}{ciska}}{}{}{}

\dictentry{escuela}{}{}{}
{\hyperlink{val:ckule}{ckule}}{}{}{}

\dictentry{escupir}{}{}{}
{\hyperlink{val:sputu}{sputu}}{}{}{}

\dictentry{escurrirse}{}{}{}
{\hyperlink{val:rinci}{rinci}}{}{}{}

\dictentry{ese}{}{}{}
{\hyperlink{val:leva}{leva}}{}{}{}

\dictentry{ése}{}{}{}
{\hyperlink{val:ta}{ta}}{}{}{}

\dictentry{esfuerzo}{}{}{}
{\hyperlink{val:ahi}{a'i}}{}{}{}

\dictentry{eslavo}{}{}{}
{\hyperlink{val:slovo}{slovo}}{}{}{}

\dictentry{eso}{}{}{}
{\hyperlink{val:ta}{ta}}{}{}{}

\dictentry{espacio}{}{}{}
{\hyperlink{val:kensa}{kensa}}{}{}{}

\dictentry{español}{}{}{}
{\hyperlink{val:spano}{spano}}{}{}{}

\dictentry{especial}{}{}{}
{\hyperlink{val:steci}{steci}}{}{}{}

\dictentry{especie}{}{}{}
{\hyperlink{val:jutsi}{jutsi}}{}{}{}

\dictentry{especie de mango}{}{}{especie/variedad de mango}
{\hyperlink{val:mango}{se mango}}{}{}{}

\dictentry{esperanza}{}{}{}
{\hyperlink{val:aho}{a'o}}{}{}{}

\dictentry{esperar}{}{}{}
{\hyperlink{val:pacna}{pacna}}{}{}{}

\dictentry{espiral}{}{}{}
{\hyperlink{val:sarlu}{sarlu}}{}{}{}

\dictentry{espíritu}{}{}{}
{\hyperlink{val:pruxi}{pruxi}}{}{}{}

\dictentry{espiritual}{}{}{}
{\hyperlink{val:rehe}{re'e}}{}{}{}

\dictentry{esponja}{}{}{}
{\hyperlink{val:panje}{panje}}{}{}{}

\dictentry{espuma}{}{}{}
{\hyperlink{val:fonmo}{fonmo}}{}{}{}

\dictentry{esquí}{}{}{}
{\hyperlink{val:skiji}{skiji}}{}{}{}

\dictentry{está por ser}{}{}{}
{\hyperlink{val:capuho}{capu'o}}{}{}{}

\dictentry{está siendo}{}{}{}
{\hyperlink{val:cacaho}{caca'o}}{}{}{}

\dictentry{estaba siendo}{}{}{}
{\hyperlink{val:pucaho}{puca'o}}{}{}{}

\dictentry{estación}{}{}{}
{\hyperlink{val:citsi}{citsi}}{}{}{}

\dictentry{estado}{}{}{}
{\hyperlink{val:zahi}{za'i}}{}{}{}

\dictentry{estadounidense}{}{}{}
{\hyperlink{val:merko}{merko}}{}{}{}

\dictentry{estándar}{}{}{}
{\hyperlink{val:vepaha}{vepa'a}}{}{}{}

\dictentry{estándar de beneficio ..}{}{}{}
{\hyperlink{val:tevahu}{teva'u}}{}{}{}

\dictentry{estaño}{}{}{}
{\hyperlink{val:tinci}{tinci}}{}{}{}

\dictentry{estar}{}{}{}
{\hyperlink{val:zvati}{zvati}}{}{}{}

\dictentry{estará siendo}{}{}{}
{\hyperlink{val:bacaho}{baca'o}}{}{}{}

\dictentry{este}{}{}{}
{\hyperlink{val:levi}{levi}}{}{}{}

\dictentry{éste}{}{}{}
{\hyperlink{val:ti}{ti}}{}{}{}

\dictentry{estereorradián}{}{}{}
{\hyperlink{val:stero}{stero}}{}{}{}

\dictentry{estimar}{}{}{}
{\hyperlink{val:sinma}{sinma}}{}{}{}

\dictentry{esto}{}{}{}
{\hyperlink{val:ti}{ti}}{}{}{}

\dictentry{estornudar}{}{}{}
{\hyperlink{val:senci}{senci}}{}{}{}

\dictentry{estrato}{}{}{}
{\hyperlink{val:senta}{senta}}{}{}{}

\dictentry{estrella}{}{}{}
{\hyperlink{val:tarci}{tarci}}{}{}{}

\dictentry{estrés}{}{}{}
{\hyperlink{val:ohunai}{o'unai}}{}{}{}

\dictentry{estructura}{}{}{}
{\hyperlink{val:stura}{stura}}{}{}{}

\dictentry{estudiar}{}{}{}
{\hyperlink{val:tadni}{tadni}}{}{}{}

\dictentry{etéreo}{}{}{}
{\hyperlink{val:mucti}{mucti}}{}{}{}

\dictentry{etiqueta}{}{}{}
{\hyperlink{val:tcita}{tcita}}{}{}{}

\dictentry{europeo}{}{}{}
{\hyperlink{val:ropno}{ropno}}{}{}{}

\dictentry{evasión}{}{}{}
{\hyperlink{val:ahanai}{a'anai}}{}{}{}

\dictentry{evento}{}{}{}
{\hyperlink{val:nu}{nu}}{}{}{}

\dictentry{evento puntual}{}{}{}
{\hyperlink{val:muhe}{mu'e}}{}{}{}

\dictentry{evitar}{}{}{}
{\hyperlink{val:rivbi}{rivbi}}{}{}{}

\dictentry{exa}{}{}{}
{\hyperlink{val:xexso}{xexso}}{}{}{}

\dictentry{exactitud}{}{}{}
{\hyperlink{val:bahucuhi}{ba'ucu'i}}{}{}{}

\dictentry{exacto}{}{}{}
{\hyperlink{val:satci}{satci}}{}{}{}

\dictentry{exageración}{}{}{}
{\hyperlink{val:bahu}{ba'u}}{}{}{}

\dictentry{exceder}{}{}{}
{\hyperlink{val:bancu}{bancu}}{}{}{}

\dictentry{excremento}{}{}{}
{\hyperlink{val:kalci}{kalci}}{}{}{}

\dictentry{excretar}{}{}{}
{\hyperlink{val:vikmi}{vikmi}}{}{}{}

\dictentry{existir}{}{}{}
{\hyperlink{val:zasti}{zasti}}{}{}{}

\dictentry{expandiendo el tanru}{}{}{}
{\hyperlink{val:tahunai}{ta'unai}}{}{}{}

\dictentry{expandirse}{}{}{}
{\hyperlink{val:preja}{preja}}{}{}{}

\dictentry{expectativa}{}{}{}
{\hyperlink{val:uenai}{uenai}}{}{}{}

\dictentry{experiencia}{}{}{}
{\hyperlink{val:bahacuhi}{ba'acu'i}}{}{}{}

\dictentry{experimentado por}{}{}{}
{\hyperlink{val:rihi}{ri'i}}{}{}{}

\dictentry{experimentando}{}{}{}
{\hyperlink{val:serihi}{seri'i}}{}{}{}

\dictentry{experimentar}{}{}{}
{\hyperlink{val:lifri}{lifri}}{}{}{}

\dictentry{experto}{}{}{}
{\hyperlink{val:certu}{certu}}{}{}{}

\dictentry{explicar}{}{}{}
{\hyperlink{val:ciksi}{ciksi}}{}{}{}

\dictentry{explotar}{}{}{}
{\hyperlink{val:spoja}{spoja}}{}{}{}

\dictentry{exponencial}{}{}{}
{\hyperlink{val:tenfa}{tenfa}}{}{}{}

\dictentry{exponencial e}{}{}{}
{\hyperlink{val:teho}{te'o}}{}{}{}

\dictentry{exportar}{}{}{}
{\hyperlink{val:barbei}{barbei}}{}{}{}

\dictentry{expresado en medio ...}{}{}{}
{\hyperlink{val:vecuhu}{vecu'u}}{}{}{}

\dictentry{expresando}{}{}{}
{\hyperlink{val:secuhu}{secu'u}}{}{}{}

\dictentry{expresión actual}{}{}{}
{\hyperlink{val:dei}{dei}}{}{}{}

\dictentry{expresión anterior}{}{}{}
{\hyperlink{val:dahu}{da'u}}{}{}{}

\dictentry{expresión eventual}{}{}{}
{\hyperlink{val:dahe}{da'e}}{}{}{}

\dictentry{expresión no especificada}{}{}{}
{\hyperlink{val:dohi}{do'i}}{}{}{}

\dictentry{expresión próxima}{}{}{}
{\hyperlink{val:dehe}{de'e}}{}{}{}

\dictentry{expresión reciente}{}{}{}
{\hyperlink{val:dehu}{de'u}}{}{}{}

\dictentry{expresión siguiente}{}{}{}
{\hyperlink{val:dihe}{di'e}}{}{}{}

\dictentry{expresión última}{}{}{}
{\hyperlink{val:dihu}{di'u}}{}{}{}

\dictentry{extenderse}{}{}{}
{\hyperlink{val:tcena}{tcena}}{}{}{}

\dictentry{extranjero}{}{}{}
{\hyperlink{val:fange}{fange}}{}{}{}

\dictentry{extraño}{}{}{}
{\hyperlink{val:cizra}{cizra}}{}{}{}

\dictentry{extremo}{}{}{}
{\hyperlink{val:traji}{traji}}{}{}{}

\dictchar{F}\phantomsection\addcontentsline{toc}{section}{F}
\dictentry{f}{}{}{}
{\hyperlink{val:fy}{fy}}{}{}{}

\dictentry{F}{}{}{elemento}
{\hyperlink{val:lihorkliru}{li'orkliru}}{}{}{}

\dictentry{fábrica}{}{}{}
{\hyperlink{val:fanri}{fanri}}{}{}{}

\dictentry{fácil}{}{}{}
{\hyperlink{val:fuhi}{fu'i}}{}{}{}

\dictentry{factorial}{}{}{}
{\hyperlink{val:fampihi}{fampi'i}}{}{}{}

\dictentry{factura}{}{}{}
{\hyperlink{val:janta}{janta}}{}{}{}

\dictentry{falda}{}{}{}
{\hyperlink{val:skaci}{skaci}}{}{}{}

\dictentry{falla}{}{}{}
{\hyperlink{val:cfila}{cfila}}{}{}{}

\dictentry{fallar}{}{}{}
{\hyperlink{val:fliba}{fliba}}{}{}{}

\dictentry{falsedad}{}{}{}
{\hyperlink{val:jehunai}{je'unai}}{}{}{}

\dictentry{falso}{}{}{}
{\hyperlink{val:jitfa}{jitfa}}{}{}{}

\dictentry{falta de arrepentimiento}{}{}{}
{\hyperlink{val:uhucuhi}{u'ucu'i}}{}{}{}

\dictentry{falta de respeto}{}{}{}
{\hyperlink{val:ionai}{ionai}}{}{}{}

\dictentry{familia}{}{}{}
{\hyperlink{val:lanzu}{lanzu}}{}{}{}

\dictentry{familiaridad}{}{}{}
{\hyperlink{val:ihu}{i'u}}{}{}{}

\dictentry{famoso}{}{}{}
{\hyperlink{val:misno}{misno}}{}{}{}

\dictentry{favorable}{}{}{}
{\hyperlink{val:zabna}{zabna}}{}{}{}

\dictentry{fecha}{}{}{}
{\hyperlink{val:detri}{detri}}{}{}{}

\dictentry{fecha de}{}{}{}
{\hyperlink{val:sedehi}{sede'i}}{}{}{}

\dictentry{fecha en lugar ...}{}{}{}
{\hyperlink{val:tedehi}{tede'i}}{}{}{}

\dictentry{felicidad}{}{}{}
{\hyperlink{val:ui}{ui}}{}{}{}

\dictentry{feliz}{}{}{}
{\hyperlink{val:gleki}{gleki}}{}{}{}

\dictentry{femto}{}{}{}
{\hyperlink{val:femti}{femti}}{}{}{}

\dictentry{fértil}{}{}{}
{\hyperlink{val:ferti}{ferti}}{}{}{}

\dictentry{ficción}{}{}{}
{\hyperlink{val:cfika}{cfika}}{}{}{}

\dictentry{figurativo}{}{}{}
{\hyperlink{val:peha}{pe'a}}{}{}{}

\dictentry{fija tiempo verbal}{}{}{}
{\hyperlink{val:ki}{ki}}{}{}{}

\dictentry{filme}{}{}{}
{\hyperlink{val:skina}{skina}}{}{}{}

\dictentry{fin}{}{}{}
{\hyperlink{val:fanmo}{fanmo}}{}{}{}

\dictentry{fin abstracción}{}{}{}
{\hyperlink{val:kei}{kei}}{}{}{}

\dictentry{fin agrupación}{}{}{}
{\hyperlink{val:kehe}{ke'e}}{}{}{}

\dictentry{fin bridi simple}{}{}{}
{\hyperlink{val:vau}{vau}}{}{}{}

\dictentry{fin de alcance de indicador}{}{}{}
{\hyperlink{val:fuho}{fu'o}}{}{}{}

\dictentry{fin de alcance de texto}{}{}{}
{\hyperlink{val:tuhu}{tu'u}}{}{}{}

\dictentry{fin de calificadores de sumti}{}{}{}
{\hyperlink{val:luhu}{lu'u}}{}{}{}

\dictentry{fin de cita}{}{}{}
{\hyperlink{val:lihu}{li'u}}{}{}{}

\dictentry{fin de cita de error}{}{}{}
{\hyperlink{val:lehu}{le'u}}{}{}{}

\dictentry{fin de cláusula relativa}{}{}{}
{\hyperlink{val:kuho}{ku'o}}{}{}{}

\dictentry{fin de conversión matemática}{}{}{}
{\hyperlink{val:tehu}{te'u}}{}{}{}

\dictentry{fin de ejemplos}{}{}{}
{\hyperlink{val:muhanai}{mu'anai}}{}{}{}

\dictentry{fin de emoción}{}{}{}
{\hyperlink{val:buhonai}{bu'onai}}{}{}{}

\dictentry{fin de frase relativa}{}{}{}
{\hyperlink{val:gehu}{ge'u}}{}{}{}

\dictentry{fin de lerfu compuesto}{}{}{}
{\hyperlink{val:foi}{foi}}{}{}{}

\dictentry{fin de mex premeditada}{}{}{}
{\hyperlink{val:kuhe}{ku'e}}{}{}{}

\dictentry{fin de modal de selbri}{}{}{}
{\hyperlink{val:fehu}{fe'u}}{}{}{}

\dictentry{fin de número o lerfu}{}{}{}
{\hyperlink{val:boi}{boi}}{}{}{}

\dictentry{fin de prenexo}{}{}{}
{\hyperlink{val:zohu}{zo'u}}{}{}{}

\dictentry{fin de sumti}{}{}{}
{\hyperlink{val:ku}{ku}}{}{}{}

\dictentry{fin de sumti a selbri}{}{}{}
{\hyperlink{val:mehu}{me'u}}{}{}{}

\dictentry{fin de sumti acoplados}{}{}{}
{\hyperlink{val:beho}{be'o}}{}{}{}

\dictentry{fin de sumti de expr. matem.}{}{}{}
{\hyperlink{val:loho}{lo'o}}{}{}{}

\dictentry{fin de texto}{}{}{}
{\hyperlink{val:faho}{fa'o}}{}{}{}

\dictentry{fin de vocativo}{}{}{}
{\hyperlink{val:dohu}{do'u}}{}{}{}

\dictentry{fin discursivo}{}{}{}
{\hyperlink{val:sehu}{se'u}}{}{}{}

\dictentry{fin grupo de términos}{}{}{}
{\hyperlink{val:nuhu}{nu'u}}{}{}{}

\dictentry{finalmente}{}{}{}
{\hyperlink{val:romai}{romai}}{}{}{}

\dictentry{físico}{}{}{}
{\hyperlink{val:roho}{ro'o}}{}{}{}

\dictentry{flauta}{}{}{}
{\hyperlink{val:flani}{flani}}{}{}{}

\dictentry{flojo}{}{}{}
{\hyperlink{val:kluza}{kluza}}{}{}{}

\dictentry{flor}{}{}{}
{\hyperlink{val:xrula}{xrula}}{}{}{}

\dictentry{flotar}{}{}{}
{\hyperlink{val:fulta}{fulta}}{}{}{}

\dictentry{flujo}{}{}{}
{\hyperlink{val:flecu}{flecu}}{}{}{}

\dictentry{fluor}{}{}{}
{\hyperlink{val:lihorkliru}{li'orkliru}}{}{}{}

\dictentry{foca}{}{}{}
{\hyperlink{val:pinpedi}{pinpedi}}{}{}{}

\dictentry{forma}{}{}{}
{\hyperlink{val:tarmi}{tarmi}}{}{}{}

\dictentry{fórmula}{}{}{}
{\hyperlink{val:mekso}{mekso}}{}{}{}

\dictentry{forzando}{}{}{}
{\hyperlink{val:sebai}{sebai}}{}{}{}

\dictentry{forzar}{}{}{}
{\hyperlink{val:bapli}{bapli}}{}{}{}

\dictentry{fósforo}{}{}{}
{\hyperlink{val:sacki}{sacki}}{}{}{}

\dictentry{Fr}{}{}{}
{\hyperlink{val:fasysodna}{fasysodna}}{}{}{}

\dictentry{fracción}{}{}{}
{\hyperlink{val:frinu}{frinu}}{}{}{}

\dictentry{frambuesa}{}{}{}
{\hyperlink{val:frambesi}{frambesi}}{}{}{}

\dictentry{francés}{}{}{}
{\hyperlink{val:fraso}{fraso}}{}{}{}

\dictentry{francio}{}{}{}
{\hyperlink{val:fasysodna}{fasysodna}}{}{}{}

\dictentry{frase}{}{}{}
{\hyperlink{val:jufra}{jufra}}{}{}{}

\dictentry{frase incidental}{}{}{}
{\hyperlink{val:ne}{ne}}{}{}{}

\dictentry{frase restrictiva}{}{}{}
{\hyperlink{val:pe}{pe}}{}{}{}

\dictentry{frecuente}{}{}{}
{\hyperlink{val:cafne}{cafne}}{}{}{}

\dictentry{frenar}{}{}{}
{\hyperlink{val:jabre}{jabre}}{}{}{}

\dictentry{frente}{}{}{}
{\hyperlink{val:mebri}{mebri}}{}{}{}

\dictentry{frente a}{}{}{}
{\hyperlink{val:cahu}{ca'u}}{}{}{}

\dictentry{fresa}{}{}{}
{\hyperlink{val:fragari}{fragari}}{}{}{}

\dictentry{fricción}{}{}{}
{\hyperlink{val:mosra}{mosra}}{}{}{}

\dictentry{frío}{}{}{}
{\hyperlink{val:lenku}{lenku}}{}{}{}

\dictentry{frívolo}{}{}{}
{\hyperlink{val:xalbo}{xalbo}}{}{}{}

\dictentry{fruncir ceño}{}{}{}
{\hyperlink{val:frumu}{frumu}}{}{}{}

\dictentry{frustrarse}{}{}{}
{\hyperlink{val:steba}{steba}}{}{}{}

\dictentry{fruta}{}{}{}
{\hyperlink{val:grute}{grute}}{}{}{}

\dictentry{frutilla}{}{}{}
{\hyperlink{val:fragari}{fragari}}{}{}{}

\dictentry{fuego}{}{}{}
{\hyperlink{val:fagri}{fagri}}{}{}{}

\dictentry{fuerte}{}{}{}
{\hyperlink{val:tsali}{tsali}}{}{}{}

\dictentry{función}{}{}{}
{\hyperlink{val:fancu}{fancu}}{}{}{}

\dictentry{futuro}{}{}{}
{\hyperlink{val:balvi}{balvi}}{}{}{}

\dictchar{G}\phantomsection\addcontentsline{toc}{section}{G}
\dictentry{g}{}{}{}
{\hyperlink{val:gy}{gy}}{}{}{}

\dictentry{gacela}{}{}{}
{\hyperlink{val:dorkada}{dorkada}}{}{}{}

\dictentry{gallina}{}{}{}
{\hyperlink{val:jipci}{jipci}}{}{}{}

\dictentry{galón}{}{}{}
{\hyperlink{val:dekpu}{dekpu}}{}{}{}

\dictentry{ganancia}{}{}{}
{\hyperlink{val:uha}{u'a}}{}{}{}

\dictentry{ganar}{}{}{}
{\hyperlink{val:jinga}{jinga}}{}{}{}

\dictentry{gancho}{}{}{}
{\hyperlink{val:genxu}{genxu}}{}{}{}

\dictentry{ganso}{}{}{}
{\hyperlink{val:gunse}{gunse}}{}{}{}

\dictentry{garbanzo}{}{}{}
{\hyperlink{val:dembi}{dembi}}{}{}{}

\dictentry{garganta}{}{}{}
{\hyperlink{val:galxe}{galxe}}{}{}{}

\dictentry{garra}{}{}{}
{\hyperlink{val:jgalu}{jgalu}}{}{}{}

\dictentry{gas}{}{}{}
{\hyperlink{val:gapci}{gapci}}{}{}{}

\dictentry{gaseosa}{}{}{}
{\hyperlink{val:sodva}{sodva}}{}{}{}

\dictentry{gastar}{}{}{}
{\hyperlink{val:xaksu}{xaksu}}{}{}{}

\dictentry{gato}{}{}{}
{\hyperlink{val:mlatu}{mlatu}}{}{}{}

\dictentry{gelatina}{}{}{}
{\hyperlink{val:jduli}{jduli}}{}{}{}

\dictentry{gema}{}{}{}
{\hyperlink{val:jemna}{jemna}}{}{}{}

\dictentry{gemir}{}{}{}
{\hyperlink{val:cmoni}{cmoni}}{}{}{}

\dictentry{gen}{}{}{}
{\hyperlink{val:jgina}{jgina}}{}{}{}

\dictentry{generación}{}{}{}
{\hyperlink{val:rorlei}{rorlei}}{}{}{}

\dictentry{generalización}{}{}{}
{\hyperlink{val:suha}{su'a}}{}{}{}

\dictentry{generosidad}{}{}{}
{\hyperlink{val:doha}{do'a}}{}{}{}

\dictentry{geometría}{}{}{}
{\hyperlink{val:caltaicmaci}{caltaicmaci}}{}{}{}

\dictentry{geometria}{}{}{}
{\hyperlink{val:caltaicmaci}{caltaicmaci}}{}{}{}

\dictentry{germen}{}{}{}
{\hyperlink{val:jurme}{jurme}}{}{}{}

\dictentry{giga}{}{}{}
{\hyperlink{val:gigdo}{gigdo}}{}{}{}

\dictentry{gimnasta}{}{}{}
{\hyperlink{val:zajba}{zajba}}{}{}{}

\dictentry{girar}{}{}{}
{\hyperlink{val:carna}{carna}}{}{}{}

\dictentry{glándula}{}{}{}
{\hyperlink{val:cigla}{cigla}}{}{}{}

\dictentry{gobernar}{}{}{}
{\hyperlink{val:turni}{turni}}{}{}{}

\dictentry{golpear}{}{}{}
{\hyperlink{val:darxi}{darxi}}{}{}{}

\dictentry{goma}{}{}{}
{\hyperlink{val:ckabu}{ckabu}}{}{}{}

\dictentry{gordo}{}{}{}
{\hyperlink{val:plana}{plana}}{}{}{}

\dictentry{gorgojo}{}{}{}
{\hyperlink{val:kurkuli}{kurkuli}}{}{}{}

\dictentry{gota}{}{}{}
{\hyperlink{val:dirgo}{dirgo}}{}{}{}

\dictentry{gracias}{}{}{}
{\hyperlink{val:kihe}{ki'e}}{}{}{}

\dictentry{grado}{}{}{}
{\hyperlink{val:gradu}{gradu}}{}{}{}

\dictentry{gramatica}{}{}{}
{\hyperlink{val:gerna}{gerna}}{}{}{}

\dictentry{gramo}{}{}{}
{\hyperlink{val:grake}{grake}}{}{}{}

\dictentry{grande}{}{}{}
{\hyperlink{val:barda}{barda}}{}{}{}

\dictentry{grandioso}{}{}{}
{\hyperlink{val:banli}{banli}}{}{}{}

\dictentry{granizo}{}{}{}
{\hyperlink{val:bratu}{bratu}}{}{}{}

\dictentry{granja}{}{}{}
{\hyperlink{val:cange}{cange}}{}{}{}

\dictentry{grano}{}{}{}
{\hyperlink{val:gurni}{gurni}}{}{}{}

\dictentry{grasa}{}{}{}
{\hyperlink{val:grasu}{grasu}}{}{}{}

\dictentry{gratis}{}{}{}
{\hyperlink{val:nonseldiha}{nonseldi'a}}{}{}{}

\dictentry{griego}{}{}{}
{\hyperlink{val:xelso}{xelso}}{}{}{}

\dictentry{grieta}{}{}{}
{\hyperlink{val:fenra}{fenra}}{}{}{}

\dictentry{gris}{}{}{}
{\hyperlink{val:grusi}{grusi}}{}{}{}

\dictentry{gritar}{}{}{}
{\hyperlink{val:krixa}{krixa}}{}{}{}

\dictentry{grueso}{}{}{}
{\hyperlink{val:rotsu}{rotsu}}{}{}{}

\dictentry{grupo}{}{}{}
{\hyperlink{val:girzu}{girzu}}{}{}{}

\dictentry{guabairo}{}{}{}
{\hyperlink{val:ctecmocpi}{ctecmocpi}}{}{}{}

\dictentry{guante}{}{}{}
{\hyperlink{val:gluta}{gluta}}{}{}{}

\dictentry{guardar}{}{}{}
{\hyperlink{val:ralte}{ralte}}{}{}{}

\dictentry{guerrear}{}{}{}
{\hyperlink{val:jamna}{jamna}}{}{}{}

\dictentry{guiar}{}{}{}
{\hyperlink{val:gidva}{gidva}}{}{}{}

\dictentry{guitarra}{}{}{}
{\hyperlink{val:jgita}{jgita}}{}{}{}

\dictentry{gusano}{}{}{}
{\hyperlink{val:curnu}{curnu}}{}{}{}

\dictentry{gustar}{}{}{}
{\hyperlink{val:kukte}{kukte}}{}{}{}

\dictentry{gusto}{}{}{}
{\hyperlink{val:vrusi}{vrusi}}{}{}{}

\dictchar{H}\phantomsection\addcontentsline{toc}{section}{H}
\dictentry{ha sido}{}{}{}
{\hyperlink{val:cabaho}{caba'o}}{}{}{}

\dictentry{había sido antes}{}{}{}
{\hyperlink{val:pupu}{pupu}}{}{}{}

\dictentry{habitar}{}{}{}
{\hyperlink{val:xabju}{xabju}}{}{}{}

\dictentry{habitualmente}{}{}{}
{\hyperlink{val:tahe}{ta'e}}{}{}{}

\dictentry{hablar}{}{}{}
{\hyperlink{val:tavla}{tavla}}{}{}{}

\dictentry{habrá sido}{}{}{}
{\hyperlink{val:bapu}{bapu}}{}{}{}

\dictentry{hacedor de política}{}{}{}
{\hyperlink{val:plajva}{se plajva}}{}{}{}

\dictentry{hacer}{}{}{}
{\hyperlink{val:zbasu}{zbasu}}{}{}{}

\dictentry{hacerse}{}{}{}
{\hyperlink{val:binxo}{binxo}}{}{}{}

\dictentry{hacia}{}{}{}
{\hyperlink{val:faha}{fa'a}}{}{}{}

\dictentry{hacia abajo}{}{}{}
{\hyperlink{val:mohiniha}{mo'ini'a}}{}{}{}

\dictentry{hacia adelante}{}{}{}
{\hyperlink{val:mohicahu}{mo'ica'u}}{}{}{}

\dictentry{hacia adentro}{}{}{}
{\hyperlink{val:mohinehi}{mo'ine'i}}{}{}{}

\dictentry{hacia afuera}{}{}{}
{\hyperlink{val:mohizeho}{mo'ize'o}}{}{}{}

\dictentry{hacia arriba}{}{}{}
{\hyperlink{val:mohigahu}{mo'iga'u}}{}{}{}

\dictentry{hacia atrás}{}{}{}
{\hyperlink{val:mohitiha}{mo'iti'a}}{}{}{}

\dictentry{hacia el este}{}{}{}
{\hyperlink{val:mohiduha}{mo'idu'a}}{}{}{}

\dictentry{hacia el norte}{}{}{}
{\hyperlink{val:mohibeha}{mo'ibe'a}}{}{}{}

\dictentry{hacia el oeste}{}{}{}
{\hyperlink{val:mohivuha}{mo'ivu'a}}{}{}{}

\dictentry{hacia el sur}{}{}{}
{\hyperlink{val:mohinehu}{mo'ine'u}}{}{}{}

\dictentry{hacia fuera de}{}{}{}
{\hyperlink{val:toho}{to'o}}{}{}{}

\dictentry{hacia la derecha}{}{}{}
{\hyperlink{val:mohirihu}{mo'iri'u}}{}{}{}

\dictentry{hacia la izquierda}{}{}{}
{\hyperlink{val:mohizuha}{mo'izu'a}}{}{}{}

\dictentry{hada}{}{}{}
{\hyperlink{val:crida}{crida}}{}{}{}

\dictentry{hamaca}{}{}{}
{\hyperlink{val:dadycka}{dadycka}}{}{}{}

\dictentry{has even odds}{}{}{}
{\hyperlink{val:pimucuho}{pimucu'o}}{}{}{}

\dictentry{hasta}{}{}{}
{\hyperlink{val:mohibuhu}{mo'ibu'u}}{}{}{}

\dictentry{hasta el borde de}{}{}{}
{\hyperlink{val:fehepuho}{fe'epu'o}}{}{}{}

\dictentry{hasta límite}{}{}{}
{\hyperlink{val:jihe}{ji'e}}{}{}{}

\dictentry{He}{}{}{}
{\hyperlink{val:solnavni}{solnavni}}{}{}{}

\dictentry{hebreo}{}{}{}
{\hyperlink{val:xebro}{xebro}}{}{}{}

\dictentry{hecho}{}{}{}
{\hyperlink{val:fatci}{fatci}}{}{}{}

\dictentry{hecho de}{}{}{}
{\hyperlink{val:semahe}{sema'e}}{}{}{}

\dictentry{hecto}{}{}{}
{\hyperlink{val:xecto}{xecto}}{}{}{}

\dictentry{helecho}{}{}{}
{\hyperlink{val:filcina}{filcina}}{}{}{}

\dictentry{helio}{}{}{}
{\hyperlink{val:solnavni}{solnavni}}{}{}{}

\dictentry{hembra}{}{}{}
{\hyperlink{val:fetsi}{fetsi}}{}{}{}

\dictentry{heredar}{}{}{}
{\hyperlink{val:cerda}{cerda}}{}{}{}

\dictentry{herir}{}{}{}
{\hyperlink{val:xrani}{xrani}}{}{}{}

\dictentry{hermana}{}{}{}
{\hyperlink{val:mensi}{mensi}}{}{}{}

\dictentry{hermano}{}{}{}
{\hyperlink{val:tunba}{tunba}}{}{}{}

\dictentry{herramienta}{}{}{}
{\hyperlink{val:tutci}{tutci}}{}{}{}

\dictentry{hervir}{}{}{}
{\hyperlink{val:febvi}{febvi}}{}{}{}

\dictentry{hidrógeno}{}{}{}
{\hyperlink{val:cidro}{cidro}}{}{}{}

\dictentry{hielo}{}{}{}
{\hyperlink{val:bisli}{bisli}}{}{}{}

\dictentry{hierro}{}{}{}
{\hyperlink{val:tirse}{tirse}}{}{}{}

\dictentry{hígado}{}{}{}
{\hyperlink{val:livga}{livga}}{}{}{}

\dictentry{higo}{}{}{}
{\hyperlink{val:figre}{figre}}{}{}{}

\dictentry{hija}{}{}{}
{\hyperlink{val:tixnu}{tixnu}}{}{}{}

\dictentry{hijo}{}{}{}
{\hyperlink{val:bersa}{bersa}}{}{}{}

\dictentry{hilo}{}{}{}
{\hyperlink{val:cilta}{cilta}}{}{}{}

\dictentry{hinchazón}{}{}{}
{\hyperlink{val:punli}{punli}}{}{}{}

\dictentry{hindú}{}{}{}
{\hyperlink{val:xindo}{xindo}}{}{}{}

\dictentry{hispánico}{}{}{}
{\hyperlink{val:xispo}{xispo}}{}{}{}

\dictentry{historia}{}{}{}
{\hyperlink{val:citri}{citri}}{}{}{}

\dictentry{hoja}{}{}{}
{\hyperlink{val:pezli}{pezli}}{}{}{}

\dictentry{hombre}{}{}{}
{\hyperlink{val:nanmu}{nanmu}}{}{}{}

\dictentry{hombro}{}{}{}
{\hyperlink{val:janco}{janco}}{}{}{}

\dictentry{hondo}{}{}{}
{\hyperlink{val:condi}{condi}}{}{}{}

\dictentry{hongo}{}{}{}
{\hyperlink{val:mledi}{mledi}}{}{}{}

\dictentry{hora}{}{}{}
{\hyperlink{val:cacra}{cacra}}{}{}{}

\dictentry{hora de (día)}{}{}{}
{\hyperlink{val:tetihu}{teti'u}}{}{}{}

\dictentry{hora en (lugar)}{}{}{}
{\hyperlink{val:vetihu}{veti'u}}{}{}{}

\dictentry{horario}{}{}{}
{\hyperlink{val:tcika}{tcika}}{}{}{}

\dictentry{horizontal}{}{}{}
{\hyperlink{val:pinta}{pinta}}{}{}{}

\dictentry{hormiga}{}{}{}
{\hyperlink{val:manti}{manti}}{}{}{}

\dictentry{horno}{}{}{}
{\hyperlink{val:toknu}{toknu}}{}{}{}

\dictentry{hospital}{}{}{}
{\hyperlink{val:spita}{spita}}{}{}{}

\dictentry{hospitalidad}{}{}{}
{\hyperlink{val:fihi}{fi'i}}{}{}{}

\dictentry{hotel}{}{}{}
{\hyperlink{val:xotli}{xotli}}{}{}{}

\dictentry{hueso}{}{}{}
{\hyperlink{val:bongu}{bongu}}{}{}{}

\dictentry{huevo}{}{}{}
{\hyperlink{val:sovda}{sovda}}{}{}{}

\dictentry{humano}{}{}{}
{\hyperlink{val:remna}{remna}}{}{}{}

\dictentry{humilde}{}{}{}
{\hyperlink{val:cumla}{cumla}}{}{}{}

\dictentry{humo}{}{}{}
{\hyperlink{val:danmo}{danmo}}{}{}{}

\dictentry{humor}{}{}{}
{\hyperlink{val:zoho}{zo'o}}{}{}{}

\dictchar{I}\phantomsection\addcontentsline{toc}{section}{I}
\dictentry{i}{}{}{}
{\hyperlink{val:ibu}{ibu}}{}{}{}

\dictentry{iba a ser}{}{}{}
{\hyperlink{val:puba}{puba}}{}{}{}

\dictentry{iceberg}{}{}{}
{\hyperlink{val:bismaha}{bisma'a}}{}{}{}

\dictentry{idea}{}{}{}
{\hyperlink{val:sidbo}{sidbo}}{}{}{}

\dictentry{identidad incidental}{}{}{}
{\hyperlink{val:nohu}{no'u}}{}{}{}

\dictentry{identidad restrictiva}{}{}{}
{\hyperlink{val:pohu}{po'u}}{}{}{}

\dictentry{idioma}{}{}{}
{\hyperlink{val:bangu}{bangu}}{}{}{}

\dictentry{ido por}{}{}{}
{\hyperlink{val:kaha}{ka'a}}{}{}{}

\dictentry{ignoreme}{}{}{}
{\hyperlink{val:juhinai}{ju'inai}}{}{}{}

\dictentry{igual}{}{}{}
{\hyperlink{val:dunli}{dunli}}{}{}{}

\dictentry{igual a}{}{}{}
{\hyperlink{val:seduhi}{sedu'i}}{}{}{}

\dictentry{igual en propiedad ...}{}{}{}
{\hyperlink{val:teduhi}{tedu'i}}{}{}{}

\dictentry{igual rango}{}{}{}
{\hyperlink{val:gahicuhi}{ga'icu'i}}{}{}{}

\dictentry{igualmente}{}{}{}
{\hyperlink{val:rahucuhi}{ra'ucu'i}}{}{}{}

\dictentry{imaginario}{}{}{}
{\hyperlink{val:xanri}{xanri}}{}{}{}

\dictentry{imaginario i}{}{}{}
{\hyperlink{val:kaho}{ka'o}}{}{}{}

\dictentry{imán}{}{}{de brújula}
{\hyperlink{val:makfartci}{se makfartci}}{}{}{}

\dictentry{imán}{}{}{magnetita}
{\hyperlink{val:maksi}{maksi}}{}{}{}

\dictentry{impedir}{}{}{}
{\hyperlink{val:fanta}{fanta}}{}{}{}

\dictentry{imperativo}{}{}{}
{\hyperlink{val:ko}{ko}}{}{}{}

\dictentry{imperio}{}{}{}
{\hyperlink{val:sorgugjeha}{sorgugje'a}}{}{}{}

\dictentry{implacable}{}{}{}
{\hyperlink{val:tolkehi}{tolke'i}}{}{}{}

\dictentry{implicar}{}{}{}
{\hyperlink{val:nibli}{nibli}}{}{}{}

\dictentry{importante}{}{}{}
{\hyperlink{val:vajni}{vajni}}{}{}{}

\dictentry{importar}{}{}{}
{\hyperlink{val:nerbei}{nerbei}}{}{}{}

\dictentry{imposibilidad}{}{}{}
{\hyperlink{val:juhonai}{ju'onai}}{}{}{}

\dictentry{imprecisión}{}{}{}
{\hyperlink{val:sahenai}{sa'enai}}{}{}{}

\dictentry{impresionar}{}{}{}
{\hyperlink{val:jenca}{jenca}}{}{}{}

\dictentry{imprimir}{}{}{}
{\hyperlink{val:prina}{prina}}{}{}{}

\dictentry{improbabilidad}{}{}{}
{\hyperlink{val:lahanai}{la'anai}}{}{}{}

\dictentry{impuesto}{}{}{}
{\hyperlink{val:cteki}{cteki}}{}{}{}

\dictentry{inaudible}{}{}{}
{\hyperlink{val:seltinynalkahe}{seltinynalka'e}}{}{}{}

\dictentry{incerteza}{}{}{}
{\hyperlink{val:juhocuhi}{ju'ocu'i}}{}{}{}

\dictentry{incidentalmente}{}{}{}
{\hyperlink{val:rahunai}{ra'unai}}{}{}{}

\dictentry{incoativo}{}{}{}
{\hyperlink{val:puho}{pu'o}}{}{}{}

\dictentry{incómodo}{}{}{}
{\hyperlink{val:burna}{burna}}{}{}{}

\dictentry{incompetencia}{}{}{}
{\hyperlink{val:ehenai}{e'enai}}{}{}{}

\dictentry{incompletitud}{}{}{}
{\hyperlink{val:uonai}{uonai}}{}{}{}

\dictentry{incompleto}{}{}{}
{\hyperlink{val:nalmuho}{nalmu'o}}{}{}{}

\dictentry{indecisión}{}{}{}
{\hyperlink{val:aicuhi}{aicu'i}}{}{}{}

\dictentry{independencia}{}{}{}
{\hyperlink{val:ehicuhi}{e'icu'i}}{}{}{}

\dictentry{indicador de 1er sumti}{}{}{}
{\hyperlink{val:fa}{fa}}{}{}{}

\dictentry{indicador de 2do sumti}{}{}{}
{\hyperlink{val:fe}{fe}}{}{}{}

\dictentry{indicador de 3er sumti}{}{}{}
{\hyperlink{val:fi}{fi}}{}{}{}

\dictentry{indicador de 4to sumti}{}{}{}
{\hyperlink{val:fo}{fo}}{}{}{}

\dictentry{indicador de 5to sumti}{}{}{}
{\hyperlink{val:fu}{fu}}{}{}{}

\dictentry{indicador de sumti adicional}{}{}{}
{\hyperlink{val:fai}{fai}}{}{}{}

\dictentry{indicador de vocativo}{}{}{}
{\hyperlink{val:doi}{doi}}{}{}{}

\dictentry{indiferencia}{}{}{}
{\hyperlink{val:iucuhi}{iucu'i}}{}{}{}

\dictentry{indiferente}{}{}{}
{\hyperlink{val:norpahi}{norpa'i}}{}{}{}

\dictentry{indonesio}{}{}{}
{\hyperlink{val:bindo}{bindo}}{}{}{}

\dictentry{industria}{}{}{}
{\hyperlink{val:gundi}{gundi}}{}{}{}

\dictentry{infante}{}{}{}
{\hyperlink{val:cifnu}{cifnu}}{}{}{}

\dictentry{infinitas veces}{}{}{}
{\hyperlink{val:cihiroi}{ci'iroi}}{}{}{}

\dictentry{infinito}{}{}{}
{\hyperlink{val:cihi}{ci'i}}{}{}{}

\dictentry{información nueva}{}{}{}
{\hyperlink{val:bihu}{bi'u}}{}{}{}

\dictentry{información previa}{}{}{}
{\hyperlink{val:bihunai}{bi'unai}}{}{}{}

\dictentry{informar}{}{}{}
{\hyperlink{val:jungau}{jungau}}{}{}{}

\dictentry{inglés}{}{}{}
{\hyperlink{val:glico}{glico}}{}{}{}

\dictentry{ingresivo}{}{}{}
{\hyperlink{val:coha}{co'a}}{}{}{}

\dictentry{inhospitalidad}{}{}{}
{\hyperlink{val:fihinai}{fi'inai}}{}{}{}

\dictentry{inocencia}{}{}{}
{\hyperlink{val:uhunai}{u'unai}}{}{}{}

\dictentry{inocente}{}{}{}
{\hyperlink{val:zernalfuhe}{zernalfu'e}}{}{}{}

\dictentry{insecto}{}{}{}
{\hyperlink{val:cinki}{cinki}}{}{}{}

\dictentry{inserción editorial}{}{}{}
{\hyperlink{val:saha}{sa'a}}{}{}{}

\dictentry{integral}{}{}{}
{\hyperlink{val:rahirsumji}{ra'irsumji}}{}{}{}

\dictentry{intención}{}{}{}
{\hyperlink{val:ai}{ai}}{}{}{}

\dictentry{intenso}{}{}{}
{\hyperlink{val:carmi}{carmi}}{}{}{}

\dictentry{intercambiar}{}{}{}
{\hyperlink{val:simbasygau}{simbasygau}}{}{}{}

\dictentry{interés}{}{}{}
{\hyperlink{val:ahu}{a'u}}{}{}{}

\dictentry{interesar}{}{}{}
{\hyperlink{val:cinri}{cinri}}{}{}{}

\dictentry{interferir}{}{}{}
{\hyperlink{val:zunti}{zunti}}{}{}{}

\dictentry{interrumpir}{}{}{}
{\hyperlink{val:dicra}{dicra}}{}{}{}

\dictentry{interrupción}{}{}{}
{\hyperlink{val:taha}{ta'a}}{}{}{}

\dictentry{intersección}{}{}{}
{\hyperlink{val:kuha}{ku'a}}{}{}{}

\dictentry{intersectar}{}{}{}
{\hyperlink{val:kruca}{kruca}}{}{}{}

\dictentry{intervalo}{}{}{}
{\hyperlink{val:sirji}{sirji}}{}{}{}

\dictentry{intervalo abierto}{}{}{}
{\hyperlink{val:kehi}{ke'i}}{}{}{}

\dictentry{intervalo cerrado}{}{}{}
{\hyperlink{val:gaho}{ga'o}}{}{}{}

\dictentry{intervalo espacial 1-D}{}{}{}
{\hyperlink{val:vihi}{vi'i}}{}{}{}

\dictentry{intervalo espacial 2-D}{}{}{}
{\hyperlink{val:viha}{vi'a}}{}{}{}

\dictentry{intervalo espacial 3-D}{}{}{}
{\hyperlink{val:vihu}{vi'u}}{}{}{}

\dictentry{intervalo espacial grande}{}{}{}
{\hyperlink{val:vehu}{ve'u}}{}{}{}

\dictentry{intervalo espacial medio}{}{}{}
{\hyperlink{val:veha}{ve'a}}{}{}{}

\dictentry{intervalo espacial pequeño}{}{}{}
{\hyperlink{val:vehi}{ve'i}}{}{}{}

\dictentry{intervalo espacial total}{}{}{}
{\hyperlink{val:vehe}{ve'e}}{}{}{}

\dictentry{intervalo especial 4-D}{}{}{}
{\hyperlink{val:vihe}{vi'e}}{}{}{}

\dictentry{intervalo no ordenado}{}{}{}
{\hyperlink{val:bihi}{bi'i}}{}{}{}

\dictentry{intervalo ordenado}{}{}{}
{\hyperlink{val:biho}{bi'o}}{}{}{}

\dictentry{intervalo temporal corto}{}{}{}
{\hyperlink{val:zehi}{ze'i}}{}{}{}

\dictentry{intervalo temporal largo}{}{}{}
{\hyperlink{val:zehu}{ze'u}}{}{}{}

\dictentry{intervalo temporal medio}{}{}{}
{\hyperlink{val:zeha}{ze'a}}{}{}{}

\dictentry{intervalo temporal total}{}{}{}
{\hyperlink{val:zehe}{ze'e}}{}{}{}

\dictentry{intuir}{}{}{}
{\hyperlink{val:jijnu}{jijnu}}{}{}{}

\dictentry{inversión de tanru}{}{}{}
{\hyperlink{val:co}{co}}{}{}{}

\dictentry{inverso}{}{}{}
{\hyperlink{val:fatne}{fatne}}{}{}{}

\dictentry{inverso aditivo}{}{}{}
{\hyperlink{val:vaha}{va'a}}{}{}{}

\dictentry{invertir}{}{}{}
{\hyperlink{val:zivle}{zivle}}{}{}{}

\dictentry{investigar}{}{}{}
{\hyperlink{val:cipcta}{cipcta}}{}{}{}

\dictentry{invierno}{}{}{}
{\hyperlink{val:dunra}{dunra}}{}{}{}

\dictentry{invisible}{}{}{}
{\hyperlink{val:selvisnalkahe}{selvisnalka'e}}{}{}{}

\dictentry{ir}{}{}{}
{\hyperlink{val:klama}{klama}}{}{}{}

\dictentry{iraquí}{}{}{}
{\hyperlink{val:rakso}{rakso}}{}{}{}

\dictentry{ironía}{}{}{}
{\hyperlink{val:ranxi}{ranxi}}{}{}{}

\dictentry{irradiar}{}{}{}
{\hyperlink{val:dirce}{dirce}}{}{}{}

\dictentry{irregular}{}{}{}
{\hyperlink{val:vitci}{vitci}}{}{}{}

\dictentry{irregularmente}{}{}{}
{\hyperlink{val:dihinai}{di'inai}}{}{}{}

\dictentry{isla}{}{}{}
{\hyperlink{val:daplu}{daplu}}{}{}{}

\dictentry{islámico}{}{}{}
{\hyperlink{val:muslo}{muslo}}{}{}{}

\dictentry{izquierda}{}{}{}
{\hyperlink{val:zunle}{zunle}}{}{}{}

\dictchar{J}\phantomsection\addcontentsline{toc}{section}{J}
\dictentry{j}{}{}{}
{\hyperlink{val:jy}{jy}}{}{}{}

\dictentry{jabón}{}{}{}
{\hyperlink{val:zbabu}{zbabu}}{}{}{}

\dictentry{japonés}{}{}{}
{\hyperlink{val:ponjo}{ponjo}}{}{}{}

\dictentry{jardin}{}{}{}
{\hyperlink{val:purdi}{purdi}}{}{}{}

\dictentry{jehovista}{}{}{}
{\hyperlink{val:jegvo}{jegvo}}{}{}{}

\dictentry{jengibre}{}{}{}
{\hyperlink{val:zingibero}{zingibero}}{}{}{}

\dictentry{jibia}{}{}{}
{\hyperlink{val:kalmari}{kalmari}}{}{}{}

\dictentry{jordano}{}{}{}
{\hyperlink{val:jordo}{jordo}}{}{}{}

\dictentry{joven}{}{}{}
{\hyperlink{val:citno}{citno}}{}{}{}

\dictentry{jugar}{}{}{}
{\hyperlink{val:kelci}{kelci}}{}{}{}

\dictentry{jugo}{}{}{}
{\hyperlink{val:jisra}{jisra}}{}{}{}

\dictentry{juntarse}{}{}{}
{\hyperlink{val:jmaji}{jmaji}}{}{}{}

\dictentry{junto a}{}{}{}
{\hyperlink{val:neha}{ne'a}}{}{}{}

\dictentry{junto con}{}{}{}
{\hyperlink{val:joi}{joi}}{}{}{}

\dictentry{juntos}{}{}{}
{\hyperlink{val:ihi}{i'i}}{}{}{}

\dictentry{justicia}{}{}{}
{\hyperlink{val:pahe}{pa'e}}{}{}{}

\dictentry{juzgar}{}{}{}
{\hyperlink{val:pajni}{pajni}}{}{}{}

\dictchar{K}\phantomsection\addcontentsline{toc}{section}{K}
\dictentry{k}{}{}{}
{\hyperlink{val:ky}{ky}}{}{}{}

\dictentry{K}{}{}{elemento}
{\hyperlink{val:sodnrkali}{sodnrkali}}{}{}{}

\dictentry{kelvin}{}{}{}
{\hyperlink{val:kelvo}{kelvo}}{}{}{}

\dictentry{kilo}{}{}{}
{\hyperlink{val:kilto}{kilto}}{}{}{}

\dictentry{kilómetro}{}{}{}
{\hyperlink{val:kihotre}{ki'otre}}{}{}{}

\dictchar{L}\phantomsection\addcontentsline{toc}{section}{L}
\dictentry{l}{}{}{}
{\hyperlink{val:ly}{ly}}{}{}{}

\dictentry{la expresión matemática}{}{}{}
{\hyperlink{val:meho}{me'o}}{}{}{}

\dictentry{la locación de}{}{}{}
{\hyperlink{val:le jaivi}{le jaivi}}{}{}{}

\dictentry{la mayoría}{}{}{}
{\hyperlink{val:sohe}{so'e}}{}{}{}

\dictentry{la mayoría de las veces}{}{}{}
{\hyperlink{val:soheroi}{so'eroi}}{}{}{}

\dictentry{la secuencia de}{}{}{}
{\hyperlink{val:vuhi}{vu'i}}{}{}{}

\dictentry{labio}{}{}{}
{\hyperlink{val:ctebi}{ctebi}}{}{}{}

\dictentry{lado}{}{}{}
{\hyperlink{val:mlana}{mlana}}{}{}{}

\dictentry{lago}{}{}{}
{\hyperlink{val:lalxu}{lalxu}}{}{}{}

\dictentry{lamentar}{}{}{}
{\hyperlink{val:xenru}{xenru}}{}{}{}

\dictentry{lámina}{}{}{}
{\hyperlink{val:boxfo}{boxfo}}{}{}{}

\dictentry{lana}{}{}{}
{\hyperlink{val:sunla}{sunla}}{}{}{}

\dictentry{lapicera}{}{}{}
{\hyperlink{val:penbi}{penbi}}{}{}{}

\dictentry{lápiz}{}{}{}
{\hyperlink{val:pinsi}{pinsi}}{}{}{}

\dictentry{largo}{}{}{}
{\hyperlink{val:clani}{clani}}{}{}{}

\dictentry{lata}{}{}{}
{\hyperlink{val:lante}{lante}}{}{}{}

\dictentry{látigo}{}{}{}
{\hyperlink{val:bikla}{bikla}}{}{}{}

\dictentry{latino}{}{}{}
{\hyperlink{val:latmo}{latmo}}{}{}{}

\dictentry{latón}{}{}{}
{\hyperlink{val:lastu}{lastu}}{}{}{}

\dictentry{lavar}{}{}{}
{\hyperlink{val:lumci}{lumci}}{}{}{}

\dictentry{lazo}{}{}{}
{\hyperlink{val:clupa}{clupa}}{}{}{}

\dictentry{leche}{}{}{}
{\hyperlink{val:ladru}{ladru}}{}{}{}

\dictentry{lechuza}{}{}{}
{\hyperlink{val:glauka}{glauka}}{}{}{}

\dictentry{leer}{}{}{}
{\hyperlink{val:tcidu}{tcidu}}{}{}{}

\dictentry{lejos}{}{}{}
{\hyperlink{val:darno}{darno}}{}{}{}

\dictentry{lengua}{}{}{}
{\hyperlink{val:tance}{tance}}{}{}{}

\dictentry{lente}{}{}{}
{\hyperlink{val:lenjo}{lenjo}}{}{}{}

\dictentry{lento}{}{}{}
{\hyperlink{val:masno}{masno}}{}{}{}

\dictentry{león}{}{}{}
{\hyperlink{val:cinfo}{cinfo}}{}{}{}

\dictentry{león marino}{}{}{}
{\hyperlink{val:pinpedi}{pinpedi}}{}{}{}

\dictentry{lerfu compuesto}{}{}{}
{\hyperlink{val:tei}{tei}}{}{}{}

\dictentry{letra}{}{}{}
{\hyperlink{val:lerfu}{lerfu}}{}{}{}

\dictentry{levanta}{}{}{}
{\hyperlink{val:lafti}{lafti}}{}{}{}

\dictentry{ley}{}{}{}
{\hyperlink{val:flalu}{flalu}}{}{}{}

\dictentry{Li}{}{}{}
{\hyperlink{val:roksodna}{roksodna}}{}{}{}

\dictentry{libanés}{}{}{}
{\hyperlink{val:lubno}{lubno}}{}{}{}

\dictentry{libertad}{}{}{}
{\hyperlink{val:einai}{einai}}{}{}{}

\dictentry{libio}{}{}{}
{\hyperlink{val:libjo}{libjo}}{}{}{}

\dictentry{libra}{}{}{}
{\hyperlink{val:bunda}{bunda}}{}{}{}

\dictentry{libre}{}{}{}
{\hyperlink{val:zifre}{zifre}}{}{}{}

\dictentry{libro}{}{}{}
{\hyperlink{val:cukta}{cukta}}{}{}{}

\dictentry{licor}{}{}{}
{\hyperlink{val:jikru}{jikru}}{}{}{}

\dictentry{limitado en (propiedad)}{}{}{}
{\hyperlink{val:tejihe}{teji'e}}{}{}{}

\dictentry{limitado por}{}{}{}
{\hyperlink{val:koi}{koi}}{}{}{}

\dictentry{limitando a}{}{}{}
{\hyperlink{val:sejihe}{seji'e}}{}{}{}

\dictentry{límite}{}{}{}
{\hyperlink{val:jimte}{jimte}}{}{}{}

\dictentry{limpio}{}{}{}
{\hyperlink{val:jinsa}{jinsa}}{}{}{}

\dictentry{línea}{}{}{}
{\hyperlink{val:linji}{linji}}{}{}{}

\dictentry{lino}{}{}{}
{\hyperlink{val:matli}{matli}}{}{}{}

\dictentry{líquido}{}{}{}
{\hyperlink{val:litki}{litki}}{}{}{}

\dictentry{lista}{}{}{}
{\hyperlink{val:liste}{liste}}{}{}{}

\dictentry{listo}{}{}{}
{\hyperlink{val:bredi}{bredi}}{}{}{}

\dictentry{listo para recibir}{}{}{}
{\hyperlink{val:rehi}{re'i}}{}{}{}

\dictentry{litio}{}{}{}
{\hyperlink{val:roksodna}{roksodna}}{}{}{}

\dictentry{litro}{}{}{}
{\hyperlink{val:litce}{litce}}{}{}{}

\dictentry{liviano}{}{}{}
{\hyperlink{val:linto}{linto}}{}{}{}

\dictentry{llave}{}{}{}
{\hyperlink{val:ckiku}{ckiku}}{}{}{}

\dictentry{llegando a}{}{}{}
{\hyperlink{val:mohifaha}{mo'ifa'a}}{}{}{}

\dictentry{llenar}{}{}{}
{\hyperlink{val:tisna}{tisna}}{}{}{}

\dictentry{llenarse}{}{}{}
{\hyperlink{val:tisna}{tisna}}{}{}{}

\dictentry{lleno}{}{}{}
{\hyperlink{val:culno}{culno}}{}{}{}

\dictentry{llevar}{}{}{}
{\hyperlink{val:bevri}{bevri}}{}{}{}

\dictentry{llorar}{}{}{}
{\hyperlink{val:klaku}{klaku}}{}{}{}

\dictentry{llover}{}{}{}
{\hyperlink{val:carvi}{carvi}}{}{}{}

\dictentry{lluvia}{}{}{}
{\hyperlink{val:sicpi}{sicpi}}{}{}{}

\dictentry{lo anterior}{}{}{}
{\hyperlink{val:lahedihu}{la'edi'u}}{}{}{}

\dictentry{lo dicho}{}{}{}
{\hyperlink{val:mihu}{mi'u}}{}{}{}

\dictentry{lobo}{}{}{}
{\hyperlink{val:labno}{labno}}{}{}{}

\dictentry{local}{}{}{}
{\hyperlink{val:diklo}{diklo}}{}{}{}

\dictentry{locura}{}{}{}
{\hyperlink{val:fenki}{fenki}}{}{}{}

\dictentry{logaritmo}{}{}{}
{\hyperlink{val:deho}{de'o}}{}{}{}

\dictentry{logaritmo natural}{}{}{}
{\hyperlink{val:rardugri}{rardugri}}{}{}{}

\dictentry{logaritmo neperiano}{}{}{}
{\hyperlink{val:rardugri}{rardugri}}{}{}{}

\dictentry{lógica}{}{}{}
{\hyperlink{val:logji}{logji}}{}{}{}

\dictentry{lograr}{}{}{}
{\hyperlink{val:snada}{snada}}{}{}{}

\dictentry{logrativo}{}{}{}
{\hyperlink{val:cohi}{co'i}}{}{}{}

\dictentry{lojbánico}{}{}{}
{\hyperlink{val:lojbo}{lojbo}}{}{}{}

\dictentry{los llamados}{}{}{}
{\hyperlink{val:lai}{lai}}{}{}{}

\dictentry{loto}{}{}{}
{\hyperlink{val:latna}{latna}}{}{}{}

\dictentry{lugar}{}{}{}
{\hyperlink{val:stuzi}{stuzi}}{}{}{}

\dictentry{lugar común}{}{}{}
{\hyperlink{val:uhenai}{u'enai}}{}{}{}

\dictentry{luna}{}{}{}
{\hyperlink{val:lunra}{lunra}}{}{}{}

\dictentry{lustrar}{}{}{}
{\hyperlink{val:spali}{spali}}{}{}{}

\dictentry{luz}{}{}{}
{\hyperlink{val:gusni}{gusni}}{}{}{}

\dictchar{M}\phantomsection\addcontentsline{toc}{section}{M}
\dictentry{m}{}{}{}
{\hyperlink{val:my}{my}}{}{}{}

\dictentry{macho}{}{}{}
{\hyperlink{val:nakni}{nakni}}{}{}{}

\dictentry{madera}{}{}{}
{\hyperlink{val:mudri}{mudri}}{}{}{}

\dictentry{madre}{}{}{}
{\hyperlink{val:mamta}{mamta}}{}{}{}

\dictentry{maduro}{}{}{}
{\hyperlink{val:makcu}{makcu}}{}{}{}

\dictentry{magenta}{}{}{}
{\hyperlink{val:nukni}{nukni}}{}{}{}

\dictentry{mágico}{}{}{}
{\hyperlink{val:makfa}{makfa}}{}{}{}

\dictentry{maíz}{}{}{}
{\hyperlink{val:zumri}{zumri}}{}{}{}

\dictentry{malayo}{}{}{}
{\hyperlink{val:meljo}{meljo}}{}{}{}

\dictentry{maldecir}{}{}{}
{\hyperlink{val:dapma}{dapma}}{}{}{}

\dictentry{malo}{}{}{}
{\hyperlink{val:xlali}{xlali}}{}{}{}

\dictentry{malvado}{}{}{}
{\hyperlink{val:palci}{palci}}{}{}{}

\dictentry{mamífero}{}{}{}
{\hyperlink{val:mabru}{mabru}}{}{}{}

\dictentry{mañana}{}{}{}
{\hyperlink{val:cerni}{cerni}}{}{}{}

\dictentry{mandar}{}{}{}
{\hyperlink{val:minde}{minde}}{}{}{}

\dictentry{mandíbula}{}{}{}
{\hyperlink{val:xedja}{xedja}}{}{}{}

\dictentry{mandioca}{}{}{}
{\hyperlink{val:mandioka}{mandioka}}{}{}{}

\dictentry{manejar}{}{}{}
{\hyperlink{val:sazri}{sazri}}{}{}{}

\dictentry{mango}{}{}{fruta}
{\hyperlink{val:mango}{mango}}{}{}{}

\dictentry{mano}{}{}{}
{\hyperlink{val:xance}{xance}}{}{}{}

\dictentry{mantequilla}{}{}{}
{\hyperlink{val:matne}{matne}}{}{}{}

\dictentry{manual}{}{}{}
{\hyperlink{val:macnu}{macnu}}{}{}{}

\dictentry{manzana}{}{}{}
{\hyperlink{val:plise}{plise}}{}{}{}

\dictentry{mapa}{}{}{}
{\hyperlink{val:cartu}{cartu}}{}{}{}

\dictentry{máquina}{}{}{}
{\hyperlink{val:minji}{minji}}{}{}{}

\dictentry{mar}{}{}{}
{\hyperlink{val:xamsi}{xamsi}}{}{}{}

\dictentry{maravillarse}{}{}{}
{\hyperlink{val:manci}{manci}}{}{}{}

\dictentry{marca}{}{}{}
{\hyperlink{val:barna}{barna}}{}{}{}

\dictentry{marca conector de grupo de términos}{}{}{}
{\hyperlink{val:pehe}{pe'e}}{}{}{}

\dictentry{marco}{}{}{}
{\hyperlink{val:greku}{greku}}{}{}{}

\dictentry{marea}{}{}{}
{\hyperlink{val:ctaru}{ctaru}}{}{}{}

\dictentry{marihuana}{}{}{}
{\hyperlink{val:marna}{marna}}{}{}{}

\dictentry{mariposa}{}{}{}
{\hyperlink{val:toldi}{toldi}}{}{}{}

\dictentry{marmota}{}{}{}
{\hyperlink{val:marmota}{marmota}}{}{}{}

\dictentry{marrón}{}{}{}
{\hyperlink{val:bunre}{bunre}}{}{}{}

\dictentry{marroquí}{}{}{}
{\hyperlink{val:morko}{morko}}{}{}{}

\dictentry{martillo}{}{}{}
{\hyperlink{val:mruli}{mruli}}{}{}{}

\dictentry{más}{}{}{}
{\hyperlink{val:suhi}{su'i}}{}{}{}

\dictentry{más allá de}{}{}{}
{\hyperlink{val:fehebaho}{fe'eba'o}}{}{}{}

\dictentry{más en (propiedad)}{}{}{}
{\hyperlink{val:temau}{temau}}{}{}{}

\dictentry{más por (cantidad)}{}{}{}
{\hyperlink{val:vemau}{vemau}}{}{}{}

\dictentry{más que}{}{}{}
{\hyperlink{val:semau}{semau}}{}{}{}

\dictentry{masa}{}{}{}
{\hyperlink{val:gunma}{gunma}}{}{}{}

\dictentry{matar}{}{}{}
{\hyperlink{val:catra}{catra}}{}{}{}

\dictentry{matemática}{}{}{}
{\hyperlink{val:cmaci}{cmaci}}{}{}{}

\dictentry{material}{}{}{}
{\hyperlink{val:marji}{marji}}{}{}{}

\dictentry{material de hamaca}{}{}{}
{\hyperlink{val:dadycka}{se dadycka}}{}{}{}

\dictentry{matríz}{}{}{}
{\hyperlink{val:johi}{jo'i}}{}{}{}

\dictentry{matríz de columnas}{}{}{}
{\hyperlink{val:sahi}{sa'i}}{}{}{}

\dictentry{matríz de filas}{}{}{}
{\hyperlink{val:piha}{pi'a}}{}{}{}

\dictentry{mayor o igual}{}{}{}
{\hyperlink{val:zmajavduhi}{zmajavdu'i}}{}{}{}

\dictentry{mayor que}{}{}{}
{\hyperlink{val:zahu}{za'u}}{}{}{}

\dictentry{mayoría}{}{}{}
{\hyperlink{val:xabmau}{xabmau}}{}{}{}

\dictentry{medial de conectivo}{}{}{}
{\hyperlink{val:gi}{gi}}{}{}{}

\dictentry{mediano}{}{}{}
{\hyperlink{val:milxe}{milxe}}{}{}{}

\dictentry{medio}{}{}{}
{\hyperlink{val:midju}{midju}}{}{}{}

\dictentry{medio de producción}{}{}{}
{\hyperlink{val:pracmu}{pracmu}}{}{}{}

\dictentry{medios de producción}{}{}{}
{\hyperlink{val:pracmu}{pracmu}}{}{}{}

\dictentry{medir}{}{}{}
{\hyperlink{val:merli}{merli}}{}{}{}

\dictentry{mega}{}{}{}
{\hyperlink{val:megdo}{megdo}}{}{}{}

\dictentry{melón}{}{}{}
{\hyperlink{val:guzme}{guzme}}{}{}{}

\dictentry{menor o igual}{}{}{}
{\hyperlink{val:mecyjavduhi}{mecyjavdu'i}}{}{}{}

\dictentry{menor que}{}{}{}
{\hyperlink{val:mehi}{me'i}}{}{}{}

\dictentry{menos}{}{}{}
{\hyperlink{val:vuhu}{vu'u}}{}{}{}

\dictentry{menos en (propiedad)}{}{}{}
{\hyperlink{val:temeha}{teme'a}}{}{}{}

\dictentry{menos por (cantidad)}{}{}{}
{\hyperlink{val:vemeha}{veme'a}}{}{}{}

\dictentry{menos que}{}{}{}
{\hyperlink{val:semeha}{seme'a}}{}{}{}

\dictentry{mensaje}{}{}{}
{\hyperlink{val:notci}{notci}}{}{}{}

\dictentry{mental}{}{}{}
{\hyperlink{val:rohe}{ro'e}}{}{}{}

\dictentry{mente}{}{}{}
{\hyperlink{val:menli}{menli}}{}{}{}

\dictentry{mera tolerancia}{}{}{}
{\hyperlink{val:ohocuhi}{o'ocu'i}}{}{}{}

\dictentry{mercado}{}{}{}
{\hyperlink{val:zarci}{zarci}}{}{}{}

\dictentry{mercurio}{}{}{}
{\hyperlink{val:margu}{margu}}{}{}{}

\dictentry{mes}{}{}{}
{\hyperlink{val:masti}{masti}}{}{}{}

\dictentry{mesa}{}{}{}
{\hyperlink{val:jubme}{jubme}}{}{}{}

\dictentry{metáfora}{}{}{}
{\hyperlink{val:tanru}{tanru}}{}{}{}

\dictentry{metal}{}{}{}
{\hyperlink{val:jinme}{jinme}}{}{}{}

\dictentry{meter}{}{}{}
{\hyperlink{val:setca}{setca}}{}{}{}

\dictentry{método}{}{}{}
{\hyperlink{val:tadji}{tadji}}{}{}{}

\dictentry{metro}{}{}{}
{\hyperlink{val:mitre}{mitre}}{}{}{}

\dictentry{mexicano}{}{}{}
{\hyperlink{val:mexno}{mexno}}{}{}{}

\dictentry{México}{}{}{}
{\hyperlink{val:mexyguhe}{mexygu'e}}{}{}{}

\dictentry{mezcla}{}{}{}
{\hyperlink{val:mixre}{mixre}}{}{}{}

\dictentry{mi}{}{}{}
{\hyperlink{val:memimoi}{memimoi}}{}{}{}

\dictentry{micro}{}{}{}
{\hyperlink{val:mikri}{mikri}}{}{}{}

\dictentry{miedo}{}{}{}
{\hyperlink{val:ii}{ii}}{}{}{}

\dictentry{miembro}{}{}{}
{\hyperlink{val:cmima}{cmima}}{}{}{}

\dictentry{mijo}{}{}{}
{\hyperlink{val:cunmi}{cunmi}}{}{}{}

\dictentry{mili}{}{}{}
{\hyperlink{val:milti}{milti}}{}{}{}

\dictentry{militar}{}{}{}
{\hyperlink{val:bilni}{bilni}}{}{}{}

\dictentry{milla}{}{}{}
{\hyperlink{val:minli}{minli}}{}{}{}

\dictentry{mineral}{}{}{}
{\hyperlink{val:kunra}{kunra}}{}{}{}

\dictentry{minimización}{}{}{}
{\hyperlink{val:bahunai}{ba'unai}}{}{}{}

\dictentry{minoría}{}{}{}
{\hyperlink{val:xabmeha}{xabme'a}}{}{}{}

\dictentry{minuto}{}{}{}
{\hyperlink{val:mentu}{mentu}}{}{}{}

\dictentry{mío}{}{}{}
{\hyperlink{val:memimoi}{memimoi}}{}{}{}

\dictentry{mirar}{}{}{}
{\hyperlink{val:catlu}{catlu}}{}{}{}

\dictentry{misceláneo}{}{}{}
{\hyperlink{val:vrici}{vrici}}{}{}{}

\dictentry{mismo}{}{}{}
{\hyperlink{val:sevzi}{sevzi}}{}{}{}

\dictentry{misterio}{}{}{}
{\hyperlink{val:ihunai}{i'unai}}{}{}{}

\dictentry{mitad}{}{}{}
{\hyperlink{val:xadba}{xadba}}{}{}{}

\dictentry{mito}{}{}{}
{\hyperlink{val:ranmi}{ranmi}}{}{}{}

\dictentry{modal ?}{}{}{}
{\hyperlink{val:cuhe}{cu'e}}{}{}{}

\dictentry{modal no específico}{}{}{}
{\hyperlink{val:dohe}{do'e}}{}{}{}

\dictentry{modelo}{}{}{}
{\hyperlink{val:morna}{morna}}{}{}{}

\dictentry{modestia}{}{}{}
{\hyperlink{val:ohacuhi}{o'acu'i}}{}{}{}

\dictentry{mojado}{}{}{}
{\hyperlink{val:cilmo}{cilmo}}{}{}{}

\dictentry{mol}{}{}{}
{\hyperlink{val:molro}{molro}}{}{}{}

\dictentry{moler}{}{}{}
{\hyperlink{val:zalvi}{zalvi}}{}{}{}

\dictentry{molestar}{}{}{}
{\hyperlink{val:fanza}{fanza}}{}{}{}

\dictentry{molibdeno}{}{}{}
{\hyperlink{val:mlibdena}{mlibdena}}{}{}{}

\dictentry{molino}{}{}{}
{\hyperlink{val:molki}{molki}}{}{}{}

\dictentry{moneda}{}{}{}
{\hyperlink{val:sicni}{sicni}}{}{}{}

\dictentry{monitor}{}{}{}
{\hyperlink{val:vidni}{vidni}}{}{}{}

\dictentry{mono}{}{}{}
{\hyperlink{val:smani}{smani}}{}{}{}

\dictentry{montaña}{}{}{}
{\hyperlink{val:cmana}{cmana}}{}{}{}

\dictentry{montón}{}{}{}
{\hyperlink{val:derxi}{derxi}}{}{}{}

\dictentry{moral}{}{}{}
{\hyperlink{val:marde}{marde}}{}{}{}

\dictentry{morder}{}{}{}
{\hyperlink{val:batci}{batci}}{}{}{}

\dictentry{morir}{}{}{}
{\hyperlink{val:mrobiho}{mrobi'o}}{}{}{}

\dictentry{morsa}{}{}{}
{\hyperlink{val:pinpedi}{pinpedi}}{}{}{}

\dictentry{mortal}{}{}{sujeto a la muerte}
{\hyperlink{val:mrodimna}{mrodimna}}{}{}{}

\dictentry{mosca}{}{}{}
{\hyperlink{val:sfani}{sfani}}{}{}{}

\dictentry{mostrador}{}{}{}
{\hyperlink{val:kajna}{kajna}}{}{}{}

\dictentry{mostrar}{}{}{}
{\hyperlink{val:jarco}{jarco}}{}{}{}

\dictentry{motivando a}{}{}{}
{\hyperlink{val:temuhi}{temu'i}}{}{}{}

\dictentry{motivo}{}{}{}
{\hyperlink{val:mukti}{mukti}}{}{}{}

\dictentry{motor}{}{}{}
{\hyperlink{val:matra}{matra}}{}{}{}

\dictentry{moverse}{}{}{}
{\hyperlink{val:muvdu}{muvdu}}{}{}{}

\dictentry{movimiento espacial}{}{}{}
{\hyperlink{val:mohi}{mo'i}}{}{}{}

\dictentry{muchas veces}{}{}{}
{\hyperlink{val:sohiroi}{so'iroi}}{}{}{}

\dictentry{mucho}{}{}{}
{\hyperlink{val:mutce}{mutce}}{}{}{}

\dictentry{mucho de}{}{}{}
{\hyperlink{val:pisohi}{piso'i}}{}{}{}

\dictentry{mucho tiempo antes}{}{}{}
{\hyperlink{val:puzu}{puzu}}{}{}{}

\dictentry{muchos}{}{}{}
{\hyperlink{val:sohi}{so'i}}{}{}{}

\dictentry{mueble}{}{}{}
{\hyperlink{val:nilce}{nilce}}{}{}{}

\dictentry{muerto}{}{}{}
{\hyperlink{val:morsi}{morsi}}{}{}{}

\dictentry{mujer}{}{}{}
{\hyperlink{val:ninmu}{ninmu}}{}{}{}

\dictentry{multiplicado por}{}{}{}
{\hyperlink{val:pihi}{pi'i}}{}{}{}

\dictentry{muro}{}{}{}
{\hyperlink{val:bitmu}{bitmu}}{}{}{}

\dictentry{músculo}{}{}{}
{\hyperlink{val:sluji}{sluji}}{}{}{}

\dictentry{museo}{}{}{}
{\hyperlink{val:muzga}{muzga}}{}{}{}

\dictentry{musgo}{}{}{}
{\hyperlink{val:clika}{clika}}{}{}{}

\dictentry{música}{}{}{}
{\hyperlink{val:zgike}{zgike}}{}{}{}

\dictentry{mutuo}{}{}{}
{\hyperlink{val:simxu}{simxu}}{}{}{}

\dictentry{muy seguro}{}{}{}
{\hyperlink{val:juhocai}{ju'ocai}}{}{}{}

\dictchar{N}\phantomsection\addcontentsline{toc}{section}{N}
\dictentry{n}{}{}{}
{\hyperlink{val:ny}{ny}}{}{}{}

\dictentry{nacer}{}{}{}
{\hyperlink{val:jbena}{jbena}}{}{}{}

\dictentry{nación}{}{}{}
{\hyperlink{val:natmi}{natmi}}{}{}{}

\dictentry{nada}{}{}{}
{\hyperlink{val:noda}{noda}}{}{}{}

\dictentry{nadar}{}{}{}
{\hyperlink{val:limna}{limna}}{}{}{}

\dictentry{nadie}{}{}{}
{\hyperlink{val:noda}{noda}}{}{}{}

\dictentry{nano}{}{}{}
{\hyperlink{val:nanvi}{nanvi}}{}{}{}

\dictentry{naranja}{}{}{}
{\hyperlink{val:narju}{narju}}{}{}{}

\dictentry{narguile}{}{}{}
{\hyperlink{val:nargile}{nargile}}{}{}{}

\dictentry{nariz}{}{}{}
{\hyperlink{val:nazbi}{nazbi}}{}{}{}

\dictentry{natural}{}{}{}
{\hyperlink{val:rarna}{rarna}}{}{}{}

\dictentry{Nb}{}{}{}
{\hyperlink{val:jinmrniobi}{jinmrniobi}}{}{}{}

\dictentry{necesariamente bajo condiciones ...}{}{}{}
{\hyperlink{val:tesau}{tesau}}{}{}{}

\dictentry{necesario}{}{}{}
{\hyperlink{val:sarcu}{sarcu}}{}{}{}

\dictentry{necesario para}{}{}{}
{\hyperlink{val:sesau}{sesau}}{}{}{}

\dictentry{necesitar}{}{}{}
{\hyperlink{val:nitcu}{nitcu}}{}{}{}

\dictentry{negación}{}{}{}
{\hyperlink{val:natfe}{natfe}}{}{}{}

\dictentry{negador de bridi}{}{}{}
{\hyperlink{val:na}{na}}{}{}{}

\dictentry{negando emoción}{}{}{}
{\hyperlink{val:rohinai}{ro'inai}}{}{}{}

\dictentry{negando físico}{}{}{}
{\hyperlink{val:rohonai}{ro'onai}}{}{}{}

\dictentry{negativo}{}{}{}
{\hyperlink{val:nonmeha}{nonme'a}}{}{}{}

\dictentry{negro}{}{}{}
{\hyperlink{val:xekri}{xekri}}{}{}{}

\dictentry{neón}{}{}{}
{\hyperlink{val:navni}{navni}}{}{}{}

\dictentry{nervio}{}{}{}
{\hyperlink{val:nirna}{nirna}}{}{}{}

\dictentry{nervioso}{}{}{}
{\hyperlink{val:xanka}{xanka}}{}{}{}

\dictentry{neutro}{}{}{}
{\hyperlink{val:nutli}{nutli}}{}{}{}

\dictentry{nido}{}{}{}
{\hyperlink{val:zdani}{zdani}}{}{}{}

\dictentry{niega palabra anterior}{}{}{}
{\hyperlink{val:nai}{nai}}{}{}{}

\dictentry{niega último bridi}{}{}{}
{\hyperlink{val:nagohi}{nago'i}}{}{}{}

\dictentry{nieve}{}{}{}
{\hyperlink{val:snime}{snime}}{}{}{}

\dictentry{niña}{}{}{}
{\hyperlink{val:nixli}{nixli}}{}{}{}

\dictentry{ningún esfuerzo especial}{}{}{}
{\hyperlink{val:ahicuhi}{a'icu'i}}{}{}{}

\dictentry{ninguno}{}{}{}
{\hyperlink{val:noda}{noda}}{}{}{}

\dictentry{niño}{}{}{}
{\hyperlink{val:verba}{verba}}{}{}{}

\dictentry{niobio}{}{}{}
{\hyperlink{val:jinmrniobi}{jinmrniobi}}{}{}{}

\dictentry{níquel}{}{}{}
{\hyperlink{val:nikle}{nikle}}{}{}{}

\dictentry{nitrógeno}{}{}{}
{\hyperlink{val:trano}{trano}}{}{}{}

\dictentry{no aprobación}{}{}{}
{\hyperlink{val:ihecuhi}{i'ecu'i}}{}{}{}

\dictentry{no escalar medio}{}{}{}
{\hyperlink{val:nohe}{no'e}}{}{}{}

\dictentry{no gracias a tí}{}{}{}
{\hyperlink{val:kihenai}{ki'enai}}{}{}{}

\dictentry{no habitualmente}{}{}{}
{\hyperlink{val:tahenai}{ta'enai}}{}{}{}

\dictentry{no hacia}{}{}{}
{\hyperlink{val:nahefaha}{na'efa'a}}{}{}{}

\dictentry{no listo para recibir}{}{}{}
{\hyperlink{val:rehinai}{re'inai}}{}{}{}

\dictentry{no más que}{}{}{}
{\hyperlink{val:semaunai}{semaunai}}{}{}{}

\dictentry{no menos que}{}{}{}
{\hyperlink{val:semehanai}{seme'anai}}{}{}{}

\dictentry{no metalinguistico}{}{}{}
{\hyperlink{val:nahi}{na'i}}{}{}{}

\dictentry{no muy sorprendido}{}{}{}
{\hyperlink{val:uecuhi}{uecu'i}}{}{}{}

\dictentry{no recibido}{}{}{}
{\hyperlink{val:jehenai}{je'enai}}{}{}{}

\dictentry{no soy yo}{}{}{}
{\hyperlink{val:mihenai}{mi'enai}}{}{}{}

\dictentry{no todo}{}{}{}
{\hyperlink{val:mehi}{me'i}}{}{}{}

\dictentry{noble}{}{}{}
{\hyperlink{val:nobli}{nobli}}{}{}{}

\dictentry{noche}{}{}{}
{\hyperlink{val:nicte}{nicte}}{}{}{}

\dictentry{nombrado por}{}{}{}
{\hyperlink{val:temehe}{teme'e}}{}{}{}

\dictentry{nombre}{}{}{}
{\hyperlink{val:cmene}{cmene}}{}{}{}

\dictentry{norma}{}{}{valor absoluto}
{\hyperlink{val:nacnilbra}{nacnilbra}}{}{}{}

\dictentry{normal}{}{}{}
{\hyperlink{val:cnano}{cnano}}{}{}{}

\dictentry{norte}{}{}{}
{\hyperlink{val:berti}{berti}}{}{}{}

\dictentry{norteamericano}{}{}{}
{\hyperlink{val:bemro}{bemro}}{}{}{}

\dictentry{nosotros sin ustedes}{}{}{}
{\hyperlink{val:miha}{mi'a}}{}{}{}

\dictentry{notación exponencial}{}{}{}
{\hyperlink{val:gei}{gei}}{}{}{}

\dictentry{noticia}{}{}{}
{\hyperlink{val:nuzba}{nuzba}}{}{}{}

\dictentry{nube}{}{}{}
{\hyperlink{val:dilnu}{dilnu}}{}{}{}

\dictentry{nudo}{}{}{}
{\hyperlink{val:jgena}{jgena}}{}{}{}

\dictentry{nueve}{}{}{}
{\hyperlink{val:so}{so}}{}{}{}

\dictentry{nuevo}{}{}{}
{\hyperlink{val:cnino}{cnino}}{}{}{}

\dictentry{nuez}{}{}{}
{\hyperlink{val:narge}{narge}}{}{}{}

\dictentry{número}{}{}{}
{\hyperlink{val:namcu}{namcu}}{}{}{}

\dictentry{número ?}{}{}{}
{\hyperlink{val:xo}{xo}}{}{}{}

\dictentry{número negativo}{}{}{}
{\hyperlink{val:nihu}{ni'u}}{}{}{}

\dictentry{número positivo}{}{}{}
{\hyperlink{val:mahu}{ma'u}}{}{}{}

\dictentry{nunca}{}{}{}
{\hyperlink{val:noroi}{noroi}}{}{}{}

\dictchar{O}\phantomsection\addcontentsline{toc}{section}{O}
\dictentry{o}{}{}{}
{\hyperlink{val:obu}{obu}}{}{}{}

\dictentry{o (sumti)}{}{}{}
{\hyperlink{val:a}{a}}{}{}{}

\dictentry{o exclusivo (sumti)}{}{}{}
{\hyperlink{val:onai}{onai}}{}{}{}

\dictentry{o exclusivo preposicionado}{}{}{}
{\hyperlink{val:gonai}{gonai}}{}{}{}

\dictentry{o preposicionado}{}{}{}
{\hyperlink{val:ga}{ga}}{}{}{}

\dictentry{obedecer}{}{}{}
{\hyperlink{val:tinbe}{tinbe}}{}{}{}

\dictentry{objeto}{}{}{}
{\hyperlink{val:dacti}{dacti}}{}{}{}

\dictentry{objeto geométrico}{}{}{}
{\hyperlink{val:caltaicmaci}{se caltaicmaci}}{}{}{}

\dictentry{objeto material ...}{}{}{}
{\hyperlink{val:mahe}{ma'e}}{}{}{}

\dictentry{obligación}{}{}{}
{\hyperlink{val:ei}{ei}}{}{}{}

\dictentry{obligado}{}{}{}
{\hyperlink{val:bilga}{bilga}}{}{}{}

\dictentry{obra de arte}{}{}{}
{\hyperlink{val:lardai}{lardai}}{}{}{}

\dictentry{observación}{}{}{}
{\hyperlink{val:zaha}{za'a}}{}{}{}

\dictentry{observado bajo condiciones ...}{}{}{}
{\hyperlink{val:vegaha}{vega'a}}{}{}{}

\dictentry{observado por ...}{}{}{}
{\hyperlink{val:gaha}{ga'a}}{}{}{}

\dictentry{observado por medio de}{}{}{}
{\hyperlink{val:tegaha}{tega'a}}{}{}{}

\dictentry{observando}{}{}{}
{\hyperlink{val:segaha}{sega'a}}{}{}{}

\dictentry{observar}{}{}{}
{\hyperlink{val:zgana}{zgana}}{}{}{}

\dictentry{ocasionalmente}{}{}{}
{\hyperlink{val:ruhinai}{ru'inai}}{}{}{}

\dictentry{ocho}{}{}{}
{\hyperlink{val:bi}{bi}}{}{}{}

\dictentry{ocupación}{}{}{}
{\hyperlink{val:jibri}{jibri}}{}{}{}

\dictentry{ocurrir}{}{}{}
{\hyperlink{val:fasnu}{fasnu}}{}{}{}

\dictentry{odiar}{}{}{}
{\hyperlink{val:xebni}{xebni}}{}{}{}

\dictentry{odio}{}{}{}
{\hyperlink{val:iunai}{iunai}}{}{}{}

\dictentry{oeste}{}{}{}
{\hyperlink{val:stici}{stici}}{}{}{}

\dictentry{oficina}{}{}{}
{\hyperlink{val:briju}{briju}}{}{}{}

\dictentry{ofrecer}{}{}{}
{\hyperlink{val:friti}{friti}}{}{}{}

\dictentry{¡Oh tú!}{}{}{}
{\hyperlink{val:doidohu}{doido'u}}{}{}{}

\dictentry{oír}{}{}{}
{\hyperlink{val:tirna}{tirna}}{}{}{}

\dictentry{ojo}{}{}{}
{\hyperlink{val:kanla}{kanla}}{}{}{}

\dictentry{oler}{}{}{}
{\hyperlink{val:sumne}{sumne}}{}{}{}

\dictentry{olla}{}{}{}
{\hyperlink{val:patxu}{patxu}}{}{}{}

\dictentry{olmo}{}{}{}
{\hyperlink{val:ulmu}{ulmu}}{}{}{}

\dictentry{olor}{}{}{}
{\hyperlink{val:panci}{panci}}{}{}{}

\dictentry{omitiendo ejemplos}{}{}{}
{\hyperlink{val:muhacuhi}{mu'acu'i}}{}{}{}

\dictentry{onda}{}{}{}
{\hyperlink{val:boxna}{boxna}}{}{}{}

\dictentry{operador a selbri}{}{}{}
{\hyperlink{val:nuha}{nu'a}}{}{}{}

\dictentry{operador de alta prioridad}{}{}{}
{\hyperlink{val:bihe}{bi'e}}{}{}{}

\dictentry{operador no específico}{}{}{}
{\hyperlink{val:fuhu}{fu'u}}{}{}{}

\dictentry{operador nulo}{}{}{}
{\hyperlink{val:geha}{ge'a}}{}{}{}

\dictentry{operador preposicionado}{}{}{}
{\hyperlink{val:peho}{pe'o}}{}{}{}

\dictentry{operando a operador}{}{}{}
{\hyperlink{val:maho}{ma'o}}{}{}{}

\dictentry{operando nulo}{}{}{}
{\hyperlink{val:tuho}{tu'o}}{}{}{}

\dictentry{opinar}{}{}{}
{\hyperlink{val:jinvi}{jinvi}}{}{}{}

\dictentry{opinión}{}{}{}
{\hyperlink{val:pehi}{pe'i}}{}{}{}

\dictentry{oponerse}{}{}{}
{\hyperlink{val:fapro}{fapro}}{}{}{}

\dictentry{opuesto polar}{}{}{}
{\hyperlink{val:tohe}{to'e}}{}{}{}

\dictentry{oración}{}{}{}
{\hyperlink{val:seduhu}{sedu'u}}{}{}{}

\dictentry{oración conector ?}{}{}{}
{\hyperlink{val:ijehi}{ije'i}}{}{}{}

\dictentry{oración o}{}{}{}
{\hyperlink{val:ija}{ija}}{}{}{}

\dictentry{oración o exclusivo}{}{}{}
{\hyperlink{val:ijonai}{ijonai}}{}{}{}

\dictentry{oración pero no}{}{}{}
{\hyperlink{val:ijenai}{ijenai}}{}{}{}

\dictentry{oración sea o no}{}{}{}
{\hyperlink{val:iju}{iju}}{}{}{}

\dictentry{oración si}{}{}{}
{\hyperlink{val:ijanai}{ijanai}}{}{}{}

\dictentry{oración sii}{}{}{}
{\hyperlink{val:ijo}{ijo}}{}{}{}

\dictentry{oración sólo si}{}{}{}
{\hyperlink{val:inaja}{inaja}}{}{}{}

\dictentry{oración y}{}{}{}
{\hyperlink{val:ije}{ije}}{}{}{}

\dictentry{orangután}{}{}{}
{\hyperlink{val:rangutano}{rangutano}}{}{}{}

\dictentry{orbitando}{}{}{}
{\hyperlink{val:mohiruhu}{mo'iru'u}}{}{}{}

\dictentry{ordenado en secuencia ...}{}{}{}
{\hyperlink{val:telihe}{teli'e}}{}{}{}

\dictentry{ordenando ítems ...}{}{}{}
{\hyperlink{val:tepohi}{tepo'i}}{}{}{}

\dictentry{ordinal de oración}{}{}{}
{\hyperlink{val:mai}{mai}}{}{}{}

\dictentry{ordinal de sección}{}{}{}
{\hyperlink{val:moho}{mo'o}}{}{}{}

\dictentry{oreja}{}{}{}
{\hyperlink{val:kerlo}{kerlo}}{}{}{}

\dictentry{organizar}{}{}{}
{\hyperlink{val:ganzu}{ganzu}}{}{}{}

\dictentry{órgano}{}{}{}
{\hyperlink{val:rango}{rango}}{}{}{}

\dictentry{orgullo}{}{}{}
{\hyperlink{val:oha}{o'a}}{}{}{}

\dictentry{orgulloso}{}{}{}
{\hyperlink{val:jgira}{jgira}}{}{}{}

\dictentry{orientado hacia otro}{}{}{}
{\hyperlink{val:sehinai}{se'inai}}{}{}{}

\dictentry{origen}{}{}{}
{\hyperlink{val:krasi}{krasi}}{}{}{}

\dictentry{orilla}{}{}{}
{\hyperlink{val:korbi}{korbi}}{}{}{}

\dictentry{orina}{}{}{}
{\hyperlink{val:pinca}{pinca}}{}{}{}

\dictentry{oro}{}{}{}
{\hyperlink{val:solji}{solji}}{}{}{}

\dictentry{osar}{}{}{}
{\hyperlink{val:darsi}{darsi}}{}{}{}

\dictentry{oscilar}{}{}{}
{\hyperlink{val:slilu}{slilu}}{}{}{}

\dictentry{oscuramente}{}{}{}
{\hyperlink{val:lihanai}{li'anai}}{}{}{}

\dictentry{oscuro}{}{}{}
{\hyperlink{val:manku}{manku}}{}{}{}

\dictentry{oso}{}{}{}
{\hyperlink{val:cribe}{cribe}}{}{}{}

\dictentry{otoño}{}{}{}
{\hyperlink{val:critu}{critu}}{}{}{}

\dictentry{otro}{}{}{}
{\hyperlink{val:drata}{drata}}{}{}{}

\dictentry{oveja}{}{}{}
{\hyperlink{val:lanme}{lanme}}{}{}{}

\dictentry{oxígeno}{}{}{}
{\hyperlink{val:kijno}{kijno}}{}{}{}

\dictchar{P}\phantomsection\addcontentsline{toc}{section}{P}
\dictentry{p}{}{}{}
{\hyperlink{val:py}{py}}{}{}{}

\dictentry{paciencia}{}{}{}
{\hyperlink{val:oho}{o'o}}{}{}{}

\dictentry{padre}{}{}{}
{\hyperlink{val:patfu}{patfu}}{}{}{}

\dictentry{pagar}{}{}{}
{\hyperlink{val:pleji}{pleji}}{}{}{}

\dictentry{página}{}{}{}
{\hyperlink{val:papri}{papri}}{}{}{}

\dictentry{país}{}{}{}
{\hyperlink{val:gugde}{gugde}}{}{}{}

\dictentry{pájaro}{}{}{}
{\hyperlink{val:cipni}{cipni}}{}{}{}

\dictentry{pala}{}{}{}
{\hyperlink{val:canpa}{canpa}}{}{}{}

\dictentry{palabra}{}{}{}
{\hyperlink{val:valsi}{valsi}}{}{}{}

\dictentry{palabra a lerfu}{}{}{}
{\hyperlink{val:bu}{bu}}{}{}{}

\dictentry{palabra efímera sigue}{}{}{}
{\hyperlink{val:zahe}{za'e}}{}{}{}

\dictentry{palanca}{}{}{}
{\hyperlink{val:vraga}{vraga}}{}{}{}

\dictentry{palestino}{}{}{}
{\hyperlink{val:filso}{filso}}{}{}{}

\dictentry{pálido}{}{}{}
{\hyperlink{val:kandi}{kandi}}{}{}{}

\dictentry{palo}{}{}{}
{\hyperlink{val:grana}{grana}}{}{}{}

\dictentry{pan}{}{}{}
{\hyperlink{val:nanba}{nanba}}{}{}{}

\dictentry{pantalón}{}{}{}
{\hyperlink{val:palku}{palku}}{}{}{}

\dictentry{papa}{}{}{}
{\hyperlink{val:patlu}{patlu}}{}{}{}

\dictentry{papel}{}{}{}
{\hyperlink{val:pelji}{pelji}}{}{}{}

\dictentry{paquete}{}{}{}
{\hyperlink{val:bakfu}{bakfu}}{}{}{}

\dictentry{paquistaní}{}{}{}
{\hyperlink{val:kisto}{kisto}}{}{}{}

\dictentry{parada}{}{}{}
{\hyperlink{val:tcana}{tcana}}{}{}{}

\dictentry{paraguas}{}{}{}
{\hyperlink{val:santa}{santa}}{}{}{}

\dictentry{paralelamente con}{}{}{}
{\hyperlink{val:paha}{pa'a}}{}{}{}

\dictentry{paralelo}{}{}{}
{\hyperlink{val:panra}{panra}}{}{}{}

\dictentry{paralelo a}{}{}{}
{\hyperlink{val:sepaha}{sepa'a}}{}{}{}

\dictentry{pararse}{}{}{}
{\hyperlink{val:sisti}{sisti}}{}{}{}

\dictentry{parecer}{}{}{}
{\hyperlink{val:simlu}{simlu}}{}{}{}

\dictentry{parque}{}{}{}
{\hyperlink{val:panka}{panka}}{}{}{}

\dictentry{párrafo}{}{}{}
{\hyperlink{val:jufmei}{jufmei}}{}{}{}

\dictentry{parsimoniosamente}{}{}{}
{\hyperlink{val:dohanai}{do'anai}}{}{}{}

\dictentry{parte}{}{}{}
{\hyperlink{val:pagbu}{pagbu}}{}{}{}

\dictentry{partícula}{}{}{}
{\hyperlink{val:cmavo}{cmavo}}{}{}{}

\dictentry{particularización}{}{}{}
{\hyperlink{val:suhanai}{su'anai}}{}{}{}

\dictentry{partir}{}{}{}
{\hyperlink{val:cliva}{cliva}}{}{}{}

\dictentry{pasado}{}{}{}
{\hyperlink{val:purci}{purci}}{}{}{}

\dictentry{pasando por}{}{}{}
{\hyperlink{val:mohizoha}{mo'izo'a}}{}{}{}

\dictentry{pasando por etapas ...}{}{}{}
{\hyperlink{val:vepuhe}{vepu'e}}{}{}{}

\dictentry{pasaporte}{}{}{}
{\hyperlink{val:jaspu}{jaspu}}{}{}{}

\dictentry{pasivo}{}{}{}
{\hyperlink{val:lehocuhi}{le'ocu'i}}{}{}{}

\dictentry{pasta}{}{}{}
{\hyperlink{val:pesxu}{pesxu}}{}{}{}

\dictentry{patear}{}{}{}
{\hyperlink{val:tikpa}{tikpa}}{}{}{}

\dictentry{pato}{}{}{}
{\hyperlink{val:datka}{datka}}{}{}{}

\dictentry{pausativo}{}{}{}
{\hyperlink{val:deha}{de'a}}{}{}{}

\dictentry{pavo}{}{}{}
{\hyperlink{val:xruki}{xruki}}{}{}{}

\dictentry{paz}{}{}{}
{\hyperlink{val:panpi}{panpi}}{}{}{}

\dictentry{pecado}{}{}{}
{\hyperlink{val:vuhenai}{vu'enai}}{}{}{}

\dictentry{pecho}{}{}{}
{\hyperlink{val:tatru}{tatru}}{}{}{}

\dictentry{pedazo}{}{}{}
{\hyperlink{val:spisa}{spisa}}{}{}{}

\dictentry{pedestal}{}{}{}
{\hyperlink{val:zbepi}{zbepi}}{}{}{}

\dictentry{pedido}{}{}{}
{\hyperlink{val:eho}{e'o}}{}{}{}

\dictentry{pedido negativo}{}{}{}
{\hyperlink{val:ehonai}{e'onai}}{}{}{}

\dictentry{pedido para enviar}{}{}{}
{\hyperlink{val:behe}{be'e}}{}{}{}

\dictentry{pedir}{}{}{}
{\hyperlink{val:cpedu}{cpedu}}{}{}{}

\dictentry{pegamento de lujvo}{}{}{}
{\hyperlink{val:zei}{zei}}{}{}{}

\dictentry{peine}{}{}{}
{\hyperlink{val:komcu}{komcu}}{}{}{}

\dictentry{pelear}{}{}{}
{\hyperlink{val:damba}{damba}}{}{}{}

\dictentry{peligroso}{}{}{}
{\hyperlink{val:ckape}{ckape}}{}{}{}

\dictentry{pelo}{}{}{}
{\hyperlink{val:kerfa}{kerfa}}{}{}{}

\dictentry{pene}{}{}{}
{\hyperlink{val:pinji}{pinji}}{}{}{}

\dictentry{pensar}{}{}{}
{\hyperlink{val:pensi}{pensi}}{}{}{}

\dictentry{pequeño}{}{}{}
{\hyperlink{val:cmalu}{cmalu}}{}{}{}

\dictentry{pera}{}{}{}
{\hyperlink{val:perli}{perli}}{}{}{}

\dictentry{perder}{}{}{}
{\hyperlink{val:cirko}{cirko}}{}{}{}

\dictentry{pérdida}{}{}{perjuicio}
{\hyperlink{val:tolprali}{tolprali}}{}{}{}

\dictentry{pérdida}{}{}{emocional}
{\hyperlink{val:uhanai}{u'anai}}{}{}{}

\dictentry{perdonar}{}{}{}
{\hyperlink{val:fraxu}{fraxu}}{}{}{}

\dictentry{perezoso}{}{}{}
{\hyperlink{val:lazni}{lazni}}{}{}{}

\dictentry{perfectivo}{}{}{}
{\hyperlink{val:baho}{ba'o}}{}{}{}

\dictentry{perfecto}{}{}{}
{\hyperlink{val:prane}{prane}}{}{}{}

\dictentry{periódico}{}{}{}
{\hyperlink{val:karni}{karni}}{}{}{}

\dictentry{perjuicio}{}{}{}
{\hyperlink{val:tolprali}{tolprali}}{}{}{}

\dictentry{permanente}{}{}{}
{\hyperlink{val:vitno}{vitno}}{}{}{}

\dictentry{permiso}{}{}{}
{\hyperlink{val:eha}{e'a}}{}{}{}

\dictentry{permitir}{}{}{}
{\hyperlink{val:curmi}{curmi}}{}{}{}

\dictentry{perro}{}{}{}
{\hyperlink{val:gerku}{gerku}}{}{}{}

\dictentry{persona}{}{}{}
{\hyperlink{val:prenu}{prenu}}{}{}{}

\dictentry{pesado}{}{}{}
{\hyperlink{val:tilju}{tilju}}{}{}{}

\dictentry{peso}{}{}{}
{\hyperlink{val:rupnu}{rupnu}}{}{}{}

\dictentry{peta}{}{}{}
{\hyperlink{val:petso}{petso}}{}{}{}

\dictentry{petróleo}{}{}{}
{\hyperlink{val:ctile}{ctile}}{}{}{}

\dictentry{pez}{}{}{}
{\hyperlink{val:finpe}{finpe}}{}{}{}

\dictentry{pezuña}{}{}{}
{\hyperlink{val:sufti}{sufti}}{}{}{}

\dictentry{pi}{}{}{}
{\hyperlink{val:pai}{pai}}{}{}{}

\dictentry{piano}{}{}{}
{\hyperlink{val:pipno}{pipno}}{}{}{}

\dictentry{picante}{}{}{}
{\hyperlink{val:cpina}{cpina}}{}{}{}

\dictentry{pico}{}{}{}
{\hyperlink{val:picti}{picti}}{}{}{}

\dictentry{pie}{}{}{}
{\hyperlink{val:jamfu}{jamfu}}{}{}{}

\dictentry{piel}{}{}{}
{\hyperlink{val:skapi}{skapi}}{}{}{}

\dictentry{pierna}{}{}{}
{\hyperlink{val:tuple}{tuple}}{}{}{}

\dictentry{pinchar}{}{}{}
{\hyperlink{val:tunta}{tunta}}{}{}{}

\dictentry{pintura}{}{}{}
{\hyperlink{val:cinta}{cinta}}{}{}{}

\dictentry{pinza}{}{}{}
{\hyperlink{val:cinza}{cinza}}{}{}{}

\dictentry{piso}{}{}{}
{\hyperlink{val:loldi}{loldi}}{}{}{}

\dictentry{placer}{}{}{}
{\hyperlink{val:oinai}{oinai}}{}{}{}

\dictentry{planear}{}{}{}
{\hyperlink{val:platu}{platu}}{}{}{}

\dictentry{planeta}{}{}{}
{\hyperlink{val:plini}{plini}}{}{}{}

\dictentry{plano}{}{}{}
{\hyperlink{val:plita}{plita}}{}{}{}

\dictentry{planta}{}{}{}
{\hyperlink{val:spati}{spati}}{}{}{}

\dictentry{plástico}{}{}{}
{\hyperlink{val:slasi}{slasi}}{}{}{}

\dictentry{plata}{}{}{}
{\hyperlink{val:rijno}{rijno}}{}{}{}

\dictentry{plataforma}{}{}{}
{\hyperlink{val:tsina}{tsina}}{}{}{}

\dictentry{plato}{}{}{}
{\hyperlink{val:palta}{palta}}{}{}{}

\dictentry{plomo}{}{}{}
{\hyperlink{val:cnisa}{cnisa}}{}{}{}

\dictentry{pluma}{}{}{}
{\hyperlink{val:pimlu}{pimlu}}{}{}{}

\dictentry{pobre}{}{}{}
{\hyperlink{val:pindi}{pindi}}{}{}{}

\dictentry{poco profundo}{}{}{}
{\hyperlink{val:caxno}{caxno}}{}{}{}

\dictentry{poco tiempo antes}{}{}{}
{\hyperlink{val:puzi}{puzi}}{}{}{}

\dictentry{pocos}{}{}{}
{\hyperlink{val:sohu}{so'u}}{}{}{}

\dictentry{poder}{}{}{}
{\hyperlink{val:vlipa}{vlipa}}{}{}{}

\dictentry{poema}{}{}{}
{\hyperlink{val:pemci}{pemci}}{}{}{}

\dictentry{polaco inverso}{}{}{}
{\hyperlink{val:fuha}{fu'a}}{}{}{}

\dictentry{polea}{}{}{}
{\hyperlink{val:pulni}{pulni}}{}{}{}

\dictentry{policía}{}{}{}
{\hyperlink{val:pulji}{pulji}}{}{}{}

\dictentry{polinesio}{}{}{}
{\hyperlink{val:polno}{polno}}{}{}{}

\dictentry{política}{}{}{accion a realizarse}
{\hyperlink{val:plajva}{plajva}}{}{}{}

\dictentry{político}{}{}{}
{\hyperlink{val:jecraha}{jecra'a}}{}{}{}

\dictentry{polvo}{}{}{}
{\hyperlink{val:purmo}{purmo}}{}{}{}

\dictentry{poner}{}{}{}
{\hyperlink{val:punji}{punji}}{}{}{}

\dictentry{por (agente)}{}{}{}
{\hyperlink{val:gau}{gau}}{}{}{}

\dictentry{por (motivo)}{}{}{}
{\hyperlink{val:muhi}{mu'i}}{}{}{}

\dictentry{por autoridad de}{}{}{}
{\hyperlink{val:cahi}{ca'i}}{}{}{}

\dictentry{por calendario ...}{}{}{}
{\hyperlink{val:vedehi}{vede'i}}{}{}{}

\dictentry{por camino ...}{}{}{}
{\hyperlink{val:vekaha}{veka'a}}{}{}{}

\dictentry{por ejemplo ...}{}{}{}
{\hyperlink{val:muhu}{mu'u}}{}{}{}

\dictentry{por favor}{}{}{}
{\hyperlink{val:pehu}{pe'u}}{}{}{}

\dictentry{por lo menos}{}{}{}
{\hyperlink{val:suho}{su'o}}{}{}{}

\dictentry{por lo tanto (causal)}{}{}{}
{\hyperlink{val:seriha}{seri'a}}{}{}{}

\dictentry{por lo tanto (lógica)}{}{}{}
{\hyperlink{val:senihi}{seni'i}}{}{}{}

\dictentry{por lo tanto (motivo)}{}{}{}
{\hyperlink{val:semuhi}{semu'i}}{}{}{}

\dictentry{por lo tanto (razón)}{}{}{}
{\hyperlink{val:sekihu}{seki'u}}{}{}{}

\dictentry{por medio de transporte}{}{}{}
{\hyperlink{val:xekaha}{xeka'a}}{}{}{}

\dictentry{por método ...}{}{}{}
{\hyperlink{val:tahi}{ta'i}}{}{}{}

\dictentry{por otro lado}{}{}{}
{\hyperlink{val:zuhunai}{zu'unai}}{}{}{}

\dictentry{por proceso ...}{}{}{}
{\hyperlink{val:puhe}{pu'e}}{}{}{}

\dictentry{¿por qué causa?}{}{}{}
{\hyperlink{val:riha ma}{ri'a ma}}{}{}{}

\dictentry{¿por qué motivo?}{}{}{}
{\hyperlink{val:muhi ma}{mu'i ma}}{}{}{}

\dictentry{¿por qué razón?}{}{}{}
{\hyperlink{val:kihu ma}{ki'u ma}}{}{}{}

\dictentry{por regla ...}{}{}{}
{\hyperlink{val:jahi}{ja'i}}{}{}{}

\dictentry{por regla en}{}{}{}
{\hyperlink{val:tejahi}{teja'i}}{}{}{}

\dictentry{por regla prescribiendo}{}{}{}
{\hyperlink{val:sejahi}{seja'i}}{}{}{}

\dictentry{por un lado}{}{}{}
{\hyperlink{val:zuhu}{zu'u}}{}{}{}

\dictentry{por un momento en el pasado}{}{}{}
{\hyperlink{val:puzehi}{puze'i}}{}{}{}

\dictentry{por un periodo en el pasado}{}{}{}
{\hyperlink{val:puzeha}{puze'a}}{}{}{}

\dictentry{por un tiempo antes}{}{}{}
{\hyperlink{val:zehapu}{ze'apu}}{}{}{}

\dictentry{por un tiempo después}{}{}{}
{\hyperlink{val:zehaba}{ze'aba}}{}{}{}

\dictentry{por un tiempo durante}{}{}{}
{\hyperlink{val:zehaca}{ze'aca}}{}{}{}

\dictentry{por una era en el pasado}{}{}{}
{\hyperlink{val:puzehu}{puze'u}}{}{}{}

\dictentry{porcentaje}{}{}{}
{\hyperlink{val:cehi}{ce'i}}{}{}{}

\dictentry{porque (lógica)}{}{}{}
{\hyperlink{val:nihi}{ni'i}}{}{}{}

\dictentry{porque (razón)}{}{}{}
{\hyperlink{val:kihu}{ki'u}}{}{}{}

\dictentry{portugués}{}{}{}
{\hyperlink{val:porto}{porto}}{}{}{}

\dictentry{posible}{}{}{}
{\hyperlink{val:cumki}{cumki}}{}{}{}

\dictentry{posicionado}{}{}{}
{\hyperlink{val:momlai}{momlai}}{}{}{}

\dictentry{positivo}{}{}{}
{\hyperlink{val:nonmau}{nonmau}}{}{}{}

\dictentry{postulado}{}{}{}
{\hyperlink{val:ruha}{ru'a}}{}{}{}

\dictentry{potasio}{}{}{}
{\hyperlink{val:sodnrkali}{sodnrkali}}{}{}{}

\dictentry{potencia}{}{}{}
{\hyperlink{val:teha}{te'a}}{}{}{}

\dictentry{potencial}{}{}{}
{\hyperlink{val:kahe}{ka'e}}{}{}{}

\dictentry{potencial actualizado}{}{}{}
{\hyperlink{val:puhi}{pu'i}}{}{}{}

\dictentry{potencial no actualizado}{}{}{}
{\hyperlink{val:nuho}{nu'o}}{}{}{}

\dictentry{pozo}{}{}{}
{\hyperlink{val:jinto}{jinto}}{}{}{}

\dictentry{pre tanru conector ?}{}{}{}
{\hyperlink{val:guhi}{gu'i}}{}{}{}

\dictentry{pre tanru o}{}{}{}
{\hyperlink{val:guha}{gu'a}}{}{}{}

\dictentry{pre tanru o exclusivo}{}{}{}
{\hyperlink{val:guhonai}{gu'onai}}{}{}{}

\dictentry{pre tanru sea o no}{}{}{}
{\hyperlink{val:guhu}{gu'u}}{}{}{}

\dictentry{pre tanru sii}{}{}{}
{\hyperlink{val:guho}{gu'o}}{}{}{}

\dictentry{pre tanru sólo si}{}{}{}
{\hyperlink{val:guhanai}{gu'anai}}{}{}{}

\dictentry{pre tanru y}{}{}{}
{\hyperlink{val:guhe}{gu'e}}{}{}{}

\dictentry{precaución}{}{}{}
{\hyperlink{val:ohi}{o'i}}{}{}{}

\dictentry{precedencia operador}{}{}{}
{\hyperlink{val:tiho}{ti'o}}{}{}{}

\dictentry{preceder}{}{}{}
{\hyperlink{val:lidne}{lidne}}{}{}{}

\dictentry{precedido por}{}{}{}
{\hyperlink{val:lihe}{li'e}}{}{}{}

\dictentry{precediendo a}{}{}{}
{\hyperlink{val:selihe}{seli'e}}{}{}{}

\dictentry{precio}{}{}{}
{\hyperlink{val:jdima}{jdima}}{}{}{}

\dictentry{precisión}{}{}{}
{\hyperlink{val:sahe}{sa'e}}{}{}{}

\dictentry{predicado}{}{}{}
{\hyperlink{val:bridi}{bridi}}{}{}{}

\dictentry{predicado var 1}{}{}{}
{\hyperlink{val:broda}{broda}}{}{}{}

\dictentry{predicado var 2}{}{}{}
{\hyperlink{val:brode}{brode}}{}{}{}

\dictentry{predicado var 3}{}{}{}
{\hyperlink{val:brodi}{brodi}}{}{}{}

\dictentry{predicado var 4}{}{}{}
{\hyperlink{val:brodo}{brodo}}{}{}{}

\dictentry{predicado var 5}{}{}{}
{\hyperlink{val:brodu}{brodu}}{}{}{}

\dictentry{pregunta}{}{}{}
{\hyperlink{val:pau}{pau}}{}{}{}

\dictentry{pregunta indirecta}{}{}{}
{\hyperlink{val:kau}{kau}}{}{}{}

\dictentry{pregunta retórica}{}{}{}
{\hyperlink{val:paunai}{paunai}}{}{}{}

\dictentry{prejuicio}{}{}{}
{\hyperlink{val:pahenai}{pa'enai}}{}{}{}

\dictentry{premiar}{}{}{}
{\hyperlink{val:cnemu}{cnemu}}{}{}{}

\dictentry{prenda}{}{}{}
{\hyperlink{val:taxfu}{taxfu}}{}{}{}

\dictentry{presencia}{}{}{}
{\hyperlink{val:behucuhi}{be'ucu'i}}{}{}{}

\dictentry{presionar}{}{}{}
{\hyperlink{val:danre}{danre}}{}{}{}

\dictentry{primavera}{}{}{}
{\hyperlink{val:vensa}{vensa}}{}{}{}

\dictentry{primer bridi externo}{}{}{}
{\hyperlink{val:noha}{no'a}}{}{}{}

\dictentry{primo}{}{}{}
{\hyperlink{val:tamne}{tamne}}{}{}{}

\dictentry{principal}{}{}{}
{\hyperlink{val:ralju}{ralju}}{}{}{}

\dictentry{principalmente}{}{}{}
{\hyperlink{val:rahu}{ra'u}}{}{}{}

\dictentry{prisionero}{}{}{}
{\hyperlink{val:pinfu}{pinfu}}{}{}{}

\dictentry{privacidad}{}{}{}
{\hyperlink{val:ihinai}{i'inai}}{}{}{}

\dictentry{privado}{}{}{}
{\hyperlink{val:sivni}{sivni}}{}{}{}

\dictentry{pro-sumti relativo}{}{}{}
{\hyperlink{val:keha}{ke'a}}{}{}{}

\dictentry{probable}{}{}{}
{\hyperlink{val:lakne}{lakne}}{}{}{}

\dictentry{probablemente}{}{}{}
{\hyperlink{val:laha}{la'a}}{}{}{}

\dictentry{problema}{}{}{}
{\hyperlink{val:nabmi}{nabmi}}{}{}{}

\dictentry{procedo}{}{}{}
{\hyperlink{val:viho}{vi'o}}{}{}{}

\dictentry{procesando}{}{}{}
{\hyperlink{val:sepuhe}{sepu'e}}{}{}{}

\dictentry{procesando en}{}{}{}
{\hyperlink{val:tepuhe}{tepu'e}}{}{}{}

\dictentry{proceso}{}{}{}
{\hyperlink{val:puhu}{pu'u}}{}{}{}

\dictentry{procrear}{}{}{}
{\hyperlink{val:rorci}{rorci}}{}{}{}

\dictentry{producir}{}{}{}
{\hyperlink{val:cupra}{cupra}}{}{}{}

\dictentry{producto}{}{}{}
{\hyperlink{val:pilji}{pilji}}{}{}{}

\dictentry{producto cruzado}{}{}{}
{\hyperlink{val:pihu}{pi'u}}{}{}{}

\dictentry{prohibición}{}{}{}
{\hyperlink{val:ehanai}{e'anai}}{}{}{}

\dictentry{prohibir}{}{}{}
{\hyperlink{val:tolcru}{tolcru}}{}{}{}

\dictentry{promesa}{}{}{}
{\hyperlink{val:nuhe}{nu'e}}{}{}{}

\dictentry{prometer}{}{}{}
{\hyperlink{val:nupre}{nupre}}{}{}{}

\dictentry{pronunciar}{}{}{}
{\hyperlink{val:bacru}{bacru}}{}{}{}

\dictentry{propiedad}{}{}{}
{\hyperlink{val:ka}{ka}}{}{}{}

\dictentry{propiedad contrastante}{}{}{}
{\hyperlink{val:tepaha}{tepa'a}}{}{}{}

\dictentry{proporción}{}{}{}
{\hyperlink{val:parbi}{parbi}}{}{}{}

\dictentry{prosa}{}{}{}
{\hyperlink{val:prosa}{prosa}}{}{}{}

\dictentry{proteína}{}{}{}
{\hyperlink{val:lanbi}{lanbi}}{}{}{}

\dictentry{protestar}{}{}{}
{\hyperlink{val:pante}{pante}}{}{}{}

\dictentry{proveer}{}{}{}
{\hyperlink{val:sabji}{sabji}}{}{}{}

\dictentry{provincia imperial}{}{}{}
{\hyperlink{val:sorgugjeha}{se sorgugje'a}}{}{}{}

\dictentry{proyectarse}{}{}{}
{\hyperlink{val:barkuhe}{barku'e}}{}{}{}

\dictentry{proyectil}{}{}{}
{\hyperlink{val:danti}{danti}}{}{}{}

\dictentry{prueba}{}{}{}
{\hyperlink{val:cipra}{cipra}}{}{}{}

\dictentry{pubis}{}{}{}
{\hyperlink{val:plibu}{plibu}}{}{}{}

\dictentry{publico}{}{}{}
{\hyperlink{val:gubni}{gubni}}{}{}{}

\dictentry{pucuyo}{}{}{}
{\hyperlink{val:ctecmocpi}{ctecmocpi}}{}{}{}

\dictentry{pudrirse}{}{}{}
{\hyperlink{val:fusra}{fusra}}{}{}{}

\dictentry{puente}{}{}{}
{\hyperlink{val:cripu}{cripu}}{}{}{}

\dictentry{puerco espín}{}{}{}
{\hyperlink{val:jesyratcu}{jesyratcu}}{}{}{}

\dictentry{puerta}{}{}{}
{\hyperlink{val:vorme}{vorme}}{}{}{}

\dictentry{pulcro}{}{}{}
{\hyperlink{val:cnici}{cnici}}{}{}{}

\dictentry{pulga}{}{}{}
{\hyperlink{val:civla}{civla}}{}{}{}

\dictentry{pulgar}{}{}{}
{\hyperlink{val:tamji}{tamji}}{}{}{}

\dictentry{pulmón}{}{}{}
{\hyperlink{val:fepri}{fepri}}{}{}{}

\dictentry{punta}{}{}{}
{\hyperlink{val:jipno}{jipno}}{}{}{}

\dictentry{punto}{}{}{}
{\hyperlink{val:mokca}{mokca}}{}{}{}

\dictentry{punto decimal}{}{}{}
{\hyperlink{val:pi}{pi}}{}{}{}

\dictentry{punto para colgar hamaca}{}{}{}
{\hyperlink{val:dadycka}{te dadycka}}{}{}{}

\dictentry{puntuar}{}{}{}
{\hyperlink{val:pandi}{pandi}}{}{}{}

\dictentry{puro}{}{}{}
{\hyperlink{val:curve}{curve}}{}{}{}

\dictentry{¿qué clase?}{}{}{}
{\hyperlink{val:sekai ma}{sekai ma}}{}{}{}

\dictchar{Q}\phantomsection\addcontentsline{toc}{section}{Q}
\dictentry{que pertenece a}{}{}{}
{\hyperlink{val:pohe}{po'e}}{}{}{}

\dictentry{quedar}{}{}{}
{\hyperlink{val:stali}{stali}}{}{}{}

\dictentry{queja}{}{}{}
{\hyperlink{val:oi}{oi}}{}{}{}

\dictentry{queja espiritual}{}{}{}
{\hyperlink{val:oirehe}{oire'e}}{}{}{}

\dictentry{queja sexual}{}{}{}
{\hyperlink{val:oirohu}{oiro'u}}{}{}{}

\dictentry{quemarse}{}{}{}
{\hyperlink{val:jelca}{jelca}}{}{}{}

\dictentry{querer}{}{}{}
{\hyperlink{val:djica}{djica}}{}{}{}

\dictentry{querido}{}{}{}
{\hyperlink{val:dirba}{dirba}}{}{}{}

\dictentry{queso}{}{}{}
{\hyperlink{val:cirla}{cirla}}{}{}{}

\dictentry{quieto}{}{}{}
{\hyperlink{val:cando}{cando}}{}{}{}

\dictentry{quitar}{}{}{}
{\hyperlink{val:lebna}{lebna}}{}{}{}

\dictchar{R}\phantomsection\addcontentsline{toc}{section}{R}
\dictentry{r}{}{}{}
{\hyperlink{val:ry}{ry}}{}{}{}

\dictentry{rabo}{}{}{}
{\hyperlink{val:rebla}{rebla}}{}{}{}

\dictentry{racional}{}{}{}
{\hyperlink{val:racli}{racli}}{}{}{}

\dictentry{radián}{}{}{}
{\hyperlink{val:radno}{radno}}{}{}{}

\dictentry{radical}{}{}{}
{\hyperlink{val:gismu}{gismu}}{}{}{}

\dictentry{radiotransmitir}{}{}{}
{\hyperlink{val:cradi}{cradi}}{}{}{}

\dictentry{raíz}{}{}{}
{\hyperlink{val:genja}{genja}}{}{}{}

\dictentry{raíz enésima de}{}{}{}
{\hyperlink{val:feha}{fe'a}}{}{}{}

\dictentry{rama}{}{}{}
{\hyperlink{val:jimca}{jimca}}{}{}{}

\dictentry{rampa}{}{}{}
{\hyperlink{val:salpo}{salpo}}{}{}{}

\dictentry{ranura}{}{}{}
{\hyperlink{val:skuro}{skuro}}{}{}{}

\dictentry{rapido}{}{}{}
{\hyperlink{val:sutra}{sutra}}{}{}{}

\dictentry{rarely}{}{}{}
{\hyperlink{val:pisohuroi}{piso'uroi}}{}{}{}

\dictentry{raro}{}{}{}
{\hyperlink{val:rirci}{rirci}}{}{}{}

\dictentry{rascar}{}{}{}
{\hyperlink{val:sraku}{sraku}}{}{}{}

\dictentry{rata}{}{}{}
{\hyperlink{val:ratcu}{ratcu}}{}{}{}

\dictentry{ratón}{}{}{}
{\hyperlink{val:smacu}{smacu}}{}{}{}

\dictentry{razón}{}{}{}
{\hyperlink{val:pahi}{pa'i}}{}{}{}

\dictentry{reaccionar}{}{}{}
{\hyperlink{val:frati}{frati}}{}{}{}

\dictentry{realidad}{}{}{}
{\hyperlink{val:caha}{ca'a}}{}{}{}

\dictentry{realmente es}{}{}{}
{\hyperlink{val:cacaha}{caca'a}}{}{}{}

\dictentry{realmente fué}{}{}{}
{\hyperlink{val:pucaha}{puca'a}}{}{}{}

\dictentry{realmente será}{}{}{}
{\hyperlink{val:bacaha}{baca'a}}{}{}{}

\dictentry{rebanada}{}{}{}
{\hyperlink{val:panlo}{panlo}}{}{}{}

\dictentry{rechazo}{}{}{}
{\hyperlink{val:aunai}{aunai}}{}{}{}

\dictentry{recibido}{}{}{}
{\hyperlink{val:jehe}{je'e}}{}{}{}

\dictentry{recíproco de}{}{}{}
{\hyperlink{val:fahi}{fa'i}}{}{}{}

\dictentry{recordar}{}{}{}
{\hyperlink{val:morji}{morji}}{}{}{}

\dictentry{rectificar}{}{}{}
{\hyperlink{val:dragau}{dragau}}{}{}{}

\dictentry{recuerdo}{}{}{}
{\hyperlink{val:bahanai}{ba'anai}}{}{}{}

\dictentry{recurrir}{}{}{}
{\hyperlink{val:rapli}{rapli}}{}{}{}

\dictentry{red}{}{}{}
{\hyperlink{val:julne}{julne}}{}{}{}

\dictentry{redondeado hacia abajo}{}{}{}
{\hyperlink{val:jihinihu}{ji'ini'u}}{}{}{}

\dictentry{redondeado hacia arriba}{}{}{}
{\hyperlink{val:jihimahu}{ji'ima'u}}{}{}{}

\dictentry{redondo}{}{}{}
{\hyperlink{val:cukla}{cukla}}{}{}{}

\dictentry{reemplazado por}{}{}{}
{\hyperlink{val:bahi}{ba'i}}{}{}{}

\dictentry{reemplazar}{}{}{}
{\hyperlink{val:basti}{basti}}{}{}{}

\dictentry{referencia}{}{}{}
{\hyperlink{val:manri}{manri}}{}{}{}

\dictentry{reflejar}{}{}{}
{\hyperlink{val:minra}{minra}}{}{}{}

\dictentry{registro}{}{}{}
{\hyperlink{val:vreji}{vreji}}{}{}{}

\dictentry{regla}{}{}{}
{\hyperlink{val:javni}{javni}}{}{}{}

\dictentry{regresando al punto principal}{}{}{}
{\hyperlink{val:tahonai}{ta'onai}}{}{}{}

\dictentry{regular}{}{}{}
{\hyperlink{val:dikni}{dikni}}{}{}{}

\dictentry{regularmente}{}{}{}
{\hyperlink{val:dihi}{di'i}}{}{}{}

\dictentry{reír}{}{}{}
{\hyperlink{val:cmila}{cmila}}{}{}{}

\dictentry{relacionado}{}{}{}
{\hyperlink{val:ckini}{ckini}}{}{}{}

\dictentry{relacionado con}{}{}{}
{\hyperlink{val:sekihi}{seki'i}}{}{}{}

\dictentry{relajación}{}{}{}
{\hyperlink{val:ohu}{o'u}}{}{}{}

\dictentry{relajarse}{}{}{}
{\hyperlink{val:surla}{surla}}{}{}{}

\dictentry{relámpago}{}{}{}
{\hyperlink{val:lindi}{lindi}}{}{}{}

\dictentry{relativo de abstracción}{}{}{}
{\hyperlink{val:cehu}{ce'u}}{}{}{}

\dictentry{relativo de alcance largo}{}{}{}
{\hyperlink{val:vuho}{vu'o}}{}{}{}

\dictentry{relevo de promesa}{}{}{}
{\hyperlink{val:nuhecuhi}{nu'ecu'i}}{}{}{}

\dictentry{religión}{}{}{}
{\hyperlink{val:lijda}{lijda}}{}{}{}

\dictentry{reloj}{}{}{}
{\hyperlink{val:junla}{junla}}{}{}{}

\dictentry{remover}{}{}{}
{\hyperlink{val:guska}{guska}}{}{}{}

\dictentry{reparar}{}{}{}
{\hyperlink{val:cikre}{cikre}}{}{}{}

\dictentry{repetirse}{}{}{}
{\hyperlink{val:krefu}{krefu}}{}{}{}

\dictentry{repita}{}{}{}
{\hyperlink{val:keho}{ke'o}}{}{}{}

\dictentry{repitiendo}{}{}{}
{\hyperlink{val:kehu}{ke'u}}{}{}{}

\dictentry{repollo}{}{}{}
{\hyperlink{val:kobli}{kobli}}{}{}{}

\dictentry{reposo}{}{}{}
{\hyperlink{val:ahinai}{a'inai}}{}{}{}

\dictentry{representado por}{}{}{}
{\hyperlink{val:kahi}{ka'i}}{}{}{}

\dictentry{representando a}{}{}{}
{\hyperlink{val:sekahi}{seka'i}}{}{}{}

\dictentry{representando en}{}{}{}
{\hyperlink{val:tekahi}{teka'i}}{}{}{}

\dictentry{representar}{}{}{}
{\hyperlink{val:krati}{krati}}{}{}{}

\dictentry{reptil}{}{}{}
{\hyperlink{val:respa}{respa}}{}{}{}

\dictentry{repugnar}{}{}{}
{\hyperlink{val:rigni}{rigni}}{}{}{}

\dictentry{repulsión}{}{}{}
{\hyperlink{val:ahunai}{a'unai}}{}{}{}

\dictentry{requiriendo}{}{}{}
{\hyperlink{val:sau}{sau}}{}{}{}

\dictentry{resbalar}{}{}{}
{\hyperlink{val:sakli}{sakli}}{}{}{}

\dictentry{respeto}{}{}{}
{\hyperlink{val:io}{io}}{}{}{}

\dictentry{respirar}{}{}{}
{\hyperlink{val:vasxu}{vasxu}}{}{}{}

\dictentry{responder}{}{}{}
{\hyperlink{val:spuda}{spuda}}{}{}{}

\dictentry{responsable}{}{}{}
{\hyperlink{val:fuzme}{fuzme}}{}{}{}

\dictentry{respuesta}{}{}{}
{\hyperlink{val:danfu}{danfu}}{}{}{}

\dictentry{restaurante}{}{}{}
{\hyperlink{val:gusta}{gusta}}{}{}{}

\dictentry{restricción}{}{}{}
{\hyperlink{val:ehi}{e'i}}{}{}{}

\dictentry{restringido}{}{}{}
{\hyperlink{val:rinju}{rinju}}{}{}{}

\dictentry{resultado}{}{}{}
{\hyperlink{val:jalge}{jalge}}{}{}{}

\dictentry{resultado a pesar de}{}{}{}
{\hyperlink{val:sejahenai}{seja'enai}}{}{}{}

\dictentry{resultado de}{}{}{}
{\hyperlink{val:sejahe}{seja'e}}{}{}{}

\dictentry{retomativo}{}{}{}
{\hyperlink{val:diha}{di'a}}{}{}{}

\dictentry{reverso de}{}{}{}
{\hyperlink{val:fahe}{fa'e}}{}{}{}

\dictentry{revolver}{}{}{}
{\hyperlink{val:jicla}{jicla}}{}{}{}

\dictentry{rico}{}{}{}
{\hyperlink{val:ricfu}{ricfu}}{}{}{}

\dictentry{rígido}{}{}{}
{\hyperlink{val:tinsa}{tinsa}}{}{}{}

\dictentry{rimar}{}{}{}
{\hyperlink{val:rimni}{rimni}}{}{}{}

\dictentry{rincón}{}{}{}
{\hyperlink{val:kojna}{kojna}}{}{}{}

\dictentry{río}{}{}{}
{\hyperlink{val:rirxe}{rirxe}}{}{}{}

\dictentry{ritmo}{}{}{}
{\hyperlink{val:rilti}{rilti}}{}{}{}

\dictentry{rito}{}{}{}
{\hyperlink{val:ritli}{ritli}}{}{}{}

\dictentry{roble}{}{}{}
{\hyperlink{val:urci}{urci}}{}{}{}

\dictentry{roca}{}{}{}
{\hyperlink{val:rokci}{rokci}}{}{}{}

\dictentry{rodar}{}{}{}
{\hyperlink{val:gunro}{gunro}}{}{}{}

\dictentry{rodear}{}{}{}
{\hyperlink{val:sruri}{sruri}}{}{}{}

\dictentry{rodilla}{}{}{}
{\hyperlink{val:cidni}{cidni}}{}{}{}

\dictentry{rogar}{}{}{}
{\hyperlink{val:pikci}{pikci}}{}{}{}

\dictentry{rojo}{}{}{}
{\hyperlink{val:xunre}{xunre}}{}{}{}

\dictentry{romperse}{}{}{}
{\hyperlink{val:porpi}{porpi}}{}{}{}

\dictentry{rosa}{}{}{}
{\hyperlink{val:rozgu}{rozgu}}{}{}{}

\dictentry{rosario}{}{}{}
{\hyperlink{val:bidjylinsi}{bidjylinsi}}{}{}{}

\dictentry{roto}{}{}{}
{\hyperlink{val:spofu}{spofu}}{}{}{}

\dictentry{rueda}{}{}{}
{\hyperlink{val:xislu}{xislu}}{}{}{}

\dictentry{ruido}{}{}{}
{\hyperlink{val:savru}{savru}}{}{}{}

\dictentry{rural}{}{}{}
{\hyperlink{val:nurma}{nurma}}{}{}{}

\dictentry{ruso}{}{}{}
{\hyperlink{val:rusko}{rusko}}{}{}{}

\dictentry{ruta}{}{}{}
{\hyperlink{val:pluta}{pluta}}{}{}{}

\dictchar{S}\phantomsection\addcontentsline{toc}{section}{S}
\dictentry{s}{}{}{}
{\hyperlink{val:sy}{sy}}{}{}{}

\dictentry{saber}{}{}{}
{\hyperlink{val:djuno}{djuno}}{}{}{}

\dictentry{sabido por}{}{}{}
{\hyperlink{val:duho}{du'o}}{}{}{}

\dictentry{sabido que}{}{}{}
{\hyperlink{val:seduho}{sedu'o}}{}{}{}

\dictentry{sabido sobre}{}{}{}
{\hyperlink{val:teduho}{tedu'o}}{}{}{}

\dictentry{sabio}{}{}{}
{\hyperlink{val:prije}{prije}}{}{}{}

\dictentry{saciedad}{}{}{}
{\hyperlink{val:behunai}{be'unai}}{}{}{}

\dictentry{sacrilegio}{}{}{}
{\hyperlink{val:rehenai}{re'enai}}{}{}{}

\dictentry{sagrado}{}{}{}
{\hyperlink{val:censa}{censa}}{}{}{}

\dictentry{sal}{}{}{}
{\hyperlink{val:silna}{silna}}{}{}{}

\dictentry{salir}{}{}{}
{\hyperlink{val:barkla}{barkla}}{}{}{}

\dictentry{salmón}{}{}{}
{\hyperlink{val:salmone}{salmone}}{}{}{}

\dictentry{salsa}{}{}{}
{\hyperlink{val:sanso}{sanso}}{}{}{}

\dictentry{saltar}{}{}{}
{\hyperlink{val:plipe}{plipe}}{}{}{}

\dictentry{saludar}{}{}{}
{\hyperlink{val:rinsa}{rinsa}}{}{}{}

\dictentry{saludo}{}{}{}
{\hyperlink{val:coi}{coi}}{}{}{}

\dictentry{saludo al pasar}{}{}{}
{\hyperlink{val:coicoho}{coico'o}}{}{}{}

\dictentry{salvaje}{}{}{}
{\hyperlink{val:cilce}{cilce}}{}{}{}

\dictentry{sándalo}{}{}{}
{\hyperlink{val:tcandana}{tcandana}}{}{}{}

\dictentry{sandwich}{}{}{}
{\hyperlink{val:snuji}{snuji}}{}{}{}

\dictentry{sangre}{}{}{}
{\hyperlink{val:ciblu}{ciblu}}{}{}{}

\dictentry{sano}{}{}{}
{\hyperlink{val:kanro}{kanro}}{}{}{}

\dictentry{sánscrito}{}{}{}
{\hyperlink{val:srito}{srito}}{}{}{}

\dictentry{sartén}{}{}{}
{\hyperlink{val:tansi}{tansi}}{}{}{}

\dictentry{satélite}{}{}{}
{\hyperlink{val:mluni}{mluni}}{}{}{}

\dictentry{satisfacer}{}{}{}
{\hyperlink{val:mansa}{mansa}}{}{}{}

\dictentry{sauce}{}{}{}
{\hyperlink{val:sailce}{sailce}}{}{}{}

\dictentry{saudita}{}{}{}
{\hyperlink{val:sadjo}{sadjo}}{}{}{}

\dictentry{sé culturalmente}{}{}{}
{\hyperlink{val:kahu}{ka'u}}{}{}{}

\dictentry{sé internamente}{}{}{}
{\hyperlink{val:seho}{se'o}}{}{}{}

\dictentry{sea o no preposicionado}{}{}{}
{\hyperlink{val:gu}{gu}}{}{}{}

\dictentry{sección 0}{}{}{}
{\hyperlink{val:nomoho}{nomo'o}}{}{}{}

\dictentry{sección 1}{}{}{}
{\hyperlink{val:pamoho}{pamo'o}}{}{}{}

\dictentry{seco}{}{}{}
{\hyperlink{val:sudga}{sudga}}{}{}{}

\dictentry{secuencia}{}{}{}
{\hyperlink{val:porsi}{porsi}}{}{}{}

\dictentry{secuencia según reglas ...}{}{}{}
{\hyperlink{val:sepohi}{sepo'i}}{}{}{}

\dictentry{seda}{}{}{}
{\hyperlink{val:silka}{silka}}{}{}{}

\dictentry{seguir}{}{}{}
{\hyperlink{val:jersi}{jersi}}{}{}{}

\dictentry{según epistemología ...}{}{}{}
{\hyperlink{val:veduho}{vedu'o}}{}{}{}

\dictentry{¿según qué lógica?}{}{}{}
{\hyperlink{val:nihi ma}{ni'i ma}}{}{}{}

\dictentry{segundo}{}{}{}
{\hyperlink{val:snidu}{snidu}}{}{}{}

\dictentry{seguridad}{}{}{}
{\hyperlink{val:iinai}{iinai}}{}{}{}

\dictentry{seguro}{}{}{}
{\hyperlink{val:snura}{snura}}{}{}{}

\dictentry{seis}{}{}{}
{\hyperlink{val:xa}{xa}}{}{}{}

\dictentry{selbri a modal}{}{}{}
{\hyperlink{val:fiho}{fi'o}}{}{}{}

\dictentry{selbri a operador}{}{}{}
{\hyperlink{val:nahu}{na'u}}{}{}{}

\dictentry{selbri a operando}{}{}{}
{\hyperlink{val:nihe}{ni'e}}{}{}{}

\dictentry{selbri cardinal}{}{}{}
{\hyperlink{val:mei}{mei}}{}{}{}

\dictentry{selbri de probabilidad}{}{}{}
{\hyperlink{val:cuho}{cu'o}}{}{}{}

\dictentry{selbri escalar}{}{}{}
{\hyperlink{val:vahe}{va'e}}{}{}{}

\dictentry{selbri fraccional}{}{}{}
{\hyperlink{val:sihe}{si'e}}{}{}{}

\dictentry{selbri ordinal}{}{}{}
{\hyperlink{val:moi}{moi}}{}{}{}

\dictentry{selección de alfabeto}{}{}{}
{\hyperlink{val:zai}{zai}}{}{}{}

\dictentry{semana}{}{}{}
{\hyperlink{val:jeftu}{jeftu}}{}{}{}

\dictentry{sembrar}{}{}{}
{\hyperlink{val:sombo}{sombo}}{}{}{}

\dictentry{semilla}{}{}{}
{\hyperlink{val:tsiju}{tsiju}}{}{}{}

\dictentry{semita}{}{}{}
{\hyperlink{val:semto}{semto}}{}{}{}

\dictentry{seno}{}{}{}
{\hyperlink{val:sinso}{sinso}}{}{}{}

\dictentry{sentarse}{}{}{}
{\hyperlink{val:zutse}{zutse}}{}{}{}

\dictentry{sentido por}{}{}{}
{\hyperlink{val:ciho}{ci'o}}{}{}{}

\dictentry{sentir}{}{}{}
{\hyperlink{val:ganse}{ganse}}{}{}{}

\dictentry{separado}{}{}{}
{\hyperlink{val:sepli}{sepli}}{}{}{}

\dictentry{separador de dígitos}{}{}{}
{\hyperlink{val:pihe}{pi'e}}{}{}{}

\dictentry{separador de miles}{}{}{}
{\hyperlink{val:kiho}{ki'o}}{}{}{}

\dictentry{separador de selbri}{}{}{}
{\hyperlink{val:cu}{cu}}{}{}{}

\dictentry{seriedad}{}{}{}
{\hyperlink{val:zohonai}{zo'onai}}{}{}{}

\dictentry{serio}{}{}{}
{\hyperlink{val:junri}{junri}}{}{}{}

\dictentry{servir}{}{}{}
{\hyperlink{val:selfu}{selfu}}{}{}{}

\dictentry{severo}{}{}{}
{\hyperlink{val:jursa}{jursa}}{}{}{}

\dictentry{sexual}{}{}{}
{\hyperlink{val:rohu}{ro'u}}{}{}{}

\dictentry{sí metalinguistico}{}{}{}
{\hyperlink{val:joha}{jo'a}}{}{}{}

\dictentry{siempre}{}{}{}
{\hyperlink{val:roroi}{roroi}}{}{}{}

\dictentry{siete}{}{}{}
{\hyperlink{val:ze}{ze}}{}{}{}

\dictentry{significado}{}{}{}
{\hyperlink{val:smuni}{smuni}}{}{}{}

\dictentry{signo}{}{}{}
{\hyperlink{val:sinxa}{sinxa}}{}{}{}

\dictentry{signo de puntuación}{}{}{}
{\hyperlink{val:lau}{lau}}{}{}{}

\dictentry{sigue más}{}{}{}
{\hyperlink{val:muhonai}{mu'onai}}{}{}{}

\dictentry{sii (sumti)}{}{}{}
{\hyperlink{val:o}{o}}{}{}{}

\dictentry{sii preposicionado}{}{}{}
{\hyperlink{val:go}{go}}{}{}{}

\dictentry{sílaba}{}{}{}
{\hyperlink{val:slaka}{slaka}}{}{}{}

\dictentry{silbar}{}{}{}
{\hyperlink{val:siclu}{siclu}}{}{}{}

\dictentry{silencio}{}{}{}
{\hyperlink{val:smaji}{smaji}}{}{}{}

\dictentry{silla}{}{}{}
{\hyperlink{val:stizu}{stizu}}{}{}{}

\dictentry{símbolo de pausa}{}{}{}
{\hyperlink{val:denpa bu}{denpa bu}}{}{}{}

\dictentry{similar}{}{}{}
{\hyperlink{val:simsa}{simsa}}{}{}{}

\dictentry{similar en (propiedad)}{}{}{}
{\hyperlink{val:vetai}{vetai}}{}{}{}

\dictentry{similarmente}{}{}{}
{\hyperlink{val:siha}{si'a}}{}{}{}

\dictentry{simple}{}{}{}
{\hyperlink{val:sampu}{sampu}}{}{}{}

\dictentry{simplemente}{}{}{}
{\hyperlink{val:sahu}{sa'u}}{}{}{}

\dictentry{sin}{}{}{}
{\hyperlink{val:secau}{secau}}{}{}{}

\dictentry{sin embargo}{}{}{}
{\hyperlink{val:kuhi}{ku'i}}{}{}{}

\dictentry{sin embargo (causal)}{}{}{}
{\hyperlink{val:serihanai}{seri'anai}}{}{}{}

\dictentry{sin embargo (lógica)}{}{}{}
{\hyperlink{val:senihinai}{seni'inai}}{}{}{}

\dictentry{sin embargo (motivo)}{}{}{}
{\hyperlink{val:semuhinai}{semu'inai}}{}{}{}

\dictentry{sin embargo (razón)}{}{}{}
{\hyperlink{val:sekihunai}{seki'unai}}{}{}{}

\dictentry{sincero}{}{}{}
{\hyperlink{val:stace}{stace}}{}{}{}

\dictentry{sirio}{}{}{}
{\hyperlink{val:sirxo}{sirxo}}{}{}{}

\dictentry{sistema}{}{}{}
{\hyperlink{val:ciste}{ciste}}{}{}{}

\dictentry{sistema con función ...}{}{}{}
{\hyperlink{val:secihe}{seci'e}}{}{}{}

\dictentry{situación}{}{}{}
{\hyperlink{val:tcini}{tcini}}{}{}{}

\dictentry{sobre}{}{}{}
{\hyperlink{val:cpana}{cpana}}{}{}{}

\dictentry{socializar}{}{}{}
{\hyperlink{val:jikca}{jikca}}{}{}{}

\dictentry{sodio}{}{}{}
{\hyperlink{val:sodna}{sodna}}{}{}{}

\dictentry{sofá}{}{}{}
{\hyperlink{val:sfofa}{sfofa}}{}{}{}

\dictentry{soga}{}{}{}
{\hyperlink{val:skori}{skori}}{}{}{}

\dictentry{soja}{}{}{}
{\hyperlink{val:sobde}{sobde}}{}{}{}

\dictentry{sol}{}{}{}
{\hyperlink{val:solri}{solri}}{}{}{}

\dictentry{soldado}{}{}{}
{\hyperlink{val:sonci}{sonci}}{}{}{}

\dictentry{sólido}{}{}{}
{\hyperlink{val:sligu}{sligu}}{}{}{}

\dictentry{solo}{}{}{}
{\hyperlink{val:nonkansa}{nonkansa}}{}{}{}

\dictentry{sólo}{}{}{}
{\hyperlink{val:poho}{po'o}}{}{}{}

\dictentry{sólo si preposicionado}{}{}{}
{\hyperlink{val:ganai}{ganai}}{}{}{}

\dictentry{sombra}{}{}{}
{\hyperlink{val:ctino}{ctino}}{}{}{}

\dictentry{sombrero}{}{}{}
{\hyperlink{val:mapku}{mapku}}{}{}{}

\dictentry{son demasiados}{}{}{}
{\hyperlink{val:duhemei}{du'emei}}{}{}{}

\dictentry{son suficientes}{}{}{}
{\hyperlink{val:raumei}{raumei}}{}{}{}

\dictentry{soñar}{}{}{}
{\hyperlink{val:senva}{senva}}{}{}{}

\dictentry{sonido}{}{}{}
{\hyperlink{val:sance}{sance}}{}{}{}

\dictentry{sonreír}{}{}{}
{\hyperlink{val:cisma}{cisma}}{}{}{}

\dictentry{sopa}{}{}{}
{\hyperlink{val:stasu}{stasu}}{}{}{}

\dictentry{sorgo}{}{}{}
{\hyperlink{val:sorgu}{sorgu}}{}{}{}

\dictentry{sorprender}{}{}{}
{\hyperlink{val:spaji}{spaji}}{}{}{}

\dictentry{sorpresa}{}{}{}
{\hyperlink{val:ue}{ue}}{}{}{}

\dictentry{sorpresa?}{}{}{}
{\hyperlink{val:uepei}{uepei}}{}{}{}

\dictentry{sostener}{}{}{}
{\hyperlink{val:sarji}{sarji}}{}{}{}

\dictentry{sosteniendo a}{}{}{}
{\hyperlink{val:sejihu}{seji'u}}{}{}{}

\dictentry{soviético}{}{}{}
{\hyperlink{val:softo}{softo}}{}{}{}

\dictentry{su}{}{}{}
{\hyperlink{val:lekoha}{leko'a}}{}{}{}

\dictentry{suave}{}{}{}
{\hyperlink{val:xutla}{xutla}}{}{}{}

\dictentry{subíndice}{}{}{}
{\hyperlink{val:xi}{xi}}{}{}{}

\dictentry{subir}{}{}{}
{\hyperlink{val:gapkla}{gapkla}}{}{}{}

\dictentry{súbito}{}{}{}
{\hyperlink{val:suksa}{suksa}}{}{}{}

\dictentry{sudamericano}{}{}{}
{\hyperlink{val:ketco}{ketco}}{}{}{}

\dictentry{Suecia}{}{}{}
{\hyperlink{val:sferies}{sferies}}{}{}{}

\dictentry{suerte}{}{}{}
{\hyperlink{val:funca}{funca}}{}{}{}

\dictentry{suficiente}{}{}{}
{\hyperlink{val:banzu}{banzu}}{}{}{}

\dictentry{suficientes}{}{}{}
{\hyperlink{val:rau}{rau}}{}{}{}

\dictentry{sugerencia}{}{}{}
{\hyperlink{val:ehu}{e'u}}{}{}{}

\dictentry{sugerido a}{}{}{}
{\hyperlink{val:tetihi}{teti'i}}{}{}{}

\dictentry{sugerido por}{}{}{}
{\hyperlink{val:tihi}{ti'i}}{}{}{}

\dictentry{sugerir}{}{}{}
{\hyperlink{val:stidi}{stidi}}{}{}{}

\dictentry{sugiriendo}{}{}{}
{\hyperlink{val:setihi}{seti'i}}{}{}{}

\dictentry{sujeto de política}{}{}{}
{\hyperlink{val:plajva}{te plajva}}{}{}{}

\dictentry{suma}{}{}{}
{\hyperlink{val:sumji}{sumji}}{}{}{}

\dictentry{sumatoria}{}{}{}
{\hyperlink{val:snisimsumji}{snisimsumji}}{}{}{}

\dictentry{sumergirse}{}{}{}
{\hyperlink{val:jinru}{jinru}}{}{}{}

\dictentry{sumisión}{}{}{}
{\hyperlink{val:gahinai}{ga'inai}}{}{}{}

\dictentry{sumti ?}{}{}{}
{\hyperlink{val:ma}{ma}}{}{}{}

\dictentry{sumti a selbri}{}{}{}
{\hyperlink{val:me}{me}}{}{}{}

\dictentry{sumti anterior}{}{}{}
{\hyperlink{val:ru}{ru}}{}{}{}

\dictentry{sumti but not}{}{}{}
{\hyperlink{val:enai}{enai}}{}{}{}

\dictentry{sumti conn ?}{}{}{}
{\hyperlink{val:ji}{ji}}{}{}{}

\dictentry{sumti en operando}{}{}{}
{\hyperlink{val:mohe}{mo'e}}{}{}{}

\dictentry{sumti no especificado}{}{}{}
{\hyperlink{val:zohe}{zo'e}}{}{}{}

\dictentry{sumti only if}{}{}{}
{\hyperlink{val:na.a}{na.a}}{}{}{}

\dictentry{sumti reciente}{}{}{}
{\hyperlink{val:ra}{ra}}{}{}{}

\dictentry{sumti recíprocos}{}{}{}
{\hyperlink{val:soi}{soi}}{}{}{}

\dictentry{sumti último}{}{}{}
{\hyperlink{val:ri}{ri}}{}{}{}

\dictentry{sumti whether}{}{}{}
{\hyperlink{val:u}{u}}{}{}{}

\dictentry{superado por}{}{}{}
{\hyperlink{val:mau}{mau}}{}{}{}

\dictentry{superando a}{}{}{}
{\hyperlink{val:meha}{me'a}}{}{}{}

\dictentry{superfectivo}{}{}{}
{\hyperlink{val:zaho}{za'o}}{}{}{}

\dictentry{superficie}{}{}{}
{\hyperlink{val:sefta}{sefta}}{}{}{}

\dictentry{superlativo en}{}{}{}
{\hyperlink{val:serai}{serai}}{}{}{}

\dictentry{superlativo entre}{}{}{}
{\hyperlink{val:verai}{verai}}{}{}{}

\dictentry{suponer}{}{}{}
{\hyperlink{val:sruma}{sruma}}{}{}{}

\dictentry{suposición}{}{}{}
{\hyperlink{val:dahi}{da'i}}{}{}{}

\dictentry{sur}{}{}{}
{\hyperlink{val:snanu}{snanu}}{}{}{}

\dictentry{sustancia}{}{}{}
{\hyperlink{val:xukmi}{xukmi}}{}{}{}

\dictentry{sustraer}{}{}{}
{\hyperlink{val:vimcu}{vimcu}}{}{}{}

\dictchar{T}\phantomsection\addcontentsline{toc}{section}{T}
\dictentry{t}{}{}{}
{\hyperlink{val:ty}{ty}}{}{}{}

\dictentry{tabaco}{}{}{}
{\hyperlink{val:tanko}{tanko}}{}{}{}

\dictentry{tabla}{}{}{}
{\hyperlink{val:tanbo}{tanbo}}{}{}{}

\dictentry{talentoso}{}{}{}
{\hyperlink{val:stati}{stati}}{}{}{}

\dictentry{tallo}{}{}{}
{\hyperlink{val:stani}{stani}}{}{}{}

\dictentry{tamaño}{}{}{}
{\hyperlink{val:nilbra}{nilbra}}{}{}{}

\dictentry{tambor}{}{}{}
{\hyperlink{val:damri}{damri}}{}{}{}

\dictentry{tangente}{}{}{}
{\hyperlink{val:tanjo}{tanjo}}{}{}{}

\dictentry{tangente a}{}{}{}
{\hyperlink{val:zoha}{zo'a}}{}{}{}

\dictentry{tanru conector ?}{}{}{}
{\hyperlink{val:jehi}{je'i}}{}{}{}

\dictentry{tanru o}{}{}{}
{\hyperlink{val:ja}{ja}}{}{}{}

\dictentry{tanru o exclusivo}{}{}{}
{\hyperlink{val:jonai}{jonai}}{}{}{}

\dictentry{tanru pero no}{}{}{}
{\hyperlink{val:jenai}{jenai}}{}{}{}

\dictentry{tanru sea o no}{}{}{}
{\hyperlink{val:ju}{ju}}{}{}{}

\dictentry{tanru sii}{}{}{}
{\hyperlink{val:jo}{jo}}{}{}{}

\dictentry{tanru sólo si}{}{}{}
{\hyperlink{val:naja}{naja}}{}{}{}

\dictentry{tanru y}{}{}{}
{\hyperlink{val:je}{je}}{}{}{}

\dictentry{tanto como}{}{}{}
{\hyperlink{val:duhi}{du'i}}{}{}{}

\dictentry{taoísta}{}{}{}
{\hyperlink{val:dadjo}{dadjo}}{}{}{}

\dictentry{tarde}{}{}{}
{\hyperlink{val:vanci}{vanci}}{}{}{}

\dictentry{tarjeta}{}{}{}
{\hyperlink{val:karda}{karda}}{}{}{}

\dictentry{taza}{}{}{}
{\hyperlink{val:kabri}{kabri}}{}{}{}

\dictentry{té}{}{}{}
{\hyperlink{val:tcati}{tcati}}{}{}{}

\dictentry{techo}{}{}{}
{\hyperlink{val:drudi}{drudi}}{}{}{}

\dictentry{tedio}{}{}{}
{\hyperlink{val:uhinai}{u'inai}}{}{}{}

\dictentry{tedioso}{}{}{}
{\hyperlink{val:tolzdi}{tolzdi}}{}{}{}

\dictentry{tejer}{}{}{}
{\hyperlink{val:nivji}{nivji}}{}{}{}

\dictentry{tela}{}{}{}
{\hyperlink{val:bukpu}{bukpu}}{}{}{}

\dictentry{teléfono}{}{}{}
{\hyperlink{val:fonxa}{fonxa}}{}{}{}

\dictentry{temblar}{}{}{}
{\hyperlink{val:desku}{desku}}{}{}{}

\dictentry{temer}{}{}{}
{\hyperlink{val:terpa}{terpa}}{}{}{}

\dictentry{témpano}{}{}{de hielo}
{\hyperlink{val:bismaha}{bisma'a}}{}{}{}

\dictentry{templo}{}{}{}
{\hyperlink{val:malsi}{malsi}}{}{}{}

\dictentry{temprano}{}{}{}
{\hyperlink{val:clira}{clira}}{}{}{}

\dictentry{tenedor}{}{}{}
{\hyperlink{val:forca}{forca}}{}{}{}

\dictentry{tener}{}{}{}
{\hyperlink{val:ponse}{ponse}}{}{}{}

\dictentry{tener hambre}{}{}{}
{\hyperlink{val:xagji}{xagji}}{}{}{}

\dictentry{tener sed}{}{}{}
{\hyperlink{val:taske}{taske}}{}{}{}

\dictentry{tenso}{}{}{}
{\hyperlink{val:trati}{trati}}{}{}{}

\dictentry{tentar}{}{}{}
{\hyperlink{val:xlura}{xlura}}{}{}{}

\dictentry{tera}{}{}{}
{\hyperlink{val:terto}{terto}}{}{}{}

\dictentry{terco}{}{}{}
{\hyperlink{val:xarnu}{xarnu}}{}{}{}

\dictentry{terreno}{}{}{}
{\hyperlink{val:tumla}{tumla}}{}{}{}

\dictentry{territorio}{}{}{}
{\hyperlink{val:tutra}{tutra}}{}{}{}

\dictentry{territorio imperial}{}{}{}
{\hyperlink{val:sorgugjeha}{te sorgugje'a}}{}{}{}

\dictentry{testículo}{}{}{}
{\hyperlink{val:ganti}{ganti}}{}{}{}

\dictentry{texto omitido}{}{}{}
{\hyperlink{val:liho}{li'o}}{}{}{}

\dictentry{textura}{}{}{}
{\hyperlink{val:tengu}{tengu}}{}{}{}

\dictentry{the x2 del último bridi}{}{}{}
{\hyperlink{val:le segohi}{le sego'i}}{}{}{}

\dictentry{the x3 del último bridi}{}{}{}
{\hyperlink{val:le tegohi}{le tego'i}}{}{}{}

\dictentry{the x4 del último bridi}{}{}{}
{\hyperlink{val:le vegohi}{le vego'i}}{}{}{}

\dictentry{throughout}{}{}{}
{\hyperlink{val:fehecaho}{fe'eca'o}}{}{}{}

\dictentry{tiempo}{}{}{}
{\hyperlink{val:temci}{temci}}{}{}{}

\dictentry{tiempo corto}{}{}{}
{\hyperlink{val:zi}{zi}}{}{}{}

\dictentry{tiempo largo}{}{}{}
{\hyperlink{val:zu}{zu}}{}{}{}

\dictentry{tiempo medio}{}{}{}
{\hyperlink{val:za}{za}}{}{}{}

\dictentry{tiempo verbal cuantificado}{}{}{}
{\hyperlink{val:roi}{roi}}{}{}{}

\dictentry{tiempo verbal ordinal}{}{}{}
{\hyperlink{val:rehu}{re'u}}{}{}{}

\dictentry{tierra}{}{}{}
{\hyperlink{val:terdi}{terdi}}{}{}{}

\dictentry{tigre}{}{}{}
{\hyperlink{val:tirxu}{tirxu}}{}{}{}

\dictentry{tijeras}{}{}{}
{\hyperlink{val:jinci}{jinci}}{}{}{}

\dictentry{timidez}{}{}{}
{\hyperlink{val:uhocuhi}{u'ocu'i}}{}{}{}

\dictentry{tímido}{}{}{}
{\hyperlink{val:toldarsi}{toldarsi}}{}{}{}

\dictentry{tinta}{}{}{}
{\hyperlink{val:xinmo}{xinmo}}{}{}{}

\dictentry{tío}{}{}{}
{\hyperlink{val:famti}{famti}}{}{}{}

\dictentry{típicamente}{}{}{}
{\hyperlink{val:naho}{na'o}}{}{}{}

\dictentry{tirar}{}{}{}
{\hyperlink{val:lacpu}{lacpu}}{}{}{}

\dictentry{tiza}{}{}{}
{\hyperlink{val:bakri}{bakri}}{}{}{}

\dictentry{tocar}{}{}{}
{\hyperlink{val:pencu}{pencu}}{}{}{}

\dictentry{todavía}{}{}{}
{\hyperlink{val:zaho}{za'o}}{}{}{}

\dictentry{todo}{}{}{}
{\hyperlink{val:piro}{piro}}{}{}{}

\dictentry{todos}{}{}{}
{\hyperlink{val:roda}{roda}}{}{}{}

\dictentry{todos excepto}{}{}{}
{\hyperlink{val:daha}{da'a}}{}{}{}

\dictentry{tomar}{}{}{}
{\hyperlink{val:cpacu}{cpacu}}{}{}{}

\dictentry{tomar prestado}{}{}{}
{\hyperlink{val:jbera}{jbera}}{}{}{}

\dictentry{tomate}{}{}{}
{\hyperlink{val:tamca}{tamca}}{}{}{}

\dictentry{tono}{}{}{}
{\hyperlink{val:tonga}{tonga}}{}{}{}

\dictentry{tonto}{}{}{}
{\hyperlink{val:bebna}{bebna}}{}{}{}

\dictentry{tópico nuevo}{}{}{}
{\hyperlink{val:niho}{ni'o}}{}{}{}

\dictentry{tópico viejo}{}{}{}
{\hyperlink{val:nohi}{no'i}}{}{}{}

\dictentry{torcerse}{}{}{}
{\hyperlink{val:torni}{torni}}{}{}{}

\dictentry{torcido}{}{}{}
{\hyperlink{val:korcu}{korcu}}{}{}{}

\dictentry{tornillo}{}{}{}
{\hyperlink{val:klupe}{klupe}}{}{}{}

\dictentry{torpeza}{}{}{}
{\hyperlink{val:juxre}{juxre}}{}{}{}

\dictentry{toser}{}{}{}
{\hyperlink{val:kafke}{kafke}}{}{}{}

\dictentry{trabajar}{}{}{}
{\hyperlink{val:gunka}{gunka}}{}{}{}

\dictentry{traducir}{}{}{}
{\hyperlink{val:fanva}{fanva}}{}{}{}

\dictentry{tragar}{}{}{}
{\hyperlink{val:tunlo}{tunlo}}{}{}{}

\dictentry{tragedia}{}{}{}
{\hyperlink{val:betri}{betri}}{}{}{}

\dictentry{transferir}{}{}{}
{\hyperlink{val:benji}{benji}}{}{}{}

\dictentry{transitorio}{}{}{}
{\hyperlink{val:zasni}{zasni}}{}{}{}

\dictentry{transmitido desde}{}{}{}
{\hyperlink{val:vebehi}{vebe'i}}{}{}{}

\dictentry{transmitido vía}{}{}{}
{\hyperlink{val:xebehi}{xebe'i}}{}{}{}

\dictentry{transmitiendo}{}{}{}
{\hyperlink{val:sebehi}{sebe'i}}{}{}{}

\dictentry{transmitir (TV)}{}{}{}
{\hyperlink{val:tivni}{tivni}}{}{}{}

\dictentry{transparente}{}{}{}
{\hyperlink{val:klina}{klina}}{}{}{}

\dictentry{transpiración}{}{}{}
{\hyperlink{val:xasne}{xasne}}{}{}{}

\dictentry{transpuesto}{}{}{}
{\hyperlink{val:reha}{re'a}}{}{}{}

\dictentry{trasero}{}{}{}
{\hyperlink{val:zargu}{zargu}}{}{}{}

\dictentry{tratar}{}{}{}
{\hyperlink{val:troci}{troci}}{}{}{}

\dictentry{través}{}{}{}
{\hyperlink{val:ragve}{ragve}}{}{}{}

\dictentry{tren}{}{}{}
{\hyperlink{val:trene}{trene}}{}{}{}

\dictentry{trepar}{}{}{}
{\hyperlink{val:cpare}{cpare}}{}{}{}

\dictentry{tres}{}{}{}
{\hyperlink{val:ci}{ci}}{}{}{}

\dictentry{tres veces}{}{}{}
{\hyperlink{val:ciroi}{ciroi}}{}{}{}

\dictentry{triángulo}{}{}{trigonometría}
{\hyperlink{val:cibjgacmaci}{se cibjgacmaci}}{}{}{}

\dictentry{trigo}{}{}{}
{\hyperlink{val:maxri}{maxri}}{}{}{}

\dictentry{trigonometría}{}{}{}
{\hyperlink{val:cibjgacmaci}{cibjgacmaci}}{}{}{}

\dictentry{trigonométrico}{}{}{}
{\hyperlink{val:cibjgacmaci}{cibjgacmaci}}{}{}{}

\dictentry{triste}{}{}{}
{\hyperlink{val:badri}{badri}}{}{}{}

\dictentry{tristeza}{}{}{}
{\hyperlink{val:uinai}{uinai}}{}{}{}

\dictentry{trompeta}{}{}{}
{\hyperlink{val:tabra}{tabra}}{}{}{}

\dictentry{trucha}{}{}{}
{\hyperlink{val:salmone}{salmone}}{}{}{}

\dictentry{tú}{}{}{}
{\hyperlink{val:do}{do}}{}{}{}

\dictentry{tu}{}{}{}
{\hyperlink{val:medomoi}{medomoi}}{}{}{}

\dictentry{tú/ustedes}{}{}{}
{\hyperlink{val:do}{do}}{}{}{}

\dictentry{tú/ustedes y otros}{}{}{}
{\hyperlink{val:doho}{do'o}}{}{}{}

\dictentry{tú/ustedes y yo}{}{}{}
{\hyperlink{val:miho}{mi'o}}{}{}{}

\dictentry{tú/ustedes, otros y yo}{}{}{}
{\hyperlink{val:maha}{ma'a}}{}{}{}

\dictentry{tubo}{}{}{}
{\hyperlink{val:tubnu}{tubnu}}{}{}{}

\dictentry{tulipán}{}{}{}
{\hyperlink{val:tujli}{tujli}}{}{}{}

\dictentry{túnica}{}{}{}
{\hyperlink{val:pastu}{pastu}}{}{}{}

\dictentry{turno}{}{}{en un juego}
{\hyperlink{val:kelkahu}{kelka'u}}{}{}{}

\dictentry{tuyo}{}{}{}
{\hyperlink{val:medomoi}{medomoi}}{}{}{}

\dictchar{U}\phantomsection\addcontentsline{toc}{section}{U}
\dictentry{u}{}{}{}
{\hyperlink{val:ubu}{ubu}}{}{}{}

\dictentry{ucraniano}{}{}{}
{\hyperlink{val:vukro}{vukro}}{}{}{}

\dictentry{último bridi actualizado}{}{}{}
{\hyperlink{val:gohiraho}{go'ira'o}}{}{}{}

\dictentry{un grupo de los que realmente son}{}{}{}
{\hyperlink{val:loi}{loi}}{}{}{}

\dictentry{un poco de}{}{}{}
{\hyperlink{val:pisohu}{piso'u}}{}{}{}

\dictentry{un tiempo antes}{}{}{}
{\hyperlink{val:puza}{puza}}{}{}{}

\dictentry{una vez}{}{}{}
{\hyperlink{val:paroi}{paroi}}{}{}{}

\dictentry{unas pocas veces}{}{}{}
{\hyperlink{val:sohuroi}{so'uroi}}{}{}{}

\dictentry{unicornio}{}{}{}
{\hyperlink{val:pavyseljirna}{pavyseljirna}}{}{}{}

\dictentry{unido}{}{}{}
{\hyperlink{val:jorne}{jorne}}{}{}{}

\dictentry{uniforme}{}{}{}
{\hyperlink{val:manfo}{manfo}}{}{}{}

\dictentry{unión}{}{}{}
{\hyperlink{val:johe}{jo'e}}{}{}{}

\dictentry{unión de cláusula relativa}{}{}{}
{\hyperlink{val:zihe}{zi'e}}{}{}{}

\dictentry{universo}{}{}{}
{\hyperlink{val:munje}{munje}}{}{}{}

\dictentry{uno}{}{}{}
{\hyperlink{val:pa}{pa}}{}{}{}

\dictentry{uno(s) que realmente es/son}{}{}{}
{\hyperlink{val:lo}{lo}}{}{}{}

\dictentry{urdu}{}{}{}
{\hyperlink{val:xurdo}{xurdo}}{}{}{}

\dictentry{usado por}{}{}{}
{\hyperlink{val:piho}{pi'o}}{}{}{}

\dictentry{usando}{}{}{}
{\hyperlink{val:sepiho}{sepi'o}}{}{}{}

\dictentry{usar}{}{}{}
{\hyperlink{val:pilno}{pilno}}{}{}{}

\dictentry{uso para}{}{}{}
{\hyperlink{val:tepiho}{tepi'o}}{}{}{}

\dictentry{útero}{}{}{}
{\hyperlink{val:gutra}{gutra}}{}{}{}

\dictentry{uva}{}{}{}
{\hyperlink{val:vanjba}{vanjba}}{}{}{}

\dictchar{V}\phantomsection\addcontentsline{toc}{section}{V}
\dictentry{v}{}{}{}
{\hyperlink{val:vy}{vy}}{}{}{}

\dictentry{va a ser}{}{}{}
{\hyperlink{val:baba}{baba}}{}{}{}

\dictentry{vaca}{}{}{}
{\hyperlink{val:bakni}{bakni}}{}{}{}

\dictentry{vacilación}{}{}{}
{\hyperlink{val:y}{y}}{}{}{}

\dictentry{vacío}{}{}{}
{\hyperlink{val:kunti}{kunti}}{}{}{}

\dictentry{vacuna}{}{}{}
{\hyperlink{val:jinku}{jinku}}{}{}{}

\dictentry{vagina}{}{}{}
{\hyperlink{val:vibna}{vibna}}{}{}{}

\dictentry{valiente}{}{}{}
{\hyperlink{val:virnu}{virnu}}{}{}{}

\dictentry{valor}{}{}{}
{\hyperlink{val:vamji}{vamji}}{}{}{}

\dictentry{valor absoluto}{}{}{}
{\hyperlink{val:nacnilbra}{nacnilbra}}{}{}{}

\dictentry{valor de verdad}{}{}{}
{\hyperlink{val:jei}{jei}}{}{}{}

\dictentry{valor típico}{}{}{}
{\hyperlink{val:noho}{no'o}}{}{}{}

\dictentry{varios}{}{}{}
{\hyperlink{val:soho}{so'o}}{}{}{}

\dictentry{veces}{}{}{}
{\hyperlink{val:roi}{roi}}{}{}{}

\dictentry{vector}{}{}{tridimensional}
{\hyperlink{val:cibnacmei}{cibnacmei}}{}{}{}

\dictentry{vector}{}{}{n dimensiones}
{\hyperlink{val:nacmei}{nacmei}}{}{}{}

\dictentry{vector}{}{}{bidimensional}
{\hyperlink{val:relnacmei}{relnacmei}}{}{}{}

\dictentry{vegetal}{}{}{}
{\hyperlink{val:stagi}{stagi}}{}{}{}

\dictentry{vehículo}{}{}{}
{\hyperlink{val:marce}{marce}}{}{}{}

\dictentry{vela}{}{}{}
{\hyperlink{val:falnu}{falnu}}{}{}{}

\dictentry{vender}{}{}{}
{\hyperlink{val:vecnu}{vecnu}}{}{}{}

\dictentry{veneno}{}{}{}
{\hyperlink{val:vindu}{vindu}}{}{}{}

\dictentry{vengarse}{}{}{}
{\hyperlink{val:venfu}{venfu}}{}{}{}

\dictentry{ventana}{}{}{}
{\hyperlink{val:canko}{canko}}{}{}{}

\dictentry{ver}{}{}{}
{\hyperlink{val:viska}{viska}}{}{}{}

\dictentry{verano}{}{}{}
{\hyperlink{val:crisa}{crisa}}{}{}{}

\dictentry{verdad}{}{}{}
{\hyperlink{val:jehu}{je'u}}{}{}{}

\dictentry{verde}{}{}{}
{\hyperlink{val:crino}{crino}}{}{}{}

\dictentry{vergüenza}{}{}{}
{\hyperlink{val:ohanai}{o'anai}}{}{}{}

\dictentry{vertical}{}{}{}
{\hyperlink{val:sraji}{sraji}}{}{}{}

\dictentry{vestir}{}{}{}
{\hyperlink{val:dasni}{dasni}}{}{}{}

\dictentry{viajar}{}{}{}
{\hyperlink{val:litru}{litru}}{}{}{}

\dictentry{víbora}{}{}{}
{\hyperlink{val:since}{since}}{}{}{}

\dictentry{vice}{}{}{}
{\hyperlink{val:vipsi}{vipsi}}{}{}{}

\dictentry{vidrio}{}{}{}
{\hyperlink{val:blaci}{blaci}}{}{}{}

\dictentry{viejo}{}{}{}
{\hyperlink{val:slabu}{slabu}}{}{}{}

\dictentry{vino}{}{}{}
{\hyperlink{val:vanju}{vanju}}{}{}{}

\dictentry{violento}{}{}{}
{\hyperlink{val:vlile}{vlile}}{}{}{}

\dictentry{violeta}{}{}{}
{\hyperlink{val:zirpu}{zirpu}}{}{}{}

\dictentry{virtud}{}{}{}
{\hyperlink{val:vuhe}{vu'e}}{}{}{}

\dictentry{virtuoso}{}{}{}
{\hyperlink{val:vrude}{vrude}}{}{}{}

\dictentry{virus}{}{}{}
{\hyperlink{val:vidru}{vidru}}{}{}{}

\dictentry{viscoso}{}{}{}
{\hyperlink{val:viknu}{viknu}}{}{}{}

\dictentry{visitar}{}{}{}
{\hyperlink{val:vitke}{vitke}}{}{}{}

\dictentry{vista}{}{}{}
{\hyperlink{val:jvinu}{jvinu}}{}{}{}

\dictentry{vivir}{}{}{}
{\hyperlink{val:jmive}{jmive}}{}{}{}

\dictentry{volar}{}{}{}
{\hyperlink{val:vofli}{vofli}}{}{}{}

\dictentry{volver}{}{}{}
{\hyperlink{val:xruti}{xruti}}{}{}{}

\dictentry{vomitar}{}{}{}
{\hyperlink{val:vamtu}{vamtu}}{}{}{}

\dictentry{voz}{}{}{}
{\hyperlink{val:voksa}{voksa}}{}{}{}

\dictentry{vulva}{}{}{}
{\hyperlink{val:vlagi}{vlagi}}{}{}{}

\dictchar{W}\phantomsection\addcontentsline{toc}{section}{W}
\dictentry{whisky}{}{}{}
{\hyperlink{val:uiski}{uiski}}{}{}{}

\dictchar{X}\phantomsection\addcontentsline{toc}{section}{X}
\dictentry{x}{}{}{}
{\hyperlink{val:xy}{xy}}{}{}{}

\dictentry{x1 ello}{}{}{}
{\hyperlink{val:voha}{vo'a}}{}{}{}

\dictentry{x2 ello}{}{}{}
{\hyperlink{val:vohe}{vo'e}}{}{}{}

\dictentry{x3 ello}{}{}{}
{\hyperlink{val:vohi}{vo'i}}{}{}{}

\dictentry{x4 ello}{}{}{}
{\hyperlink{val:voho}{vo'o}}{}{}{}

\dictentry{x5 ello}{}{}{}
{\hyperlink{val:vohu}{vo'u}}{}{}{}

\dictchar{Y}\phantomsection\addcontentsline{toc}{section}{Y}
\dictentry{y}{}{}{}
{\hyperlink{val:ybu}{ybu}}{}{}{}

\dictentry{y (sumti)}{}{}{}
{\hyperlink{val:e}{e}}{}{}{}

\dictentry{y preposicionado}{}{}{}
{\hyperlink{val:ge}{ge}}{}{}{}

\dictentry{y respectivamente}{}{}{}
{\hyperlink{val:fahu}{fa'u}}{}{}{}

\dictentry{yacer}{}{}{}
{\hyperlink{val:vreta}{vreta}}{}{}{}

\dictentry{yo}{}{}{}
{\hyperlink{val:mi}{mi}}{}{}{}

\dictentry{yocto}{}{}{}
{\hyperlink{val:gocti}{gocti}}{}{}{}

\dictentry{yota}{}{}{}
{\hyperlink{val:gotro}{gotro}}{}{}{}

\dictchar{Z}\phantomsection\addcontentsline{toc}{section}{Z}
\dictentry{z}{}{}{}
{\hyperlink{val:zy}{zy}}{}{}{}

\dictentry{zambullirse}{}{}{}
{\hyperlink{val:sfubu}{sfubu}}{}{}{}

\dictentry{zapato}{}{}{}
{\hyperlink{val:cutci}{cutci}}{}{}{}

\dictentry{zarza}{}{}{}
{\hyperlink{val:frambesi}{frambesi}}{}{}{}

\dictentry{zarzamora}{}{}{}
{\hyperlink{val:frambesi}{frambesi}}{}{}{}

\dictentry{zepto}{}{}{}
{\hyperlink{val:zepti}{zepti}}{}{}{}

\dictentry{zeta}{}{}{}
{\hyperlink{val:zetro}{zetro}}{}{}{}

\dictentry{zinc}{}{}{}
{\hyperlink{val:zinki}{zinki}}{}{}{}

\dictentry{zorro}{}{}{}
{\hyperlink{val:lorxu}{lorxu}}{}{}{}

\dictentry{zumaque}{}{}{}
{\hyperlink{val:urci}{urci}}{}{}{}



\end{multicols}


\chapter*{Copyleft}
\addcontentsline{toc}{chapter}{Copyleft}
El lenguaje \textbf{Lojban} está en Dominio Público, y este libro también lo está por el momento.

Este libro fue generado a partir de datos extraidos de la página de jbovlaste (\url{http://jbovlaste.lojban.org/}), a partir del programa \textbf{\ttfamily pyvalcku}, creado por \textbf{Leo Molas}, y se puede encontrar en \url{http://github.com/leosmolas/pyvlacku}, bajo la licencia GPL v3 (\url{http://www.gnu.org/licenses/gpl.html}).
\end{document}
