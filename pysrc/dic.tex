\documentclass[ipa,twoside]{report} 

\usepackage[spanish]{babel}
\usepackage[phonetic]{lexikon}
\usepackage[latin1]{inputenc}
\usepackage{fullpage}
\usepackage{multicol}
\usepackage[colorlinks=true,urlcolor=Violet]{hyperref}
\usepackage{fancyhdr}

\author{Comunidad Lojb�nica}
\title{Diccionario\\Lojban - Castellano\\Castellano - Lojban}

\begin{document}
\setlength{\leftfield}{0.1\textwidth}
\setlength{\rightfield}{0.3\textwidth}
\setcounter{secnumdepth}{-2} 
\maketitle



\tableofcontents

\section{Historial de versiones}

Esta es la versi�n {\bfseries v0.01}, s�lo para m� :)

\section{C�mo usar este diccionario}

En la secci�n \textbf{Lojban - Castellano}, encontrar� la lista con todos los \textsl{valsi} ordenados alfab�ticamente (en el alfabeto de Lojban). Se parecer�n a:

\begin{center}
\begin{minipage}[t]{0.55\textwidth}
	\dictchar{by.}
	\dictentry{bacru}{/ba\esh ru/}{g}{}
	  {}{}{}{$x_{1}$ pronuncia/ dice/ verbaliza/ vocaliza/ hace sonido $x_{2}$ (Ver tambi�n krixa, cusku.)}
\end{minipage}
\end{center}

En la columna de la derecha, se encuentra el \textsl{valsi} que se definir�. Estar� seguido por una de las siguientes letras, que indicar�n el tipo de palabra, seg�n la siguiente lista:

\begin{tabular}{ l l }
  \textit{g} & \textbf{\emph{gismu}}  \\
  \textit{f} & \textbf{\emph{fu'ivla}}  \\
  \textit{l} & \textbf{\emph{lujvo}}  \\
  \textit{c} & \textbf{\emph{cmavo}} \\
  \textit{n} & \textbf{\emph{cmene}} \\
\end{tabular}

Abajo del \textsl{valsi}, entre barras (``/'') se encontrar� la pronunciaci�n de la palabra, escrita en la representaci�n \textbf{IPA}.

En la columna de la derecha est� la definici�n en castellano, con su correspondiente esctructura de lugares.

\newpage

\setlength{\headheight}{25pt}
\fancyhead{}				          % empty out the header
\fancyfoot{}        				  % empty out the footer
\fancyhead[LE,LO]{\rightmark} % left side, odd and even pages
\fancyhead[RE,RO]{\leftmark}  % right side, odd and even pages
\fancyfoot[LE,RO]{\thepage}   % left side even, right side odd
\setlength{\columnseprule}{0.3pt}
\pagestyle{fancy}
\thispagestyle{plain}
\section*{Lojban - Castellano}
\addcontentsline{toc}{section}{Lojban - Castellano}
\label{sec:lojcas}
\begin{multicols}{2}

\include{jbocas}
  
\end{multicols}

\newpage 
\setlength{\leftfield}{0.25\textwidth}
\setlength{\rightfield}{0.25\textwidth}
\thispagestyle{plain}
\section*{Castellano - Lojban}
\addcontentsline{toc}{section}{Castellano - Lojban}
\label{sec:casloj}
\begin{multicols}{2}
%\dictchar{B}
%\dictentry{banana}{}{}{}
%  {\hyperlink{badna}{badna}}{}{}{}
%\dictchar{P}
%\dictentry{pronunciar}{}{}{en el sentido de decir}
%  {\hyperlink{bacru}{bacru}}{}{}{}

\end{multicols}
\end{document}
