\documentclass{article} %notitlepage

\usepackage[spanish]{babel}
\usepackage{lexikon}
%\usepackage[landscape,a4paper]{geometry}
\usepackage[latin1]{inputenc}
\usepackage{fullpage}
\usepackage{multicol}
\usepackage{hyperref}
\usepackage{fancyhdr}
%\usepackage[print,four]{booklet}

\author{Leonardo Molas}
%\def\dictfile{test}
\begin{document}

\setlength{\leftfield}{0.1\textwidth}
\setlength{\rightfield}{0.3\textwidth}
\setlength{\headheight}{25pt}
\fancyhead{}          % empty out the header
\fancyfoot{}          % empty out the footer
\fancyhead[LE,LO]{\rightmark} % left side, odd and even pages
\fancyhead[RE,RO]{\leftmark}  % right side, odd and even pages
\fancyfoot[LE,RO]{\thepage}   % left side even, right side odd
\setlength{\columnseprule}{0.3pt}
\pagestyle{fancy}

\begin{multicols}{2}
	
\pdfbookmark{by.}{}
\dictchar{by.}
\dictentry{bacru}{}{g}{\hypertarget{bacru}{}}
  {}{}{}{$x_{1}$ pronuncia/ dice/ verbaliza/ vocaliza/ hace sonido $x_{2}$ (Ver tambi�n krixa, cusku.)}\markboth{bacru}{bacru}
\dictentry{badna}{}{g}{\hypertarget{badna}{}}
  {}{}{}{$x_{1}$ es una banana/ pl�tano (fruta) de especie/ variedad $x_{2}$}
  
\end{multicols}

\newpage 
\setlength{\leftfield}{0.25\textwidth}
\setlength{\rightfield}{0.25\textwidth}
\begin{multicols}{2}
\dictchar{B}
\dictentry{banana}{}{}{}
  {\hyperlink{badna}{badna}}{}{}{}
\dictchar{P}
\dictentry{pronunciar}{}{}{en el sentido de decir}
  {\hyperlink{bacru}{bacru}}{}{}{}

\end{multicols}
\end{document}
